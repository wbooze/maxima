%latex
\documentclass{article}
\usepackage{fullpage}
\title {Building Algebra Systems by Overloading Lisp: Automatic Differentiation}
%% BASOL ??
\author {Richard Fateman\\
Computer Science\\
University of California\\
Berkeley, CA, USA}

\begin{document}
\maketitle
\begin{abstract}
In an earlier paper \cite{basol} we began a discussion of the use of
overloaded languages for support of computer algebra systems.  Here we
extend that notion to provide a more detailed approach to Automatic
Differentiation or Algorithm Differentiation (AD).

This paper makes three points. 1. It is extremely easy to do express
AD by overloading in Common Lisp.  2. While the resulting program is
not the most efficient approach in terms of run-time, it is quite
small and very general. It also interacts nicely with some other kinds
of generic arithmetic.  3. A more efficient AD compile-time program
generation approach is described as well.
\end{abstract} 

\section{Introduction}

Given a programming language intended to manipulate primarily numbers,
arrays of numbers, and other data objects such as strings, we can
consider extending the language so that a new set of objects,
call them derivative-function pairs.  We will try to use them in
as many contexts as possible, in particular in place of numbers. That is, just
as we can routinely add $2+2$ to get $4$, the extended language should
be able to $x+x$ to get $2 x$.  Tools for such an enhancement makes it
easy to build AD programs. If instead of just representing the function
$x$, we also represent its derivative, namely 1,  we can add it to other
similar objects, and by adding the derivatives, get the derivative of
the sum. This simple idea, pushed to various
extremes, turns out to be rather useful.

Many of the issues related to generic or overloaded arithmetic
have been discussed in an earlier paper \cite{basol}, and so
will not be emphasized here. We instead proceed to discuss
differentiation.


\section{Symbolic Differentiation and AD}

In this section we motivate the ``automatic differentiation'' (AD)
idea to follow. We contrast it with the notion of symbolic differentiation.

{The Lisp programming language has a number of strengths, prominent
among them the ease with which it can be used for prototyping other
programming languages. However, in brief surveys of languages
it is often characterized (or even dismissed) as
a language which is suitable for computing ``symbolic differentiation''.}

In fact, the compact representation for a symbolic differentiation
program is cited as one driving application for the original Lisp
language design.  In a conventional setup for Lisp, $x \sin x \log x
+3$ would be written as {\tt (+ (* x (sin x) (log x)) 3)}\footnote{As
mentioned earlier, there are parsers from infix available.}
%you are uncomfortable with this notation, many parsers from more
%``conventional'' infix notation are available.  The interface may be
%as simple as using a normal Lisp program text, but enclosing such
%infix expressions in markers, for example {\tt
%[x*sin(x)*log(x)+3]}. The output to this form is simple too.}.  
A brief program\footnote{see Appendix 1} can differentiate this with
respect to {\tt x} to get
\begin{verbatim}
(+ (* (* x (sin x) (log x))
      (+ (* 1 (expt x -1)) (* (* (cos x) 1) (expt (sin x) -1))
         (* (* (expt x -1) 1) (expt (log x) -1))))
   0)
\end{verbatim}
an answer which is correct but clumsy in appearance, even if it is
converted to more conventional infix:\\
{\verb|(x*sin(x)*log(x))*(1*x^(-1)+(cos(x)*1)*sin(x)^(-1)+(x^(-1)*1)*log(x)^(-1))+0|}.

A proper CAS would contain a simplification
program (probably much longer than the differentiation program!)
to take the symbolic result and reduce it to simpler terms,
removing multiplications by 1 and additions with 0, perhaps cancelling
factors, etc. resulting in something like $ \log x\,\sin x+\sin
x+x\,\cos x\,\log x $

This differentiation program {\tt d} takes an expression and a symbol, and
computes a derivative.  To be more precise and pedantic, the program
identifies a Lisp symbol say
 {\tt x}
with a corresponding mathematical indeterminate, say $x$, and the Lisp expression
is based on a recipe which encodes  $\sin x$ as  {\tt (sin x)} etc.
In the short course of this program, it does no harm to conflate
these concepts.  In fact, the Lisp program is pretty good this way:
even setting {\tt x} to 3 does not ``confuse'' it into thinking it should
take the derivative with respect to 3, if that had any meaning. Lisp does
not mistake a symbol for its value.

\section{Introduction to Automatic Algorithm Differentiation (AD)}

Automatic Differentiation or AD has an extensive literature of its own,
mostly quite separate from Lisp, and AD provides
features that are quite different from the previously described symbolic
differentiation\footnote{At least initially}.  The mostly-true one-sentence summary:
AD will read the text of a FORTRAN program $f(x)$ that computes some result $k$ and 
will write the text of a new program $g(x)$ that computes a pair: $k$ and $dk/dx$.

But, you may say, that's not really possible, is it? Well, to the
extent possible, AD tries to do it. Philosophically, all non-constant functions
on floating-point objects represent discrete (not continuous) mappings,
and so they have no derivatives. This detail is only one
of several ``in principle'' problems ignored ``in practice.''

For a complete introduction to the topic of Automatic Differentiation
as used in this paper we recommend a visit to the website {\tt
www.autodiff.org}.  Here you can find a collection of links to the AD
standard literature, including AD programs and applications.  There
are also links to recent conference publications and articles.

There are two major variants of AD techniques, ``forward'' and ``reverse''
differentiation.  Although we have programmed both, in this paper
we will discuss only forward, because it fits more nicely into this overloading
concept. For brevity we have name our system ADIL for AD in Lisp.

{\subsection{A brief defense of Forward Differentiation for ADIL}

Consider a space of ``differentiable functions evaluated at a point
c.''  In this space we can represent a ``function $f$ at a point $c$''
by a pair $\langle f(c),~ f'(c)\rangle$.  That is, in this system
every object is a pair: a value $f(c)$ and the derivative with respect
to its argument $D_x f(x)$, evaluated at $c$.  (written as $f'(c)$).

For a start, note that every number $n$ is really a special case of
its own Constant function, $C_n(x)$ such that $C_n(x) = n$ for all
$x$.  $C_3(x)$ is thus $\langle3,~ 0\rangle$.  The constant $\pi$ is
$t_1 = \langle 3.14159265\cdots, 0.0\rangle$, which represents a
function that is always $\pi$ and has zero slope.  The object
$t_2=\langle c,1\rangle $ represents the function $f(x)=x$ evaluated
at $c$. At this point we must be clear that all our functions are
functions of the {\em same variable}, and that furthermore we will be fixing
a point $x=c$ of interest.  It does not make sense to operate
collectively on $\langle f(y),D_y f(y)\rangle$ at $y=a$ and $\langle
g(x),D_x g(x)\rangle$ at $x=b$.\footnote{We have not excluded the possible
encoding of a vector $x$, however.}

For example, sin operating on $t_2$ is the pair $\langle \sin(c),
\cos(c)\rangle$.  In general, $\sin(\langle a, a'\rangle)$ is $\langle
\sin(a)~,~\cos(a) \times a'\rangle$.

We can compute other operations unsurprisingly, as, for example the
sum of two pairs: $\langle a ,a'\rangle + \langle b , b'\rangle =
\langle a+b ~,~a'+b'\rangle.  $ {\em Note, we have abused notation
somewhat: The ``+'' on the left is adding in our pair-space, the ``+''
on the right is adding real numbers. Such distinctions are important
when you write programs!}  Similarly, the product of two pairs in this
space is $\langle a ,a'\rangle \times \langle b , b'\rangle = \langle
a \times b ,~ a\times b' + a' \times b \rangle.  $ This can be
extended to many standard arithmetic operations in programming
languages, at least the differentiable ones\cite{Griewank91}.  AD
implementors seek to find some useful analogy for other operations
which do not have obvious derivatives.  Falling in this category are
most data structure manipulations, calls to external routines, loops,
creating arrays, etc.\footnote{It is a mistake to {\tt declare}
variables to be (say) double-floats, when in the ADIL framework they
will be {\tt df} structures. We are not aware of any other systematic
problems.}  In ADIL, these mostly come free.}

\subsection{AD as Taylor series}

AD technology is not ``finding symbolic derivatives'' but from a CAS
perspective is closer to performing arithmetic  
with truncated Taylor series.  AD converts
a {\em conventional program} computing with conventional scalar values
like $u$ to a {\em new AD program} operating pairs such as $p=\langle
u,~v\rangle$ representing that scalar function $u(t)$ and its
derivative $v(t)$ with respect to an implicit parameter, say $t$ at
some (presumably numeric) point $t=c$ In particular, a constant $k$
looks like $\langle k,~0\rangle$, and the parameter $t$ at the point
$c$ looks like $\langle c ,~1\rangle$. That is, $dt/dt=1$ As a Taylor
series, the pair $p$ represents $ s=u+v\,t+\cdots $.  The equivalence
to computation with Taylor series is perhaps the clearest guidance as
to how to compute with these pairs.  (Generally a CAS Taylor series
program can provide an appropriate result, at least for functions with
continuous derivatives.)  The Taylor expansion of $\cos s$ is $ \cos
u-\sin u\,v\,t+\cdots $ and so $\cos p=\langle \cos u,~-v \times \sin
u\rangle$. Some programming operations have discontinuities; if their
derivatives are needed, some escape must be arranged: perhaps an error
is signalled, or more often a derivative is defined by some pragmatic
justification.

The extension of this AD idea to produce programs for higher
derivatives is straightforward although perhaps tedious. Here is how it could
work: we use triples, e.g. if the second derivative is called $w$: If
$p=\langle u,~v,~w \rangle$, compute the Taylor expansion of each
operator. See, for example, that $\cos p$ yields:
$$ \cos p = \cos u-\sin u\,v\,t-{{\left(\cos u\,v^{2}+2\,\sin
u\,w\right)\,t^{ 2}}\over{2}}+\cdots $$ The result triple consists of
the coefficients of different powers of $t$.  For a given set of
values for $u$, the program computing those coefficients naturally
will need to compute $\sin u$ and $\cos u$ only once.

\subsection{ADIL limitations}
Let us be up-front about several limitations. 

\begin{enumerate}
\item If you have a scientific program written in C or C++ or most
likely, FORTRAN, you will not be initially attracted to a tool that
works on Lisp programs. (There are translators from FORTRAN to
Lisp (f2cl) that have been used for automatic translation of large
libraries: so if there is an additional rationale
to use Lisp for part of the computation it is not out of the question.)
\item The AD computer program's running time should be no more
than a small multiple of the ordinary program.  There is a way
of trading time for space by using ``reverse'' AD, but we will not 
discuss that here.  The overloading scheme is slower (we discuss a faster way, soon).
\item Neat things like arithmetic on exact integers or rationals 
or symbols is not possible in FORTRAN. Neither is recursion (in
FORTRAN 77)
Supplying those features even in cases when they are not needed
may give FORTRAN a speed advantage. If speed is critical, this may
be an issue.  Often speed is thought to be critical even when other
issues dominate the time or cost to a solution\footnote{Benchmarking for speed, even acknowledging
that it is subject to abuse, is still more easily quantifiable than
say, correctness, generality, or modularity. Thus speed 
becomes a proxy for all good
qualities. And anything that does not improve speed is questionable.
Thus we have seen programmers write in ``C'' because it is faster than some
other more convenient language.
This would, logically, drive all programmers to write in assembler, but 
it doesn't.}.

\end{enumerate}

On the other hand, just because we don't expect anyone to
shift from FORTRAN to Lisp to use this package
does not prevent us from discussing the design, especially
since it looks so much nicer than corresponding programs in
other languages. 
%(Some of which
%AD in Maple has all the limitations (and then some), but appears to
%have one package, GRADIENT \cite{maple} and another ADrien
%\cite{adrien} described in publications.  
ADIL has, among several major advantages, the obvious one of not
having to parse its own programs to represent them as data.

\subsection{ADIL implementation}
There are two fundamentally different
approaches to building a forward AD system, and a third that looks plausible,
at least briefly, for Lisp.
\begin{enumerate}
\item We can take a text for a program $P$ in (more-or-less) unchanged form and by
{\em overloading} each of its operations, make $P$ executable in another
domain in which scalars are replaced by pairs.
\item We can transform the source code of $P$ into another program in which
computations are ``doubled'' in a particular way to compute the original
function as well as the requested derivative.
\item A third technique, usually ignored, could be used in Lisp: write
a new version of Lisp's {\tt eval} to do AD. This is not a good idea,
even in Lisp, because most Lisp programs don't use {\tt eval}: they are
compiled into assembler.  We don't want to slow down normal
programs doing normal things by running them through a
{\em any} version of {\tt eval}.
The technique of overloading operators
is conceptually similar to patching the {\tt eval} program anyway, and
so we discard this option.
\end{enumerate}

We provide code for each of the first two approaches.  Because it
inherits all the Lisp facilities not specifically shadowed by AD
variations, the overloading method has the advantage of covering just
about everything one can do in Lisp. The code is structured in a way
that is fairly easy to understand.

Overloading is unfortunately slower in execution compared to
the second method, which tends to run at about the minimal
extra cost necessary: a (small) multiple of the
original code's speed.  But source code transformation is ticklish,
requiring attention to almost every aspect of the language. Thus our
version {\tt dcomp} does not cover as much of Lisp as overloading.
The two techniques can, fortunately, be used together, so that parts
of the code can be transformed for extra speed; these eligible parts are typically
the more costly ones anyway:
expression evaluation of the composition of
built-in functions (or in a specified way, ``friendly'' functions). That
is this code can be compiled in-line.

For the purpose of overloading, we must decide how to wrap up the pair
of function value and derivative.  We choose a {\tt defstruct} named
{\tt df} with two parts, {\tt f} and {\tt d}, though if we wanted
higher derivatives, we might use a vector.  A {\tt df} is easily
distinguished from a conventional scalar, and so methods for
operations like ``+'' and ``*'' can be
overloaded.  (The source-code transformation technique we describe
later does not require these structures; the generated code carries
along near-duplicate names, e.g. if the original program has variable
{\tt v}, we generate {\tt v\_diff\_x} (etc.))

\subsection{Overloading for AD}
In this section we describe in more detail how we overload the
arithmetic operators and let the rest of the language be inherited
from Lisp's standard implementation. (or other variations on generic
arithmetic!)

Here's the beginning
of that package declaration needed for AD. It could be extended easily. (For
example, to add the hyperbolic tangent requires only one line).

\begin{verbatim}
(defpackage :ga ;generic arithmetic for AD
  (:shadow "+" "-" "/" "*" "expt"       ;binary arith
           "=" "/=" ">" "<" "<=" ">="   ;binary comparisons
           "sin" "cos" "tan"            ;... more trig
           "atan" "asin" "acos"         ;... more inverse trig
           "sinh" "cosh" "atanh"        ;... more hyperbolic
           "expt" "log" "exp" "sqrt"    ;... more exponential, powers
           "1-" "1+" "abs"              ;... odds and ends
           )
  (:use :common-lisp))
\end{verbatim}

{The implementation of AD looks like any system that overloads
a generic arithmetic package, shadowing all the built-in arithmetic
with new methods that do the arithmetic on pairs. Comparisons,
however, need some work. We defined {\tt two-arg- } variants for {\tt
df} pairs by ignoring the derivative and looking only at the function
value.  To accomplish this we wrote macro programs so that for
example, the macro-expansion of {\tt (defcomparison /= )}, says all we
need to say about {\tt /=}, the not-equal function. We also used
macro-expansions to set up programs based on the derivative property
for each operator. This looks like a single-argument differentiable
function.  To define all that our {\tt df} handling needs to know
about the $\sin$ function, we must say
no more than {\tt (r sin (cos x))}. The {\tt
r} macro expansion defines the necessary piece of code for the chain
rule.

Other than these one-argument functions we need to write some specific programs for 
{\tt two-arg-+}  etc. 

This AD overload system constitutes about 130 lines of Lisp code other than the
package declarations and some examples.

\subsection{Using ADIL}

Consider the function $f(x)=x  \sin x  \log x + 3$
written in Lisp as\\
 {\tt (defun f(x)(+ (* x (sin x) (log x)) 3))}.\\
Using the syntax in \cite{basol} we could define it by
{\tt \$f(x):=x sin x log x + 3}.
We can evaluate $f$ at a point $x$ where $x$ is 1.23 by
{\tt (f (df 1.23 1.0))} which returns
\begin{verbatim}
<3.2399835288524628d0, 1.2227035>
\end{verbatim}
In this system we have defined {\tt df} as a constructor taking two numbers.
This pair is displayed in angle-brackets.
$p=\langle 1.23,~1.0 \rangle$.
Note that the text of the program {\tt f} has not changed {\tt at all}; 
applying the function to a different type of argument, and running the
program within the {\tt :ga} package is all that is needed.
Of course the meaning of a function like ``+'' within the {\tt :ga}
package is different from the standard ``+'';
in spite of the observation that the text is the same.  (The function {\tt f} can be
invoked from Lisp programs in other packages by calling it by
a fully-qualified name like {\tt ga::f}.)
\medskip

A more interesting program is this one.
\begin{verbatim}
(defun s(x) (if (< (abs x) 1.0d-5) x 
              (let ((z (s (* -1/3 x))))
                (-(* 4 (expt z 3))
                  (* 3 z)))))
\end{verbatim}
Not at all obviously, this computes an approximation to $\sin x$ by
applying an identity recursively. That is, $\sin x$ is a polynomial
$4z^3-3z$ in $z=\sin (-x/3)$, and that for small enough $x$, $\sin x =
x$. Let us try it out by typing {\tt (s (df 1.23 1))}.  We get
\begin{verbatim}
<0.942489, 0.33423734>
\end{verbatim}
This not only provides a value for $\sin(1.23)$
but also computes 0.33423734, the second part of the pair, which happens to be
$\cos(1.23)$.
It appears that ADIL somehow knew that {\tt s} computed $\sin()$ and
therefore it also computed its derivative, $\cos()$. 
ADIL just followed directions:  It computed something that looked like
the derivative of sin only because it ran the {\em derivative of the program
that computed something that looked like sin.}
\medskip

The classic recursive program in Lisp is factorial:\\
{\tt (defun fact(x) (if (= x 1) 1 (* x (fact (1- x)))))}\\
which we modify to\\
{\tt (defun fact(x) (if (= x 1) (df 1 0.422784335098d0) (* x (fact (1- x)))))}\\
whose peculiar base case is now the value one, with a peculiar
derivative.
We do this so as to make it correspond to the 
derivative of the Gamma function\footnote{a continuous
version of factorial well-known among the ``special functions''.}. 
With this form we can provide not only
the factorial but its ``derivative'' (at least at integer points).
\medskip
In these examples we have not modified Lisp syntax at all.  Other parts
not mentioned in the Lisp language, namely parts of the language not overloaded,
are imported without attention or comment. Thus to
compute and print ten $\sin x$ values, we can use {\tt dotimes} as in\\
{\tt (dotimes (i 10) (print (s (df i 1))))}.

\subsection{Newton Iteration, or, Why is AD useful?}

{The point is that if we are given a complicated function
$F:~R\rightarrow R$ arranged as an expression, and all the
sub-functions ``cooperate'' properly,  we can feed in a pair $\langle
c,1\rangle$, representing the expression $x$ and its derivative with
respect to $x$, namely 1, each evaluated at $x=c$.  We get out pairs
$\langle f,f'\rangle = F(\langle c,1\rangle )$ where the latter is the
pair $f=F(c)$, and $f'= D_x(F(x))|_{x=c}$.  and this may be exactly
what we want in an application. These are discussed in papers available
via www.autodiff.org.  Here we take only the simplest and easiest
to motivate example, Newton iteration.

For example, to converge to a root in a Newton iteration for $f(z)=0$
given an initial guess $c_0$ or $t_0= \langle c_0,0\rangle$ , we
compute $ F(t_{i}) = \langle f(t_i), f'(t_i)\rangle .$ Then the next
iteration $t_{i+1} = t_i - f(t_i)/f'(t_i)$.  If this is not
sufficiently accurate we consider repeating with $\langle f,f'\rangle
= F(t_{i+1}),$ etc.

The program is shorter than the explanation.
{
\begin{verbatim} Newton iteration: (ni fun guess) usage: fun is a
;; function of one argument guess is an estimate of solution of
;; fun(x)=0 output: a new guess. (Not a df structure, just a number)

(defun ni (f z) ;one Newton step
  (let* ((pt (if (df-p z) z (df z 1)))  ; make sure init point is a df
         (v (funcall f pt))) ;compute f, f' at pt
   (df-f (- pt (/ (df-f v)(df-d v))))))
\end{verbatim}

As a simple example, consider $f(x)=sin(1+2x)$ or\\
{\tt (defun f(x)(sin(+ 1 (* 2 x))))}.
 then
\begin{verbatim}
(setf h 2.0d0);; just a guess
(setf h (ni 'f h))
(setf h (ni 'f h))
;; h converges to 1.0707963267948966d0
\end{verbatim}

We can write a version of Newton iteration to return the value too.
Then both the residual and the derivative can be taken into account in
testing whether the Newton iteration has converged sufficiently.  That
program would look like:
\begin{verbatim}
(defun ni2 (f z)
  (let* ((pt (if (df-p z) z (df z 1))) ;if z is not a df, make it one
         (v (funcall f pt))) ;compute f, f' at pt
  (values  (df-f (- pt (/ (df-f v)(df-d v)))) ;the next guess
           v))) ; the residual and derivative
\end{verbatim}
%;; A harder test for Newton iteration is this function,
%\begin{verbatim}
%(defun test(x)(+ (* 1/3 x) 1 (sin x))) ;; f(x) = x/3+1+sin x.
%;; which is good if you start close enough to -0.8 or -3.2 or -5.4
%;; but ni fails for most other initial guesses.
%
%\end{verbatim}
Now that we have a program for only one step, we can show a program
that can use it to find the zero. This is a ticklish proposition
because sometimes this iteration does not converge. We have to use
some stopping heuristic.  We could stop when two successive iterations
are close in terms of relative or absolute error, or use some other
measure.

A particularly simple iteration driver program uses {\tt ni2} which
returns the value of the residual which we test. We also
quit with an error message if some count is exceeded.

\begin{verbatim}
(defun run-newt2(f guess &key (abstol 1.0d-8) (count 18)) ;; Solve f=0
  ;; It looks only at the residual.
    (dotimes (i count  ;; do at most count times. failure prints msg
               (error "~%Newton quits after ~s iterations: ~s" count guess))
      (multiple-value-bind
           (newguess v)
           (ni2 f guess)
        (if (< (abs (df-f v)) abstol) 
          (return newguess)
          (setf guess newguess)))))
\end{verbatim}

In this Newton iteration example, our program must, by its nature,
separately get a value and derivative.  We have used three
functions particular to ADIL, namely {\tt df-p} a predicate which will
return true if applied to a {\tt df} object, as well as the two
selection functions, {\tt df-f} and {\tt df-d} for extracting the
value and derivative respectively from a {\tt df} object.

\subsection{What about speed?}
The generic arithmetic (ga) system for ADIL
produces code that is slower than ordinary Lisp code,
especially if we make efforts to optimize 
the ordinary Lisp\footnote{ Note that without
type declarations and compilation, ordinary Lisp already has its own
burden of generic arithmetic.  Short and long (arbitrary length)
integers, single or double floats, rationals, and complex numbers, as
well as all plausible combinations are handled by standard ANSI Common
Lisp. With appropriate instructions to the compiler
Common Lisp arithmetic can be compiled to ``non-generic'' straight-line
floating-point code, comparable to that of other high-level languages.
}.

How much of a speed difference is there?
On a variety of benchmarks involving mostly computations of how to
dispatch-to-the-right-method, using the generic arithmetic for ADIL
seems to be about a factor of 10 over ``normal Lisp'' and perhaps a
factor of 50 over ``optimized'' Lisp (when that Lisp is constrained say,
to double-floats.)
On tests where most of the work is done in subroutines such as
$\sin$ or $\log$, the difference is much less:  the $\log$ routine
takes the same time whether it is called from the {\tt ga} package
or from the {\tt user} package. On parts of tests where the operations
are {\em other than generic arithmetic} such as looping over indexes,
the {\tt ga} programs run at full speed because they are in fact
running identical instructions.

\subsection{Source code transformation for AD}

As mentioned earlier, we wrote another part of ADIL, {\tt dcomp} that
can compile programs, in-line, in a restricted language subset of
Lisp, essentially that of functional-style arithmetic programs.  In
this situation, benchmarking a simple example suggests that the
comparison between the ordinary Lisp for computing a function $f$ and
the ADIL Lisp for computing a function $f$ and its derivative, is
a small factor; we have observed
a typical factor of about two; we expect perhaps a factor of up to 5, judging
from the literature.

This experiment computes $3+z*(4+z)$ where $z=\sin x$. 
The {\tt dcomp}
version is shown with the {\tt defdiff} defining form.
All code is run through a Lisp compiler before timing, and the
reported times are for a run of 10,000 computations.
The first set shows that the code transformation provides function
and derivative values only 2.3 times slower than 
the fastest program we could reasonably expect to provide just the
function value.
\begin{verbatim}

(defun kk(x &aux z)   ;;optimized version
  (declare (double-float x z)(optimize (speed 3)(safety 0)))
  (setf z (sin x))
  (+ 3.0d0 (* z (+ 4.0d0 z))))

(defdiff kz(x)
  (progn (setf z (sin x))
         (+ 3.0d0 (* z (+ 4.0d0 z)))))

(kk 1.2d0) -->    30 ms  /no derivs!
(kz 1.2d0) -->    70 ms  /with derivs!!
\end{verbatim}
In the next example we use only one function,  but define
it in the generic arithmetic package. By calling it on 
different types we get different behavior.
\begin{verbatim}

(defun kk(x &aux z) (setf z (sin x))(+ 3.0d0 (* z (+ 4.0d0 z))))

(kk (   1.2d0)) -->  1523 ms /no derivs!  ;  51 times slower
(kk (df 1.2d0)) --> 40100 ms /with derivs ; 573 times slower

\end{verbatim}
The time spent in the pure overloaded case is substantially higher.
Fortunately the program {\tt kz} could be used instead
of {\tt kk} in writing a presumably more elaborate
computation using the overloading methods.

The body of {\tt kz} can be examined before it is compiled
by tracing the program {\tt dc}. This shows:
\begin{verbatim}
    (lambda (g94)
           "(progn (setf z (sin x)) (+ 3.0d0 (* z (+ 4.0d0 z)))) wrt x"
           (declare (double-float g94))
           (declare (optimize (speed 3) (debug 0) (safety 0)))
           (let ((t102 0.0d0) (f101 0.0d0)
                 (t100 0.0d0) (f99 0.0d0)
                 (t98 0.0d0)  (f97 0.0d0)
                 (z_DIF_x 0.0d0) (t96 0.0d0)
                 (f95 0.0d0))
             (declare (double-float t102 f101 t100 f99 t98 f97 z_DIF_x t96 f95))
             (setf f95 (sin g94))
             (setf t96 (cos g94))
             (setf z f95)
             (setf z_DIF_x t96)
             (setf t98 0.0d0 f97 3.0d0)
             (setf t100 z_DIF_x f99 z)
             (setf t102 0.0d0 f101 4.0d0)
             (setf t102 (+ z_DIF_x t102))
             (setf f101 (+ z f101))
             (setf t100 (+ (* f101 t100) (* t102 f99)))
             (setf f99 (* f101 f99))
             (setf t98 (+ t100 t98))
             (setf f97 (+ f99 f97))
             (df f97 t98))) 
\end{verbatim}
What else can {\tt dcomp} do?  In addition to the usual arithmetic operations
and built-in functions (sin, cos, log, etc.) as declared in the generic arithmetic
package, 
{\tt dcomp} can handle {\tt if, progn, setf}. However,
{\tt dcomp} is not as general as overloading, at least not
as we have implemented it. Functions defined with {\tt defdiff}
take only scalar arguments, not {\tt df} structures. They return only
{\tt df} structures. The derivative is always with respect to the first formal
argument.  Thus {\tt (defdiff f(x y z)(cos (+ x y z)))} is legal, but only
one derivative is computed.

Within the {\tt dcomp} framework, a restricted version of recursion is
possible. Explaining the restriction (or removing it!) led to too much
complexity and was simply taken out.

How does this fit into the generic framework, then? As shown in the timings,
a function like {\tt f} above, running at full speed.

The {\tt dcomp file} is about 290 lines of code. It would not be difficult
to alter the code-generation part of dcomp to produce FORTRAN or C function
subroutines, and so one could argue that we could
do this task, starting and ending with FORTRAN, 
by transmuting the middle of the AD processing into Lisp; we expect that a
the middle part of the code used in other AD tools is, in effect, simulating
what we do in Lisp.

\section{Does it really make sense to use ADIL?}

For fans of Lisp, there is no question that one motivation is to show
off the natural advantage of the language: Lisp provides a natural
representation for programs as data and a natural form for writing
programs that write programs, which is what we do in ADIL. It also has
an unobtrusive object-oriented programming system, only part of which
is used here. We used subsetting of types and dispatched on methods
based on the types of {\em all} the parameters, not just the methods
associated with the type of the ``first argument''.

The code is short, and is in ANSI standard Common Lisp. It should run
without any change in any conforming system. While it is not using the
most obvious idioms of introductory Lisp, that is should not be a
barrier. There are now several excellent books on Common Lisp; some
people have argued that reading them will make you a better program in
any language.

For persons only slightly familiar with Lisp, or whose acquaintance
comes from one of the too-common texts that use the Lisp 1.5 of
1959 as the definition, a glance at the Lisp shows how short
such code can be.  If you are familiar with the complexities of
automatic differentiation when presented in another language,
the relatively brevity should also be observable.

If you care not a whit about Lisp or implementation strategies, you
may prefer to refer to the recently-revised online documentation for ADIFOR
2.0 version D, some 99 pages, and read in detail how programs to be
differentiated must be distinguished from ordinary FORTRAN (77)
programs.  Using ADIFOR requires declaring and marking program
variables, adjusting numerous parameters, perhaps revising the FORTRAN
in order to obey various restrictions, and following detailed
guidelines on the use of these programs.

The flexibility one gets is apparent by looking at the code in which
Lisp macro-definitions have made the addition of new derivative
information simple. For example, to insert a new rule for
differentiation of $\tanh x$ one adds {\tt "tanh"} to the {\tt shadow}
list, and execute\\
 {\tt (r tan (expt (cosh x) -2))}.\\ 
By comparison,
large and monolithic systems are relatively rigid, and require some
effort to port, extend or optimize: the original designers must
anticipate the spectrum of possible choices, perhaps freezing some
choices, and then describe all the variants in detail. The user must
then read the details, and hope there are no bugs; the user is
unlikely to be able, in any case, to repair them.  Some of the
restrictions of the large systems seem to be arbitrary--perhaps a
small point, but it did not occur to us to have to exclude recursive
functions, although recursion is a feature lacking in ADIFOR.

Finally, we wish to point out that Lisp is perfectly adequate for
expressing numeric computation; sophisticated compilers are
available for generating efficient code.
%*******************************
{\section{Composing Overloads}


In our earlier paper we illustrated
 overloading of {\em several types simultaneously}. We repeat them
here for coherence, but with the emphasis on {\tt df}.

In these examples we show that combining special
operations {\em can be} as simple as composing them.
Here is a recursive definition of Legendre polynomials of the first kind.

\begin{verbatim}
(defun p(m x)(cond ((= m 0) 1)
                   ((= m 1) x)
                   (t (*  (/ 1 m) (+ (* (1- (* 2 m)) x (p (1- m) x))
                                     (* (- 1 m) (p (- m 2) x)))))))
\end{verbatim}
Here are some possible ways of using it in the generic arithmetic package.
\begin{verbatim}
: (p 3 1/2)              ;exact value of legendre[3,1/2]
-7/16
: (p 3 0.5)              ;single-float value
-0.4375
: (p 3 (df 0.5 1.0))     ;single-float value, as well as derivative
<-0.4375, 0.375>
: (p 3 (df 0.5d0 1.0))   ;double-float value and derivative
<-0.4375d0, 0.375d0>
: (p 3 (df 1/2 1))       ;exact value and derivative
<-7/16, 3/8>
: (p 3 \%x)              ;exact value in terms of x
(1/3)*((5*x)*((1/2)*((-1)+(3*x)*x))+(-2)*x)
: (simp (p 3 \%x))       ;exact value, somewhat simplified
(1/3)*((5/2)*x*((-1)+3*x^2)+(-2)*x)
: (simp (p 3 (df \%x 1)));exact value and derivative, simplified
<(1/3)*((5/2)*x*((-1)+3*x^2)+(-2)*x), 
 (1/3)*((-2)+15*x^2+(5/2)*((-1)+3*x^2))>
: (ratexpand (p 3 (df \%x 1))) 
<(1/2)*((-3)*x+5*x^3), (1/2)*((-3)+15*x^2)>
                         ;same answer, using canonical rational expansion
\end{verbatim}

There are a number of kinds of arithmetic overloads mentioned in
our earlier paper; this paper has discussed in detail one of
the more unusual ones.

{\section {Conclusion}

Generic arithmetic can be easily supported in Common Lisp.  We show
how it can support Automatic Differentiation.  These results are not surprising, though
we hope that our presentation has some novelties, in particular
the brevity and effectiveness of the program.  What is perhaps
surprising is that no one has written this paper earlier, since the
Lisp language is such a fine host for this effort.
}
\section*{Appendix 1}

First we display a seven line Lisp differentiation program (similar to many
others written over the years) that is distinguished
by brevity.  This one was posted on a Lisp newsgroup by Pisin Bootvong,
recently, and makes use of the Common Lisp object system (CLOS)
and {\tt destructuring-bind} nicely:

\begin{verbatim}
(defmethod d ((x symbol) var) (if (eql x var) 1 0))
(defmethod d ((x number) var) 0)
(defmethod d ((expr list) var)
   (destructuring-bind (op e1 e2) expr
     (case op
       (+ `(+ ,(d e1 var) ,(d e2 var)))
       (* `(+ (* ,(d e1 var) ,e2) (* ,e1 ,(d e2 var))))))) 
\end{verbatim}

Here's a more elaborate, but still short Lisp program with more
capabilities and greater extensibility \cite{fateman98}.

\begin{verbatim}
(defun d(e v)(if(atom e)(if(eq e v)1 0)
               (funcall(get(car e)'d #'(lambda (e v) `(d ,e, v))) e v)))

(defmacro r(op s)`(setf(get ',op 'd) ;;define a rule to diff operator op!
                (compile() '(lambda(e v)
                             (let((x(cadr e)))
                               (list '* (subst x 'x ',s) (d x v)))))))
(r cos (* -1 (sin x)))
(r sin (cos x))
(r exp (exp x))
(r log (expt x -1)) ;; etc, 
(setf(get '+ 'd)  ;; rules for +, *, handle n args, not just 1
  #'(lambda(e v) `(+,@(mapcar #'(lambda(r)(d r v))(cdr e)))))
(setf(get '* 'd)
  #'(lambda(e v) `(*,e(+,@(mapcar #'(lambda(r) `(*,(d r v)(expt,r -1)))(cdr e))))))
(setf(get 'expt 'd)
  #'(lambda(e v) `(*,e,(d `(*,(caddr e)(log,(cadr e)))v))))
\end{verbatim}
Other programming languages, especially ones with a ``functional''
approach, can usually handle this task nicely, but the major issue
(and one not addressed here) is simplification of the result.  Trying
to write a simplifier adds substantially to the programmer's
burden. The differentiation program in a computer algebra system like
Macsyma is much larger, not only because it handles a larger class of
functions, and trades code-size for speed, not generating such naive
forms, and simplifying along the way.

Another program whose specifications seem superficially like the
previous one does not build any list structure.  It assume that the
result of interest is the value of the derivative at a point, not its
symbolic representation.  Thus the {\tt d} function takes another
argument, the point {\tt p}.  It returns the derivative of the {\tt
expr} with respect to the {\tt var} at the point {\tt p}.  We are no
longer returning lists.  Note that we now have to evaluate expressions
involving the variable, for which we have defined the {\tt val}
method.  Although such a program can be written entirely using the
skeleton of the previous few programs, we illustrate a different
approach using generic programming (a handle on object-oriented
programs) supported in Common Lisp.

\begin{verbatim}
(defmethod d ((x symbol) var p) (if (eql x var) 1 0))
(defmethod d ((x number) var p) 0)
(defmethod d ((expr list) var p)
   (destructuring-bind (op e1 e2) expr
     (case op
       (+ (+ (d e1 var p) (d e2 var p)))
       (* (+ (* (d e1 var p) (val e2 var p)) (* (val e1 var p) (d e2 var p))))))) 

(defun val(expr var p)(funcall  `(lambda (,var) ,expr) p))
\end{verbatim}
This program's {\tt case} statement would have to be expanded
for other two-argument functions, and would also need to be
altered for one-argument functions like sin and cos.
\section*{Appendix 3 DF}
\begin{verbatim}
;;; Automatic Differentiation code for Common Lisp (ADIL)
;;; using overloading, forward differentiation.

;; code extended by Richard Fateman, November, 2005

(defpackage :df				;derivative and function package
  (:use  :cl)
  (:shadowing-import-from 
   :ga
   "+" "-" "/" "*" "expt"		;binary arith
   "=" "/=" ">" "<" "<=" ">="		;binary comparisons
   "sin" "cos" "tan"			;... more trig
   "atan" "asin" "acos"			;... more inverse trig
   "sinh" "cosh" "atanh"		;... more hyperbolic
   "expt" "log" "exp" "sqrt"		;... more exponential, powers
   "1-" "1+" "abs" "incf" "decf"
   "numerator" "denominator"	  
   "tocl" 	  )
   (:export "df")
   )
  
(require "ga" )
(provide "df" )
(in-package :df)

;;  structure for f,d: f is function value, and d derivative, default 0
;; df is also the constructor for an object with 2 components, f and d..
(defstruct (df (:constructor df (f &optional (d 0)))) f d )
(defmethod print-object ((a df) stream)(format stream "<~a, ~a>" (df-f a)(df-d a)))

;;comparison of df objects depends only on their f parts (values)
;;extend the generic arithmetic for this purpose

(defmacro defcomparison (op)
  (let ((two-arg (intern (concatenate 'string "two-arg-" 
				      (symbol-name op))    :ga ))
        (cl-op (tocl op)))
    `(progn
        ;; only extra methods not in ga are defined here.
      (defmethod ,two-arg ((arg1 df) (arg2 df))    (,cl-op (df-f arg1)(df-f arg2)))
      (defmethod ,two-arg ((arg1 number) (arg2 df))(,cl-op arg1(df-f arg2)))
      (defmethod ,two-arg ((arg1 df) (arg2 number))(,cl-op (df-f arg1) arg2 ))
      (compile ',two-arg)
      (compile ',op)
      ',op)))

(defcomparison >)
(defcomparison =)
(defcomparison /=)
(defcomparison <)
(defcomparison <=)
(defcomparison >=)

;; extra + methods specific to df
(defmethod ga::two-arg-+ ((a df) (b df))    (df  (+ (df-f a)(df-f b))
						 (+ (df-d a)(df-d b))))
(defmethod ga::two-arg-+ ((b df)(a number))   (df  (+ a (df-f b))    (df-d b)))
(defmethod ga::two-arg-+ ((a number)(b df))   (df  (+ a (df-f b))    (df-d b)))

;;extra - methods

(defmethod ga::two-arg-- ((a df) (b df))    (df  (- (df-f a)(df-f b))
						 (- (df-d a)(df-d b))))
(defmethod ga::two-arg-- ((b df)(a number))   (df  (-  (df-f b) a)    (df-d b)))
(defmethod ga::two-arg-- ((a number)(b df))   (df  (- a (df-f b))    (df-d (- b))))

;;extra * methods
(defmethod ga::two-arg-* ((a df) (b df)) 
  (df  (* (df-f a)(df-f b))
       (+ (* (df-d a) (df-f b)) (* (df-d b) (df-f a)))))
(defmethod ga::two-arg-* 
    ( (b df)(a number)) (df  (* a (df-f b))  (* a (df-d b))))
(defmethod ga::two-arg-* 
    ((a number) (b df)) (df  (* a (df-f b))  (* a (df-d b))))

;; extra divide methods
(defmethod ga::two-arg-/  ((u df) (v df)) 
  (df  (/ (df-f u)(df-f v))
	    (/ (+ (* -1 (df-f u)(df-d v))
			  (* (df-f v)(df-d u)))
		    (* (df-f v)(df-f v)))))
(defmethod ga::two-arg-/  ((u number) (v df)) 
  (df  (/ u (df-f v))
	    (/ (* -1  (df-f u)(df-d v))
		    (* (df-f v)(df-f v)))))
(defmethod ga::two-arg-/  ((u df) (v number)) 
  (df  (/ (df-f u) v)
	    (/ (df-d u) v)))

;; extra expt methods
(defmethod ga::two-arg-expt  ((u df) (v number))
  (df  (expt (df-f u) v)
       (* v (expt (df-f u) (1- v)) (df-d u))))

(defmethod ga::two-arg-expt ((u df) (v df))
  (let* ((z (expt (df-f u) (df-f v)));;z=u^v
	 (w;;   u(x)^v(x)*(dv*LOG(u(x))+du*v(x)/u(x)) = z*(dv*LOG(u(x))+du*v(x)/u(x))
	  (* z (+
		(* (log (df-f u))	;log(u)
		   (df-d v))		;dv
		(/ (* (df-f v)(df-d u)) ;v*du/ u
		   (df-f u))))))
    (df  z  w)))

(defmethod ga::two-arg-expt ((u number) (v df))
  (let* ((z (expt u (df-f v))) ;;z=u^v
	 (w   ;;    z*(dv*LOG(u(x))
	  (* z (* (log u) ;log(u)
			 (df-d v)))))
    (df  z  w)))

;; A rule to define rules.

(defmacro r (op s)
  `(progn
     (defmethod ,op ((a df)) ;; the chain rule d(f(u(x)))=df/du*du/dx
       (df  (,op (df-f a))
	    (* (df-d a) ,(subst '(df-f a) 'x s))))
     (defmethod ,op ((a number)) (,(tocl op) a))))

;; add as many rules as you can think of here.
;; should insert them in the shadow list too.
(r sin (cos x))
(r cos (* -1 (sin x)))
(r asin (expt (+ 1 (* -1 (expt x 2))) -1/2))
(r acos (* -1 (expt (+ 1 (* -1 (expt x 2))) -1/2)))
(r atan (expt (+ 1 (expt x 2)) -1))
(r sinh (cosh x))
(r cosh (sinh x))
(r atanh (expt (1+ (* -1 (expt x 2))) -1))
(r log (expt x -1))
(r exp (exp x))
(r sqrt (* 1/2 (expt x -1/2)))
(r 1-  1)
(r 1+  1)
(r abs x);; hm.

;; some examples of functions that can be differentiated, including recursive factorial
;; A factorial with "right" derivative at x=1 to match gamma function

 (defun fact(x) (if (= x 1) (df 1 0.422784335098d0) (* x (fact (1- x)))))

;; stirling approximation to factorial
(defun stir(n) (* (expt n n) (exp (- n))(sqrt (*(+ (* 2 n) 1/3) 3.141592653589793d0))))

(defun ex(x)(ex1 x 1 15))
(defun ex1(x i lim)  ;; generate Taylor summation exactly for exp()
  (if (= i lim) 0 (+ 1 (* x  (ex1 x (+ i 1) lim)(expt i -1)))))

\end{verbatim}
\section*{Appendix 3: Some examples and timing}

Some timing results suggest that we lose about a factor of 10 in speed
by running in a generic arithmetic system.  That is, even if we are
using (say) floating point numbers. This is the cost of the extra
checking, just in case.  Under these conditions, if we actually use
{\tt df} numbers, it is perhaps a factor of two additionally slower.

That means the big slowdown is in generic over the built-in (but still generic)
Lisp arithmetic that allows floats, doubles, rationals, bignums.

If we compile with declarations for fixnum types among the lisp built-ins,
we improved by another 30 percent.
\begin{verbatim}
#|

;; slowmul is a program that multiplies by doing repeated adds.
;; slowmulx is the same program with optimization turned on

(defun slowmul(x y ans)(if (= x 0) ans (slowmul (1- x) y (+ y ans))))

(defun slowmulx(x y ans)
  (declare (fixnum x y ans) ;; or (double-float x y ans)
	   (optimize (speed 3)(safety 0)(debug 0)))
	   (if (= x 0) ans (slowmulx (1- x) y (+ y ans))))

(time (dotimes (i 10)(slowmul  1000000  2  0))) ;compiled 19.38s in :ga
(time (dotimes (i 10)(slowmul  1000000  2  (df 0)))) ;    24.45s in :ga
(time (dotimes (i 10)(slowmul  1000000  2  0))) ;compiled   .37s in :user
(time (dotimes (i 10)(slowmulx 10000.0d0 2.0d0 0.0d0)));    .38s in :user declared doubles
(time (dotimes (i 10)(slowmulx 1000000 2 0)))        ;      .29s in :user declared fixnums

;; this is an interesting function that numerically computes (sin x)

(defun s(x) (if (< (abs x) 1.0d-5) x 
	      (let ((z (s (* -1/3 x))))
		(-(* 4 (expt z 3))
		  (* 3 z)))))

 (s  (df 1.23d0 1.0))   computes both sin and cos of 1.23.

;; runtimes vary as to whether the program is compiled in generic arithmetic (ga)
;; or the ordinary arithmetic (user) package.

(time (dotimes (i 100) (s (df 100000.0d0 1))))   :ga 40ms
(time (dotimes (i 100) (s  100000.0d0  )))       :ga 20ms
(time (dotimes (i 100) (s  100000.0d0  ))) :user 10ms av over more runs
(time (dotimes (i 100) (ss 100000.0d0  ))) :user  8ms av over more runs

;; ss is same as s, but compiled to run faster and to work only on double-floats.

(defun ss(x) (declare (double-float x)
		      (optimize (speed 3)(safety 0) (debug 0)))
       (if (< (abs x) 1.0d-5) x (let ((z (ss (* #.(/ -1 3.0d0) x))))
					   (-(* 4.0d0 (expt z 3))
					     (* 3.0d0 z)))))

;; for additional on-line comments, tests, etc. see
;; www.cs.berkeley.edu/~fateman/generic/df.lisp
\end{verbatim}


\section*{Appendix 4 Dcomp}
\begin{verbatim}
;; code for dcomp.lisp is about 291 lines, available on request.
;; a more serious version generating code for backwards diff
;; version of AD is also available.
\end{verbatim}
%Code for compiling expressions via AD .
%Code for backward differentiation.

\begin{thebibliography}{99}

\bibitem{SICP}
H. Abelson, G. Sussman, {\em Structure and Interpretation of
Computer Programs,} 2nd edition 1996. MIT Press. Full text
available on-line.
http://mitpress.mit.edu/sicp/full-text/sicp/book/book.html
\bibitem{fateman98}
 ``A Short Note on Short Differentiation Programs in Lisp, and
a Comment on Logarithmic Differentiation,''
{\em SIGSAM Bulletin Volume 32, Number 3}, Sept., 1998,pp.~2-7.
\verb|www.cs.berkeley.edu/~fateman/papers/deriv.pdf|. 

\bibitem{extrat},
R.~Fateman and Tak Yan,
``Computation with the Extended Rational Numbers
and an Application to Interval Arithmetic'' 1994//
\verb|http://www.cs.berkeley.edu/~fateman/papers/extrat.pdf|.

\bibitem{basol}
R. Fateman, ``Building Algebra Systems by Overloading Lisp'',
(submitted for publication)
\bibitem{bdiff}
R. Fateman, ``Backward Automatic Differentiation in Lisp,''
(in progress)

\bibitem{Griewank89}
A. Griewank,  ``On Automatic
Differentiation,'' in M. Iri \& K. Tanabe (Eds.) {\em MATHEMATICAL PROGRAMMING,}
Kluwer Academic Publishers, 1989, pp.~83-107.

\bibitem{Griewank91} A. Griewank and G. Corliss (eds), 
{\em Automatic Differentiation
       of Algorithms}, SIAM, 1991. esp. paper by L. Rall
\bibitem{perlis}
A. Perlis, Itturiaga, R., Standish, T. ``A Definition of Formula Algol,''
in {\em Proc. Symposium on Symbolic and Algebraic Manipulation of the ACM},
 Washington, D.C., March, 1966.
\bibitem{jenks}
Richard D. Jenks and Robert S. Sutor,
{\em {\sc AXIOM:} The Scientific Computation System,}
Springer-Verlag, 1992.
\bibitem{paip}
P. Norvig, {\em Paradigms of Artificial Intelligence Programming},
Morgan Kaufmann, 1992.

\bibitem{tobey}
R.~G.~Tobey,
``Experience with FORMAC algorithm design'',
Comm.~ACM 9 no.~8 (1966) p.~589--597.
%[@article{365773,
% author = {R. G. Tobey},
% title = {Experience with FORMAC algorithm design},
% journal = {Commun. ACM},
% volume = {9}, number = {8}, year = {1966}, issn = {0001-0782}, pages = {589--597},
% doi = {http://doi.acm.org/10.1145/365758.365773},
% publisher = {ACM Press},
% address = {New York, NY, USA},
% }]
\bibitem{wyatt}
W. T. Wyatt, Jr. D.W. Lozier, and D.J. Orser,
``A Portable Extended Precision Package and Library with Fortran
Precompiler''
 ACM Trans on Math Software ISSN:0098-3500 
1976, vol. 2 no. 3 p. 209--231.
)
\end{thebibliography}
\end{document}
;;;;;;;;;;;;;;;;;;;;;;;;;;;;;;;;;;;;;;;;leftover stuff

\begin{verbatim}
(defun run-newt1(f guess &key (count 18)) ;; Solve f=0
  ;; This program looks only at the sequence of guesses.
  ;; though if the derivative goes to 0, we are stuck
  ;; in the newton step.
  (let ((guesses (list guess))
        (reltol #.(* 100 double-float-epsilon))
        (abstol #.(* 100 least-positive-double-float)))
    (dotimes (i 6)(push (ni f (car guesses))
                        guesses)) ;; make 6 iterations.
    (incf count -6)
    (cond ((< (abs (car guesses)) abstol)
           (car guesses))
          ((< (abs(/ (-(car guesses)(cadr guesses))(car guesses))) reltol)
           (car guesses))
          ((<= count 0)
           (format t "~%newt1 failed to converge; guess =~s" (car guesses))
           (car guesses))
          (t (run-newt1 f (car guesses) count)))))
;; to run this, write
;; (run-newt1 'sin 3.0d0) ;; converges to pi.
;; (run-newt1 'test 3.0d0);; fails to converge
\end{verbatim}


                                        

;; Automatic Differentiation code for Common Lisp
;; Richard Fateman, November, 2005
;; This is all provided in the context of a Generic Arithmetic Package.
;; Package based in part on code posted on comp.lang.functional newsgroup by
;; Ingvar Mattsson <ing...@cathouse.bofh.se> 09 Oct 2003

(defpackage :ga ;generic arithmetic
  (:shadow "+" "-" "/" "*" "expt"       ;binary arith
           "=" "/=" ">" "<" "<=" ">="   ;binary comparisons
           "sin" "cos" "tan"            ;... more trig
           "atan" "asin" "acos"         ;... more inverse trig
           "sinh" "cosh" "atanh"        ;... more hyperbolic
           "expt" "log" "exp" "sqrt"    ;... more exponential, powers
           "1-" "1+" "abs"
           )
  (:use :common-lisp))

(in-package :ga)

;;  df structure for f,d: f is function value, and d derivative, default 0
(defstruct (df (:constructor df (f &optional (d 0)))) f d )

;;  print df structures with < , >
(defmethod print-object ((a df) stream)(format stream "<~a, ~a>" (df-f a)(df-d a)))

;;function ARITHMETIC-IDENTITY: When fed an operator and a non-nil
;;argument, it returns a value for unary application. What does (+ a) mean?
;;A nil arg means there were NO operands. What does (+ ) mean.
;;It is used only by defarithmetic, which in turn helps 
;; us to write out + * - / of arbitrary number of args.

(defmacro arithmetic-identity (op arg)
  `(case ,op
    (+ (or ,arg 0))
    (- (if ,arg (two-arg-* -1 ,arg) 0))
    (* (or ,arg 1))
    (/ (or ,arg (error "/ given no arguments")))
    (expt (or ,arg (error "expt given no arguments")))
    (otherwise nil))) ;binary comparisons?

(defun tocl(n)                          ; get corresponding name in cl-user package
  (find-symbol (symbol-name n) :cl-user))

(defmacro defarithmetic (op)
  (let ((two-arg
           (intern (concatenate 'string "two-arg-" (symbol-name op))
                   :ga ))
        (cl-op (tocl op)))
    `(progn
      (defun ,op (&rest args)
         (cond ((null args) (arithmetic-identity ',op nil))
               ((null (cdr args))(arithmetic-identity ',op (car args)))
               (t (reduce (function ,two-arg)
                          (cdr args)
                          :initial-value (car args)))))
      (defgeneric ,two-arg (arg1 arg2))
      (defmethod ,two-arg ((arg1 number) (arg2 number))
        (,cl-op arg1 arg2))
      (compile ',two-arg)
      (compile ',op)
      ',op)))

(defarithmetic +) ;; defines some of + programs. See below for more
(defarithmetic -)
(defarithmetic *)
(defarithmetic /)
(defarithmetic expt)

;; defcomparison helps us generate numeric comparisons: 2 args
;; and n-arg. CL requires they be monotonic. 
;; That is in Lisp, (> 3 2 1) is true.

(defun monotone (op a rest)(or (null rest)
                               (and (funcall op a (car rest))
                                    (monotone op (car rest)(cdr rest)))))

(defmacro defcomparison (op)
  (let ((two-arg (intern (concatenate 'string "two-arg-" 
                                      (symbol-name op))    :ga ))
        (cl-op (tocl op)))
    `(progn
        (defun ,op (&rest args)
         (cond ((null args) (error "~s wanted at least 1 arg"  ',op))
               ((null (cdr args)) t) ;; one arg e.g. (> x) is true
               (t (monotone (function ,two-arg)
                            (car args)
                             (cdr args)))))
           
      (defgeneric ,two-arg (arg1 arg2))
      (defmethod ,two-arg ((arg1 number) (arg2 number)) (,cl-op arg1 arg2))
      (defmethod ,two-arg ((arg1 df) (arg2 df))    (,cl-op (df-f arg1)(df-f arg2)))
      (defmethod ,two-arg ((arg1 number) (arg2 df))(,cl-op arg1(df-f arg2)))
      (defmethod ,two-arg ((arg1 df) (arg2 number))(,cl-op (df-f arg1) arg2 ))
      (compile ',two-arg)
      (compile ',op)
      ',op)))

(defcomparison >) ;;provides ALL the comparison methods
(defcomparison =)
(defcomparison /=)
(defcomparison <)
(defcomparison <=)
(defcomparison >=) ;; that's all

;; extra + methods specific to df
(defmethod ga::two-arg-+ ((a df) (b df)) 
   (df  (cl:+ (df-f a)(df-f b))
             (cl:+ (df-d a)(df-d b))))
(defmethod ga::two-arg-+ ((b df)(a number)) 
  (df  (cl:+ a (df-f b))    (df-d b)))
(defmethod ga::two-arg-+ ((a number)(b df)) 
  (df  (cl:+ a (df-f b))    (df-d b)))

;;extra - methods
(defmethod ga::two-arg-- ((a df) (b df)) 
   (df  (cl:- (df-f a)(df-f b))
        (cl:- (df-d a)(df-d b))))
(defmethod ga::two-arg-- ((b df)(a number)) 
  (df  (cl:-  (df-f b) a)    (df-d b)))
(defmethod ga::two-arg-- ((a number)(b df)) 
  (df  (cl:- a (df-f b))    (df-d (cl:- b))))

;;extra * methods
(defmethod ga::two-arg-* ((a df) (b df)) 
  (df  (cl:* (df-f a)(df-f b))
       (cl:+ (cl:* (df-d a) (df-f b)) (cl:* (df-d b) (df-f a)))))
(defmethod ga::two-arg-* 
    ((b df)(a number))  (df  (cl:* a (df-f b))  (cl:* a (df-d b))))
(defmethod ga::two-arg-* 
    ((a number) (b df)) (df  (cl:* a (df-f b))  (cl:* a (df-d b))))

;; extra divide methods
(defmethod ga::two-arg-/  ((u df) (v df)) 
  (df  (cl:/ (df-f u)(df-f v))
            (cl:/ (cl:+ (cl:* -1 (df-f u)(df-d v))
                          (cl:* (df-f v)(df-d u)))
                    (cl:* (df-f v)(df-f v)))))
(defmethod ga::two-arg-/  ((u number) (v df)) 
  (df  (cl:/ u (df-f v))
            (cl:/ (cl:* -1  (df-f u)(df-d v))
                    (cl:* (df-f v)(df-f v)))))
(defmethod ga::two-arg-/  ((u df) (v number)) 
  (df  (cl:/ (df-f u) v)
            (cl:/ (df-d u) v)))

;; extra expt methods
(defmethod ga::two-arg-expt  ((u df) (v number))
  (df  (cl:expt (df-f u) v)
            (cl:* v (cl:expt (df-f u) (cl:1- v)) (df-d u))))
(defmethod ga::two-arg-expt ((u df) (v df))
  (let* ((z (cl:expt (df-f u) (df-f v))) ;;z=u^v
         (w   ;;   u(x)^v(x)*(dv*log(u(x))+du*v(x)/u(x)) 
              ;;   = z*(dv*log(u(x))+du*v(x)/u(x))
          (cl:* z (cl:+
                   (cl:* (cl:log (df-f u)) ;log(u)
                         (df-d v))      ;dv
                    (cl:/ (cl:* (df-f v)(df-d u)) ;v*du/ u
                          (df-f u))))))
    (df  z  w)))
(defmethod ga::two-arg-expt ((u number) (v df))
  (let* ((z (cl:expt u (df-f v))) ;;z=u^v
         (w   ;;    z*(dv*LOG(u(x))
          (cl:* z (cl:* (cl:log u) ;log(u)
                         (df-d v)))))
    (df  z  w)))

;; A rule to define rules, a new method for df, the old method for numbers
(defmacro r (op s)
  `(progn
    (defmethod ,op ((a df))
           (df  (,(tocl op) (df-f a))
                     (,(tocl '*) (df-d a) ,(subst '(df-f a) 'x s))))
    (defmethod ,op ((a number)) (,(tocl op) a))))

;; Add rules for every built-in numeric program.
;; Must insert the name in the shadow list too.
;; This is just a sampler.
(r sin (cos x))  ;; provides EVERYTHING ADIL needs about sin
(r cos (* -1 (sin x)))
(r asin (expt (+ 1 (* -1 (expt x 2))) -1/2))
(r acos (* -1 (expt (+ 1 (* -1 (expt x 2))) -1/2)))
(r atan (expt (+ 1 (expt x 2)) -1))
(r sinh (cosh x))
(r cosh (sinh x))
(r atanh (expt (1+ (* -1 (expt x 2))) -1))
(r log (expt x -1))
(r exp (exp x))
(r sqrt (* 1/2 (expt x -1/2)))
(r 1-  1)
(r 1+  1)
(r abs x);; hm does this matter?

(defun re-intern(s p) ;; move expression to :ga package
  (cond ((or (null s)(numberp s)) s)
        ((symbolp s)(intern (symbol-name s) p))
        (t(cl-user::cons (re-intern (car s) p)
                         (re-intern (cdr s) p)))))

(defun cl-user::deval(r x p &optional (dx 1))
  (ga::deval (ga::re-intern r :ga)
             (ga::re-intern x :ga) p dx))

;; factorial with "right" derivative at x=1 to match gamma function

 (defun fact(x) (if (= x 1) (df 1 0.422784335098d0) (* x (fact (1- x)))))

;; an integer (arbitrary! ) factorial
(defun ifact(x) (if (= x 1) x  (* x (ifact (1- x)))))

;; compare fact, ifact to Stirling's approximation to factorial

(defun stir(n) (* (expt n n) (exp (- n))
                  (sqrt (*(+ (* 2 n) 1/3) 3.141592653589793d0))))

(defun ex(x)(ex1 x 1 15))

(defun ex1(x i lim)  ;; generate Taylor summation for exp()
  (if (= i lim) 0 (+ 1.0d0 (* x  (ex1 x (+ i 1) lim)(expt i -1)))))

'(defun ex1(x i lim)  ;; generate Taylor summation exactly for exp()
  (if (= i lim) 0 (+ 1 (* x  (ex1 x (+ i 1) lim)(expt i -1)))))

;; what you say depends on which package is current in your
;; top-level read-eval-print loop
;; in :user (setf one (ga::df  1  1))
;; in :user (deval '(dotimes (i 10)(print (fact (+ i one)))) 'x 'irrelevant)

;; in :ga   (setf one (df  1  1))
;; in :ga   (dotimes (i 10)(print (fact (+ i one))))

;;;;;;;;;;;;;;;;;;;;;;;;;;;;;;;;;;;;;;;;;;

#|Some timing results suggest that we lose about a factor of 10 in
speed just using generic arithmetic, even if we are using (say)
floating point numbers, if we are doing mostly calls and returns, but
only trivial arithmetic. Under these conditions, if we actually use df
numbers, it is perhaps a factor of two further.

The big slowdown here is in generic over the built-in (but still generic)
Lisp arithmetic that allows floats, doubles, rationals, bignums.

If we compile with declarations for fixnum types among the lisp built-ins,
we improved by another 30 percent.


(time (dotimes (i 10)(slowmul  1000000  2  0))) ;compiled 19.38s in :ga
(time (dotimes (i 10)(slowmul  1000000  2  (df 0)))) ;    24.45s in :ga
(time (dotimes (i 10)(slowmul  1000000  2  0))) ;compiled   .37s in :user
(time (dotimes (i 10)(slowmulx 10000.0d0 2.0d0 0.0d0)));    .38 in :user declared doubles
(time (dotimes (i 10)(slowmulx 1000000 2 0)))        ;      .29 in :user declared fixnums


(defun slowmul(x y ans)(if (= x 0) ans (slowmul (1- x) y (+ y ans))))

(defun slowmulx(x y ans)
  (declare (fixnum x y ans) ;; or (double-float x y ans)
           (optimize (speed 3)(safety 0)(debug 0)))
           (if (= x 0) ans (slowmulx (1- x) y (+ y ans))))


|#
;; this is an interesting function that numerically computes (sin x)

(defun s(x) (if (< (abs x) 1.0d-5) x 
              (let ((z (s (* -1/3 x))))
                (-(* 4 (expt z 3))
                  (* 3 z)))))

#| (s  (df 1.23d0 1.0))   computes both sin and cos of 1.23.

(time (dotimes (i 100) (s (df 100000.0d0  1))) ) :ga 40ms
(time (dotimes (i 100) (s  100000.0d0  ))) :ga 20 ms
(time (dotimes (i 100) (s  100000.0d0  ))) :user 10ms av over more runs

;; we can compile this to run faster..

(defun ss(x) (declare (double-float x)
                      (optimize (speed 3)(safety 0) (debug 0)))
       (if (< (abs x) 1.0d-5) x (let ((z (ss (* #.(/ -1 3.0d0) x))))
                                           (-(* 4.0d0 (expt z 3))
                                             (* 3.0d0 z)))))

(time (dotimes (i 100) (ss 100000.0d0  ))) :user  8ms av over more runs
|#


;; newton iteration  (ni fun guess)
;; usage:  fun is a function of one arg.
;;         guess is an estimate of solution of fun(x)=0
;; output: a new guess. (Not a df structure, just a number

(defun ni (f z) ;one newton step
  (let* ((pt (if (df-p z) z (df z 1)))  ; make sure init point is a df
         (v (funcall f pt))) ;compute f, f'
    (df-f (- pt (/ (df-f v)(df-d v))))))


(defun ni2 (f z) ;one newton step, give more info
  (let* ((pt (if (df-p z) z (df z 1)))
         (v (funcall f pt)))            ;compute f, f' at pt
    (format t "~%v = ~s" v)
  (values
   (df-f (- pt (/ (df-f v)(df-d v)))) ;the next guess
    v)))


(defun run-newt1(f guess &optional (count 18)) ;; Solve f=0
  ;; Look only at the guesses.
  ;; though if the derivative goes to 0, we are stuck
  ;; in the newton step.
  (let ((guesses (list guess))
        (reltol #.(* 100 double-float-epsilon))
        (abstol #.(* 100 least-positive-double-float)))
    (dotimes (i 6)(push (ni f (car guesses))
                        guesses)) ;; make 6 iterations.
    (incf count -6)
    (cond ((< (abs (car guesses)) abstol)
           (car guesses))
          ((< (abs(/ (-(car guesses)(cadr guesses))(car guesses))) reltol)
           (car guesses))
          ((<= count 0)
           (format t "~%newt1 failed to converge; guess =~s" (car guesses))
           (car guesses))
          (t (run-newt1 f (car guesses) count)))))

(defun run-newt2(f guess &key (abstol 1.0d-8) (count 18)) ;; Solve f=0
  ;; Looks only at the residual.
    (dotimes (i count 
               (format t  "~%Newton iteration not convergent after ~s iterations: ~s" count guess))
      (multiple-value-bind
          (newguess v)
          (ni2 f guess)
        (if (< (abs (df-f v)) abstol) (return newguess)
          (setf guess newguess)))))

;; dcomp

\begin{verbatim}
;;; -*- Mode: Lisp; Syntax: Common-Lisp; -*-
;; dcomp, use instead of /or with/  generic3.lisp for ADIL, automatic differentiation
;; Richard Fateman 11/2005
;; structure for f,d: f is function value, and d derivative, default 0

(defstruct (df (:constructor df (f &optional (d 0)))) f d )
(defmethod print-object ((a df) stream)(format stream "<~a, ~a>" (df-f a)(df-d a)))

(defvar *difprog* nil);; the text of the program
(defvar *bindings* nil) ;; bindings used for program

(defun dcomp (f x p) 
  ;; the main program
  (let ((*difprog* nil)
        (*bindings* nil))
    (multiple-value-bind
        (val dif)
       (dcomp1 f x p)
      (emit `(values ,val ,dif))
      `(let ,*bindings* 
         ,(format nil "~s wrt ~s" f x)
         ,@(nreverse *difprog*)))))

;; Assist in compilation of a function f and its derivative at a point x=p
;; dcomp1 changes *bindings* and *difprog* as it munches on the expression f.

(defun dcomp1 (f x p)
  (declare (special *v*))
  (cond((atom f)
        (cond ((eq f x) (values p 1.0d0))
              ((numberp f)(values f 0.0d0))
              (t (let ((r (make-dif-name f x)))
                   (push r  *bindings*)
                   (emit `(setf ,r 0.0d0)); unnecessary? already bound to 0.0d0
                   (push (cons f r) *lvkd*)
                   (values f r)))))
       (t       (let* ((op (first f))
                       (program (get op 'dopcomp)));otherwise, look up the operator's d/dx
                  (if program
                      (funcall program (cdr f) x p)
                    (error "~% dcomp cannot do ~s" f);; for now
                    )))))
                   
(defun gentempb(r)
  (push (gentemp r) *bindings*)   (car *bindings*))

(defun emit(item)(unless (and (eq (car item) 'setf)
                              (eq (cadr item)(caddr item)))
                   (push item *difprog*)))

;; simple unary  f(x) programs.
;;We define them all with the same template.

(defmacro dr(op s)
  (setf(get op 'dopcomp) ;;define a rule for returning v, d for dcomp
    (dr2 op (treefloat s))))

(defun dr2 (op s) ;; a rule to make rules for dcomp
   `(lambda(l x p)              
                  (cond ((cdr l)
                         (error "~&too many arguments to ~s: ~s" ',op l))
                        (t (let ((v (gentempb 'f))
                                 (d (gentempb 't)))
                             (multiple-value-bind
                                 (theval thedif)
                                 (dcomp1 (car l) x p)
                               (emit (list 'setf v (list ',op theval)))
                               (emit (list 'setf d (*chk thedif (subst theval 'x ',s)))))
                             (values v d))))))

(defun treefloat(x) ;; convert all numbers to double-floats
                    (cond ((null x) nil)
                          ((numberp x)(coerce x 'double-float))
                          ((atom x) x)
                          (t (mapcar #'treefloat x))))


;; Here's how the rule definition program works.
;; Set up rules.
(dr tan (power (cos x) -2)) 
(dr sin (cos x))
(dr cos (* -1 (sin x)))
(dr asin (power (+ 1 (* -1 (power x 2))) -1/2))
(dr asin (power (+ 1 (* -1 (power x 2))) -1/2))
(dr acos (* -1 (power (+ 1 (* -1 (power x 2))) -1/2)))
(dr atan (power (+ 1 (power x 2)) -1))
(dr sinh (cosh x))
(dr cosh (sinh x))
(dr log (power x -1))
(dr exp (exp x))
(dr sqrt (* 1/2 (power x -1/2)))
;; etc etc

;; Next, functions of several arguments depending on x 
;; These include + - * / expt, power

(setf (get '* 'dopcomp)  '*rule)
(setf (get '+ 'dopcomp)  '+rule)
(setf (get '/ 'dopcomp)  '/rule)
(setf (get '- 'dopcomp)  '-rule)
(setf (get 'power 'dopcomp) 'power-rule)
(setf (get 'expt 'dopcomp)  'expt-rule)

(defun +rule (l x p)
  (let ((valname (gentempb 'f))
        (difname (gentempb t)))
   (multiple-value-bind (v d) (dcomp1 (car l) x p)
                   (emit `(setf ,difname ,d ,valname ,v)))
    (dolist (i (cdr l))(multiple-value-bind (v d) (dcomp1 i x p)
                   (emit `(setf ,difname ,(+chk d difname)))
                   (emit `(setf  ,valname ,(+chk v valname)))))
    (values valname difname)))

(defun -rule (l x p)
  (let ((valname (gentempb 'f))
        (difname (gentempb t)))
    (cond ((cdr l)
           (multiple-value-bind (v d) (dcomp1 (car l) x p)
             (emit `(setf ,difname ,d ,valname ,v)))
           (dolist (i (cdr l))(multiple-value-bind (v d) (dcomp1 i x p)
                                (emit `(setf ,difname (- ,difname ,d)))
                                (emit `(setf  ,valname (- ,valname ,v ))))) )
          ;; just one arg
          (t (multiple-value-bind (v d) (dcomp1 (car l) x p)
               (emit `(setf ,difname (- ,d) ,valname (- ,v))))))
    (values valname difname)))

;; (- x y z) means (+ x (* -1 y) (* -1  z)).  (- x) means (* -1 x)

(defun *rule (l x p)
  (let ((valname (gentempb 'f))
        (difname (gentempb t)))
    (multiple-value-bind (v d) (dcomp1 (car l) x p)
                   (emit `(setf ,difname ,d ,valname ,v)))
    (dolist (i (cdr l))(multiple-value-bind (v d) (dcomp1 i x p)
                   (emit `(setf ,difname ,(+chk (*chk v difname)
                                                (*chk d valname))))
                   (emit `(setf 
                                ,valname ,(*chk v valname)))))
    (values valname difname)))

(defun /rule (l x p)
  (let ((vname (gentempb 'f))
                  (dname (gentempb t)))
              (multiple-value-bind (v d) (dcomp1 (car l) x p)
                (emit `(setf ,dname ,d ,vname ,v)))
              (multiple-value-bind (v d) (dcomp1 (cadr l) x p)
                (emit `(setf ,dname (/ (- (* ,v ,dname)(* ,vname ,d))
                                       (* ,v ,v))
                             ,vname (/ ,vname ,v))))
              (values vname dname)))

(defun power-rule (l x p)
  (cond ((cddr l) (error "~&too many arguments to power: ~s" l))
        ;; now we assume that everything is a-ok
        ;; i.e. power has only two arguments, the second argument
        ;; which is the power to be raised to is independent of x
        ;; so we can use for sqrt, cube-root etc.
        (t  (let ((vname (gentempb 'f))
                  (dname (gentempb t)))
              (multiple-value-bind (v d) (dcomp1 (car l) x p)
                ;;; We could use +chk and *chk in the following emissions
                ;;; but they are really inessential.
                (emit `(setf ,vname (power ,v ,(cadr l))))
                (emit `(setf ,dname (* ,d (power ,v (1- ,(cadr l))) ,(cadr l)))))
              (values vname dname)))))

(defun expt-rule (l x p)
  "this is the general power rule where the exponent could be an 
    arbitrary function of x"
  (cond ((cddr l) (error "~&too many arguments to expt : ~s" l))
        ((numberp (cadr l))(power-rule l x p))
        ;; we had z = (expt f g)
        ;; z' = z*(g*log f)' = z*(g*f'/f + g'*log f)
        (t  (let ((vname (gentempb 'f))
                  (dname (gentempb t)))
              (multiple-value-bind (v d) (dcomp1 (cadr l) x p)
                (emit `(setf ,vname ,v ,dname ,d)))
              (multiple-value-bind (v d) (dcomp1 (car l) x p)
                (emit `(setf ,dname ,(+chk `(/ ,(*chk vname d) ,v)
                                           (*chk dname `(log ,v)))))
                (emit `(setf ,vname (expt ,v ,vname)))
                (emit `(setf ,dname ,(*chk vname dname))))
              (values vname dname)))))


;; the next two programs, "optimize": 
;; generated code so as to not add 0 or mult by 0 or 1
(defun +chk (a b)(cond ((and (numberp a)(= a 0)) b)
                       ((and (numberp b)(= b 0)) a)
                       (t `(+ ,a ,b))))

(defun *chk (a b)(cond ((and (numberp a)(= a 1)) b)
                       ((and (numberp b)(= b 1)) a)
                       ((and (numberp a)(= a 0)) a)
                       ((and (numberp b)(= b 0)) b)
                       (t `(* ,a ,b))))

;;;; this is the main program for compiling

(defun dc (f x &optional (otherargs nil)) 
  ;; produce a program, p(v) ready to go into the compiler to
  ;; compute f(v), f'(v), returning result as a structure, a df
  (let ((*difprog* nil)
        (*bindings* nil)
        (*v* (gentemp "g")))
    (declare (special *v*))
    (multiple-value-bind
        (val dif)
        (dcomp1 f x *v*)
      (emit `(df ,val ,dif))
      `(lambda (,*v* ,@otherargs)
                 ,(format nil "~s wrt ~s" f x)
         ;;; comment out these declares  if you prefer v's type to be unknown
          (declare (double-float ,*v*))
          (declare (optimize (speed 3)(debug 0)(safety 0)))
          ;    (assert (typep ,*v* 'double-float))
         (let ,(mapcar #'(lambda (r)(list r 0d0)) *bindings*)
         (declare (double-float ,@*bindings*))
           ,@(nreverse *difprog*))))))

;; A plausible way to use  dc  is as follows:

(defmacro defdiff (name arglist body) ;; put the pieces together
  (progn
    (setf (get name 'defdiff) name)
  (let ((r (dc body (car arglist) (cdr arglist)))) ;; returns a df.
    `(defun  ,name , (cadr r) ,@(cddr r)))))

;; usage (defdiff f (x) (* x (sin x)))
;; Then you can call (f 3.0d0)

;; a few more pieces to allow inside defdiff: if, progn, setf.
;; we could try for a few more.  Oh, > < = etc. work "automagically."

(defun if-rule (l x p)
  (let ((valname (gentempb 'f))
        (difname (gentempb t)))
    (emit `(multiple-value-setq
               (,valname ,difname)
             ;; assume only variables, not derivatives in condition
             (if ,(subst p x (car l))
                 (funcall (function ,(dc (cadr l) x)) ,p)
               (funcall (function ,(dc (caddr l) x)) ,p))))
    (values valname difname)))

(defun progn-rule(l x p)
  (mapc #'(lambda (k)(dcomp1 k x p)) (butlast l))
  (dcomp1 (car (last l)) x p))

(defun setf-rule(l x p)
  (if (not(symbolp (car l)))(error "dcomp cannot do setf ~s" (car l)))
  (multiple-value-bind (v d)
      (dcomp1 (cadr l) x p)
      (emit `(setf ,(car l) ,v))
      (let ((dname (make-dif-name (car l) x)))
        (push dname *bindings*)
        (emit `(setf ,dname ,d)))
      (values v d)))

(defun make-dif-name(s x) ;s is a symbol. make a new one like s_dif_x
  (intern (concatenate 'string (symbol-name s) "_DIF_" (symbol-name x))))

(setf (get 'if 'dopcomp)     'if-rule)
(setf (get 'progn 'dopcomp)  'progn-rule)
(setf (get 'setf 'dopcomp)   'setf-rule)

\end{verbatim}


{\subsection*{A Tangent}
Shouldn't AD (whether of Lisp or not) be widely used since it appears
to be so useful?  Our understanding is that getting computational
scientists to adopt the relatively unfamiliar technique of AD at all,
rather than using finite differences or programming a separate
derivative algorithm, already presents a barrier.  There are simpler
(but generally far less accurate and perhaps slower) ways of
approximating derivatives of ``programs'' with finite differences.
And further complicating {\em our} particular ``sales pitch'' is the
implicit assumption we make that the computation to be differentiated
was originally written in Lisp, or that people are willing to see
their programs translated (perhaps automatically) into Lisp.  For a
variety of reasons people are far more likely to wish to differentiate
FORTRAN.

The computational scientists may be attracted to AD only after the
having written large programs (essentially treated as black
boxes). AD comes into the picture in using these programs as modules
in an optimization tool, or for sensitivity analysis. The optimization
tools then may require separate modules to compute values for
functions and derivatives.  So a typical example might be a
``function'' {\tt F} from computational fluid dynamics that is defined
as a FORTRAN program (or a C program).  The function might, for
example, represent a solution method for a differential equation. The
goal is to produce the moral equivalent of {\tt FPRIME}, or a function
that produces the pairs referred to earlier.

That's what the AD people have been trying to support,
ADIFOR, for FORTRAN,  ADOL-C for C or C++, etc.  

It would be wrong to think that the AD programs for FORTRAN, C, C++
generally work on ``black box'' programs in those languages,
automatically.  They may reduce the programming effort to take a
program and produce a ``derivative'' program considerably, say
reducing the time from a year to a week.  A visit to the previously
mentioned http://www.autodiff.org website has links to some
applications illustrating the level of human effort to build
automation tools, document them for others to use, and then the
continued effort and partnerships needed to use them.  While the goal
has been to differentiate nearly any program written in FORTRAN or C,
it seems that some attention is required for success.

As for our own tools, we could, oddly enough, convert FORTRAN to
Common Lisp using a program {\tt f2cl} and try to differentiate that, and
we could easily print out equivalent FORTRAN or C (etc.).

However, to keep this paper brief, we start and end with the Lisp; the
integration of tools in Lisp is quite good, even if you are addicted
to graphical interactive development environments. We ordinarily
convert the AD code into assembly language without ever leaving the
Lisp system.

There may be additional issues arising in ADIL if it enjoys wider use,
but we are confident that the structure we have set out is consistent
with solving the forward-differentiation AD task for Lisp.  The major
barrier may be convincing a computational scientist to write serious
numerical code in Lisp.  (or to continue to work on the code when the
FORTRAN is translated to Lisp).

End of tangent.
}

