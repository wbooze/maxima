% file fft.tex
%
% Maxima by Example, Ch. 11, Fast Fourier Transform Tools
%   Copyright 2009 by Edwin L. Woollett
%   I release this file under the terms of the GNU General Public License.
%
%   Edwin L Woollett, July,Aug, 2009
%   woollett@charter.net
%   http://www.csulb.edu/~woollett
%   last edit  08-11-09 with Maxima 5.19.0
%   history: feb 08, april may 08, August 09
%
% edit with Notepad++,  http://notepad-plus.sourceforge.net/uk/site.htm ,
% then load into LED, http://www.latexeditor.org/ ,  for latexing
% latex by miktex 2.7 , http://miktex.org/ , -->  intro.dvi  
\documentclass[11pt]{article}
\usepackage[dvips,top=1.5cm,left=1.5cm,right=1.5cm,foot=1cm,bottom=1.5cm]{geometry}
\usepackage{times,amsmath,amsbsy,graphicx,fancyvrb,url}
\usepackage[usenames]{color}
%\definecolor{MyDarkBlue}{rgb}{0,0.08,0.45}
\definecolor{mdb}{rgb}{0.1,0,0.55}
\newcommand{\tcdb}{\textcolor{mdb}}
\newcommand{\tcbr}{\textcolor{BrickRed}}
\newcommand{\tcb}{\textcolor{blue}}
\newcommand{\tcr}{\textcolor{red}}
\urldef\tedhome\url{ http://www.csulb.edu/~woollett/  }
\urldef\tedmail\url{ woollett@charter.net}
%1.  this is for maxima code: red framed bold, footnotesize 
\DefineVerbatimEnvironment%
   {myVerbatim}%
   {Verbatim}%
   {fontfamily=courier,fontseries=b,fontsize=\footnotesize ,frame=single,rulecolor=\color{BrickRed}}
\DefineVerbatimEnvironment%
   {myVerbatim1}%
   {Verbatim}%
   {fontfamily=courier,fontseries=b,fontsize=\scriptsize ,frame=single,rulecolor=\color{BrickRed}}
%2.  this is for blue framed bold 
\DefineVerbatimEnvironment%
   {myVerbatim2}%
   {Verbatim}%
   {fontfamily=courier,fontseries=b,frame=single,rulecolor=\color{blue}}
\DefineVerbatimEnvironment%
   {myVerbatim2s}%
   {Verbatim}%
   {fontfamily=courier,fontseries=b,fontsize=\small,frame=single,rulecolor=\color{blue}}
\DefineVerbatimEnvironment%
   {myVerbatim2f}%
   {Verbatim}%
   {fontfamily=courier,fontseries=b,fontsize=\footnotesize,frame=single,rulecolor=\color{blue}}
% 3.  this is for black framed  bold
\DefineVerbatimEnvironment%
   {myVerbatim3}%
   {Verbatim}%
   {fontfamily= courier, fontseries=b, frame=single}
% 4.  this is for no frame bold
\DefineVerbatimEnvironment%
   {myVerbatim4}%
   {Verbatim}%
   {fontfamily=courier, fontseries=b}
% 6.  for defaults use usual verbatim
\newcommand{\mv}{\Verb[fontfamily=courier,fontseries=b]}
\newcommand{\mvs}{\Verb[fontfamily=courier,fontseries=b,fontsize=\small]}
\newcommand{\mvf}{\Verb[fontfamily=courier,fontseries=b,fontsize=\footnotesize]}

\renewcommand{\thefootnote}{\ensuremath{\fnsymbol{footnote}}}
%%%%%%%%%%%%%%%%%%%%%%%%%%%%%%%%%%%%%%%%%%%%%%%%%%%%%%%%%%%%%%%%%%%%%%%%
%   title page
%%%%%%%%%%%%%%%%%%%%%%%%%%%%%%%%%%%%%%%%%%%%%%%%%%%%%%%%%%%%%%%%%%%%%%

\title{ Maxima by Example:\\ Ch.11: Fast Fourier Transform Tools
            \thanks{This is a live document. This version uses \textbf{Maxima 5.19.0}.
			 Check \; \textbf{ \tedhome } \; for the latest version of these notes. Send comments and
			 suggestions for improvements to \textbf{\tedmail} } }


\author{Edwin L. Woollett}
\date{\today}
%%%%%%%%%%%%%%%%%%%%%%%%%%%%%%%%%%%%%%%%%
%          document
%%%%%%%%%%%%%%%%%%%%%%%%%%%%%%%%%%%%%%%%%%
\begin{document}
%
%\small
\maketitle
\tableofcontents
\numberwithin{equation}{section}
\newpage
\subsubsection*{Preface}
\begin{myVerbatim2s} 
COPYING AND DISTRIBUTION POLICY    
This document is part of a series of notes titled
"Maxima by Example" and is made available
via the author's webpage http://www.csulb.edu/~woollett/
to aid new users of the Maxima computer algebra system.	
	
NON-PROFIT PRINTING AND DISTRIBUTION IS PERMITTED.
	
You may make copies of this document and distribute them
to others as long as you charge no more than the costs of printing.	

These notes (with some modifications) will be published in book form
eventually via Lulu.com in an arrangement which will continue
to allow unlimited free download of the pdf files as well as the option
of ordering a low cost paperbound version of these notes.
\end{myVerbatim2s}	
\smallskip
\noindent \tcbr{Feedback from readers is the best way for this series of notes
  to become more helpful to new users of Maxima}.
\tcdb{\emph{All} comments and suggestions for improvements will be appreciated and
  carefully considered}.
\smallskip
\begin{myVerbatim2s}
LOADING FILES
The defaults allow you to use the brief version load(fft) to load in the
Maxima file fft.lisp.
To load in your own file, such as qfft.mac (used in this chapter),
using the brief version load(qfft), you either need to place 
qfft.mac in one of the folders Maxima searches by default, or
else put a line like:
  
file_search_maxima : append(["c:/work3/###.{mac,mc}"],file_search_maxima )$
  
in your personal startup file maxima-init.mac (see Ch. 1, Introduction
to Maxima for more information about this).

Otherwise you need to provide a complete path in double quotes,
as in load("c:/work3/qfft.mac"),
 
We always use the brief load version in our examples, which are generated 
using the Xmaxima graphics interface on a Windows XP computer, and copied
into a fancy verbatim environment in a latex file which uses the fancyvrb
and color packages.
\end{myVerbatim2s} 
\smallskip
\begin{myVerbatim2s}
  Maxima, a Computer Algebra System.
  Version 5.19.0 (2009) using Lisp GNU Common Lisp (GCL) GCL 2.6.8 
  (aka GCL).  http://maxima.sourceforge.net/
\end{myVerbatim2s}   
\small
The homemade function \mv|fll(x)| (first, last, length) is used to return the first and last
  elements of lists (as well as the length), and is defined in the Ch.1 utility
  file  \mv|mbe1util.mac|. We will include a reference to this definition when working with lists.\\
  
\noindent This function is defined by:
\begin{myVerbatim2s}
fll(x) := [first(x),last(x),length(x)]$
declare(fll,evfun)$
\end{myVerbatim2s}
The author would like to thank the Maxima developers for their friendly help via 
  the Maxima mailing list.
\normalsize
\newpage
\setcounter{section}{11}
\pagestyle{headings}
\subsection{Examples of the Use of the Fast Fourier Transform Functions \textbf{fft} and \textbf{inverse\_fft}}
We discuss the use of Maxima's fast Fourier transform package
  \mv|fft.lisp|, which has been rewritten (effective with version 5.19)
  to allow list input and output (rather than being restricted to array input 
  and output).\\
  
\noindent We first present five simple examples of using Maxima's fast Fourier
  transform functions \textbf{fft} and \textbf{inverse\_fft}.
We then discuss our notation and the utility functions available in the Ch.11
  file \textbf{qfft.mac}.
We then present a derivation of the discrete Fourier transformation pairs, using
  Maxima's conventions.
\subsubsection{Example 1: FFT Spectrum of a Monochromatic Signal}    
We load in the Maxima package \mv|fft.lisp| and also
  our own package \mv|qfft.mac|.
\begin{myVerbatim}
(%i1) load(fft);
(%o1)  C:/PROGRA~1/MA89DF~1.0/share/maxima/5.19.0/share/numeric/fft.lisp
(%i2) load(qfft);
(%o2)                          c:/work3/qfft.mac
\end{myVerbatim}
Our first example uses the simplest possible signal which still contains some
  information, a signal having one intrinsic frequency.  
We assume the signal is  $\mathbf{F(t) = \boldsymbol{\cos(6\,\pi \,t)}}$.
This signal has an \textcolor{blue}{angular frequency} $\mathbf{\boldsymbol{\omega = 6\,\pi = 2\,\pi \,f}}$
  where $\mathbf{f}$ is the \tcbr{frequency} in Hertz (ie., in $\mathbf{sec^{-1}}$).
Thus the frequency $\mathbf{f = 3 \,sec^{-1}}$.
We bind the symbol \textbf{e} to this signal expression for later use:
\begin{myVerbatim}
(%i3) e : cos ( 6*%pi*t )$
\end{myVerbatim}  
Let \mv|ns| be the (\tcb{even}) number of function samples, and \mv|fs| be the
  \tcbr{sampling frequency}.\\
  
\noindent The \textbf{qfft} package function \textbf{nyquist(ns, fs)} computes
  the \tcb{time interval} between signal samples \mv|dt = 1/fs|, 
  the \tcbr{Nyquist integer} \, \mv|knyq = ns/2|,
  the \tcb{frequency resolution} \mv|df = fs/ns|, 
  and the \tcbr{Nyquist frequency}\\
  \mv|fnyq = knyq * df = fs/2|.\\
  
\noindent The product of the \tcbr{time interval} \mv|dt| between samples of the signal 
  and the \tcb{frequency resolution} \mv|df| is always the inverse of the total number of
  signal samples \mv|ns|:
\begin{myVerbatim2}
            dt * df = 1 / ns
\end{myVerbatim2} 
We select \mv|fs| and \mv|ns| to satisfy \tcbr{two conditions}.
The only intrinsic frequency of the signal is $\mathbf{f0 = 3 \,s^{-1}}$.
\textbf{First} we need $\mathbf{fs > 2\,f0}$ which means that $\mathbf{fs > 6\,s^{-1}}$.
\textbf{Secondly} we need the frequency resolution \mv|df| to satisfy
$\mathbf{df < f0}$, which will be satisfied if we choose \mv|df| such that 
  $\mathbf{f0 = 3 \, df}$, or \mv|df = 1 hertz|.
But then \mv|df = fs / ns = 1|, or \mv|fs = ns|, so the
  first condition becomes $\mathbf{ns > 6}$, but we also need
  \mv|ns| to be an even number of the form $\mathbf{2^{m}}$ for some
  integer $\mathbf{m}$,  so we choose \mv|ns = 8|.\\

\noindent We then bind both \mv|ns| and \mv|fs| to the value \mv|8|
  and assign \mv|dt| (the sampling interval) to the first element of
  the list produced by the package function \mv|nyquist( ns, fs )|, which
   also prints out \mv|dt|, \mv|knyq|, \mv|fnyq|, and \mv|df|.
\begin{myVerbatim}
(%i4) ( ns : 8, fs : 8 )$
(%i5) dt : first (nyquist (ns,fs));
sampling interval dt =  0.125 
Nyquist integer knyq =  4 
Nyquist freq fnyq =  4.0 
freq resolution df =  1.0 
(%o5)                                0.125
\end{myVerbatim} 
\newpage
\noindent We then use the package function \mv|sample(expr, var, ns, dvar)| to generate a list of floating
  point numbers as the expression \mv|expr| is evaluated \mv|ns| times, generating
  the list (with \mv|dvar| being the time step \mv|dt| here)\\
  \mvs|[F(0),F(dt),F(2*dt),...,F((ns-1)*dt )]| holding the values
  \mvs|F(m*dt)|, for \mvs|m = 0,1,2,...,ns-1|.
More details of the syntax of the \mv|qfft.mac| package functions can
  be found in  Section \ref{syntax}. 
Recall that we have just bound \mv|ns| and \mv|dt| to numerical values
  and that the expression \mv|e| depends on the variable \mv|t|. 
\begin{myVerbatim}
(%i6) flist : sample (e,t,ns,dt);
(%o6) [1.0, - 0.707, - 1.83691E-16, 0.707, - 1.0, 0.707, 5.51073E-16, - 0.707]
\end{myVerbatim}
We see that elements three, \mv| F(2*dt)|, and seven, \mv| F(6*dt)|, are tiny 
  numbers of the order of the default floating point errors, and are 
  numerically equivalent to zero.\\
 
\noindent We first make a plot of \textbf{both} the given function (the signal) 
  and a list of points \mv|[m*dt, F(m*dt)]| constructed using the package function \mv|vf (flist, dt)|.
Recall that the signal is first sampled at \mv|t = 0|.
\begin{myVerbatim}
(%i7) tflist : vf (flist,dt);
(%o7) [[0, 1.0], [0.125, - 0.707], [0.25, - 1.83691E-16], [0.375, 0.707], 
           [0.5, - 1.0], [0.625, 0.707], [0.75, 5.51073E-16], [0.875, - 0.707]]
\end{myVerbatim}
We let \mv|tmax| be the ``sampling time'', the number of signal samples
  times the time interval between samples.
\begin{myVerbatim}
(%i8) tmax : ns*dt;
(%o8)                                 1.0
(%i9) plot2d([e ,[discrete,tflist]], [t,0,tmax],
             [style,[lines,3],[points,3,2,1]],
              [legend,false])$
\end{myVerbatim}
%\newpage
%\noindent The resulting plot looks like:
\smallskip
\begin{figure} [h]  
   \centerline{\includegraphics[scale=.8]{ch11p1.eps} }
	\caption{ $\mathbf{F(t) = \boldsymbol{\cos(6\,\pi \,t)}}$ with Sample Points }
\end{figure} 

%plot2d([e ,[discrete,tflist]], [t,0,tmax],
%             [style,[lines,5],[points,4,2,1]],
%              [gnuplot_preamble,"set nokey;"],
%              [psfile,"ch11p1.eps"])$

\newpage
\noindent Now that we have looked at both the signal and sample signal values (on the same
  plot), we next look at the \textbf{fast Fourier frequency spectrum} implied by this single frequency 
  signal sample.
We first define \mv|glist| to be the fast fourier transform of \mv|flist|
\begin{myVerbatim}
(%i10) glist : fft (flist);
(%o10) [- 2.34662E-17, 1.11022E-16 - 3.30546E-17 %i, 
1.52656E-16 %i - 4.59227E-17, 0.5 - 1.18171E-16 %i, 1.15312E-16, 
2.16746E-16 %i + 0.5, - 1.52656E-16 %i - 4.59227E-17, 
5.55112E-17 - 6.55197E-17 %i]
(%i11) fchop (%);
(%o11)             [0.0, 0.0, 0.0, 0.5, 0.0, 0.5, 0.0, 0.0]
\end{myVerbatim}
  in which we have used Maxima's fast fourier transform function \textbf{fft}.
We then used our package function \textbf{fchop} to set tiny floating point numbers to zero.  
The \textbf{qfft} package function \mv|current_small()| prints out the
  current setting of the \mv|qfft.mac| parameter \mv|_small%| used by \textbf{fchop}.
\begin{myVerbatim}
(%i12) current_small()$
     current default small chop value =  1.0E-13  
\end{myVerbatim}
\small
The first element of \mvs|glist| corresponds to \mvs|k = 0|, and the list \mvs|glist|
  contains the values of the fast Fourier transform  amplitudes \mvs|G(k*df)| for
  \mvs|k = 0, 1, 2, ..., ns -1|, where \mvs|df| is the
  frequency resolution (for this example, \mvs|df = 1 hertz|).  
We call the value \mvs|G(k*df)| the ``fast Fourier transform amplitude''
  corresponding to the frequency \mvs|f = k*df|.\\

\noindent If we look at the chopped version, we see that the first non-zero
  fast fourier transform amplitude occurs at element four, which corresponds
  to \mvs|k = 3|, which corresponds to the frequency \mvs|f = k*df = 3 * 1 = 3 hertz|.  
In a later section exploring the basic ideas of the fast Fourier transform, we explain
  why all the usable spectrum information is contained in the interval \mvs|k = 0|
  to \mvs|k = knyq = 4|.\\
  
\noindent To make a simple point plot of the fourier amplitudes, we first use the
  package function \mvs|kg| to make a list of the points \mvs|[k, G(k*df)]| out
  to the Nyquist integer \mvs|knyq|.
Since we want real numbers for a plot, this function takes the 
  absolute value of each Fourier amplitude and also chops tiny numbers.
\begin{myVerbatim1}
(%i13) kglist : kg (glist);
(%o13)        [[0, 0.0], [1, 0.0], [2, 0.0], [3, 0.5], [4, 0.0]]
(%i14) plot2d ( [discrete, kglist],[x,0,5],[y,0,0.8],
                  [style,[points,5,1,1]],[xlabel,"k"],
                  [ylabel," "],[gnuplot_preamble,"set grid;"])$
\end{myVerbatim1}
%\newpage
which produces the simple plot
\smallskip
\begin{figure} [h]  
   \centerline{\includegraphics[scale=.5]{ch11p2.eps} }
	\caption{Spectrum of $\mathbf{\boldsymbol{\cos(6\,\pi \,t)}}$  }
\end{figure} 

%plot2d ( [discrete, kglist],[x,0,5],[y,0,0.8],
%                  [style,[points,5,1,1]],[xlabel,"k"],
%                  [ylabel," "],[gnuplot_preamble,"set grid;"],
%                [psfile,"ch11p2.eps"])$

\newpage
\normalsize
\noindent We can also use \textbf{plot2d} to make a simple histogram, using \textbf{kglist}.
\begin{myVerbatim}
(%i15) vbars : makelist ( [discrete,
                  [[kglist[i][1],0],[kglist[i][1],kglist[i][2]]]] ,
                      i,1,length(kglist) );
(%o15) [[discrete, [[0, 0], [0, 0.0]]], [discrete, [[1, 0], [1, 0.0]]], 
[discrete, [[2, 0], [2, 0.0]]], [discrete, [[3, 0], [3, 0.5]]], 
[discrete, [[4, 0], [4, 0.0]]]]
(%i16) plot2d ( vbars,[y,0,0.8],[style,[lines,5,1]],
               [ylabel," "],[xlabel," k "],[legend,false],
                [gnuplot_preamble,"set grid;"] )$
\end{myVerbatim}
%\newpage
which produces the plot
\smallskip
\begin{figure} [h]  
   \centerline{\includegraphics[scale=.8]{ch11p3.eps} }
	\caption{Line Spectrum of $\mathbf{\boldsymbol{\cos(6\,\pi \,t)}}$  }
\end{figure} 

%plot2d ( [ [discrete,[[0,0]]],[discrete,[[1,0]]],
%              [discrete,[[2,0]]],
%               [discrete,[[3,0],[3,0.5]]],
%			   [discrete,[[4,0]]]],
%                [y,0,0.8],
%				[style,[points,3,1,1],[points,3,1,1],
%				  [points,3,1,1], [lines,8,1],
%				  [points,3,1,1]],
%               [ylabel," "],[xlabel," k "],[legend,false],
%                [gnuplot_preamble,"set grid;"],
%                  [psfile,"ch8p16b.eps"])$
\noindent The production of such a frequency space spectrum plot can be
  automated by accepting the fast Fourier transform list \textbf{glist} and defining
  a Maxima function \textbf{spectrum (glist, nlw, ymax) } which would be used 
  with the specific syntax \textbf{spectrum (glist, 5, 0.8)} to obtain a
  histogram similar to the above but cleaner.
(\textbf{nlw} is the line width and \textbf{ymax} is the vertical extent
  of the canvas starting at \mv|y = 0|.)\\
  
\noindent We have designed \textbf{spectrum} to also allow the expanded syntax
  \textbf{spectrum (glist, nlw, ymax, k1, k2) } which zooms in on
  the interval \mv|k1 <= k <= k2|, where  \mv|k2 <= knyq|.
In addition, we have avoided, in \textbf{spectrum}, creating vertical bars when a fast Fourier
  transform amplitude is less than a very small number.
\newpage
\noindent The entry
\begin{myVerbatim}
(%i17) spectrum (glist, 5, 0.8 )$
\end{myVerbatim}
produces the plot
\smallskip
\begin{figure} [h]  
   \centerline{\includegraphics[scale=.8]{ch11p4.eps} }
	\caption{spectrum ( glist, 5, 0.8 ) }
\end{figure} 

%spectrum_eps (glist,8, 0.8, "ch11p4.eps" )$

\noindent We will use \textbf{spectrum}, included in \textbf{qfft.mac},
  in our following examples as a quick way to look at the frequency spectrum.\\
  
\noindent Finally, let's use the Maxima function \textbf{inverse\_fft} to apply
  the inverse fast Fourier transform to the list of fast Fourier transform amplitudes
  \textbf{glist}, and see how closely the result matches the
  original signal sample list \textbf{flist}.
\begin{myVerbatim}
(%i18) flist1 : inverse_fft(glist);
(%o18) [1.0, 1.11022E-16 %i - 0.707, 2.24949E-32 %i - 1.83691E-16, 
6.77259E-17 %i + 0.707, - 1.0, 0.707 - 1.11022E-16 %i, 
5.51073E-16 - 2.24949E-32 %i, - 6.77259E-17 %i - 0.707]
(%i19) fchop(%);
(%o19)      [1.0, - 0.707, 0.0, 0.707, - 1.0, 0.707, 0.0, - 0.707]
(%i20) fchop( flist);
(%o20)      [1.0, - 0.707, 0.0, 0.707, - 1.0, 0.707, 0.0, - 0.707]
(%i21) lmax ( abs (flist1 - flist));
(%o21)                            1.30049E-16
\end{myVerbatim}
We see that \textbf{inverse\_fft (glist)} recovers our original signal
  sample list \textbf{flist} to within floating point errors.
\newpage
\subsubsection{Example 2: FFT Spectrum of a Sum of Two Monochromatic Signals }  \label{twof}
Almost all signals will contain more than one intrinsic frequency, and to recover 
   a portion of the frequency spectrum with fidelity, we need to satisfy the two conditions
\begin{equation}
\mathbf{ fs > 2\,f_{high} \quad \text{and} \quad  df \leq f_{low} },
 \end{equation}   
   in which  $\mathbf{f_{low}}$ is the lowest intrinsic frequency to be identified,
   and $\mathbf{f_{high}}$ is the highest intrinsic frequency to be identified.
   (Again, \textbf{fs} is the sampling frequency, and \textbf{df} is the
   frequency resolution).\\
   
\small
\noindent We assume now that the signal is 
  $\mathbf{F(t) = \boldsymbol{\cos(2\,\pi \,t)} + \boldsymbol{\sin(4\,\pi \,t)}}$.
We thus have a signal with the frequencies $\mathbf{f_{1} = 1\, s^{-1} }$\\
  and $\mathbf{f_{2} = 2\, s^{-1} }$.
With $\mathbf{fs}$ being the sampling frequency, and $\mathbf{ns}$ being the number of signal
  samples, we require $\mathbf{fs > 2\,f_{high} = 4}$, as well as $\mathbf{df \leq f_{low} = 1}$.
If we choose $\mathbf{f_{low} = 3\,df}$,
  then \mvs|df = 1/3 = fs/ns|, or \mvs|fs = ns/3|.
So the first condtion then
  implies we need $\mathbf{ns/3 > 4}$, or $\mathbf{ns > 12}$.
Since \mvs|ns| also should be equal to $\mathbf{2^{m}}$ for some 
  integer $\mathbf{m}$, we choose \mvs|ns = 16|.
Then \mvs|fs = 16/3| and \mvs|df = fs/ns = 1/3 hertz|. 
\begin{myVerbatim1}
(%i1) ( load(fft), load(qfft) )$
(%i2) e : cos(2*%pi*t) + sin(4*%pi*t)$
(%i3) (ns : 16, fs : 16/3)$
(%i4) dt : first (nyquist (ns,fs));
sampling interval dt =  0.188 
Nyquist integer knyq =  8 
Nyquist freq fnyq =  2.6667 
freq resolution df =  0.333 
(%o4)                                0.188
(%i5) flist : sample (e,t,ns,dt)$
(%i6) %,fll;
(%o6)                         [1.0, - 0.324, 16]
(%i7) tmax: ns*dt;
(%o7)                                 3.0
(%i8) tflist : vf (flist,dt)$
(%i9) %,fll;
(%o9)                  [[0, 1.0], [2.8125, - 0.324], 16]
(%i10) plot2d([e ,[discrete,tflist]], [t,0,tmax],
             [style,[lines,3],[points,3,2,1]],
              [legend,false])$
\end{myVerbatim1} 
We have used our utility function \mv|fll| described in the preface, which 
  returns the first and last element of a list, as well as the length of the
  list.
%\newpage
The plot thus produced is
\smallskip
\begin{figure} [h]  
   \centerline{\includegraphics[scale=.6]{ch11p5.eps} }
	\caption{ $\mathbf{\boldsymbol{\cos(2\,\pi \,t)} + \boldsymbol{\sin(4\,\pi \,t)}}$ with Sample Points  }
\end{figure} 
%plot2d([e ,[discrete,tflist]], [t,0,tmax],
%             [style,[lines,3],[points,3,2,1]],
%              [legend,false],[psfile,"ch11p5.eps"])$

\newpage
\normalsize
\noindent We next generate the list \textbf{glist} of fast Fourier transform amplitudes \mv|G(k*df)|
  and use \textbf{spectrum} to look at the implied frequency spectrum.
\begin{myVerbatim}
(%i11) glist : fft (flist)$
(%i12) %,fll;
(%o12)         [- 7.94822E-17, 1.52656E-16 - 5.7151E-17 %i, 16]
(%i13) spectrum (glist,5,0.6)$
\end{myVerbatim}
%\newpage
which produces the spectrum histogram:
\begin{figure} [h]  
   \centerline{\includegraphics[scale=1]{ch11p6.eps} }
	\caption{Spectrum of $\mathbf{\boldsymbol{\cos(2\,\pi \,t)} + \boldsymbol{\sin(4\,\pi \,t)}}$  }
\end{figure} 
% spectrum_eps (glist,8,0.6,"ch11p6.eps")$

\noindent Since \mvs|df = 1/3|, the line at \mvs|k = 3| corresponds to the
  frequency \mvs|f = k*df = 3*(1/3) = 1 Hertz| and the line at \mvs|k = 6|
  corresponds to the frequency \mvs|f = k*df = 6*(1/3) = 2 Hertz|.
\newpage
\subsubsection{Example 3: FFT Spectrum of a Rectangular Wave }
A rectangular wave as a function of time \mvs|t| with period equal to \mvs|64 sec|
  which cycles between plus and minus \mvs|1| can be constructed from 
   \mvs|floor| and \mvs|mod| in the form:
   \mv|rwave(t) := 2*mod(floor(t/32),2) - 1|.\\
   
\noindent For \mvs|t = 0, t/32 = 0, floor(t/32) = floor(0) = 0, mod(0,2) = 0|, 
  so \mvs|rwave(0) = -1|. 
\begin{myVerbatim}
(%i1) rwave(t):= 2*mod(floor(t/32),2) -1 $
(%i2) [floor(0),mod(0,2),rwave(0)];
(%o2)                             [0, 0, - 1]
\end{myVerbatim} 
For \mvs|t = 32, floor(32/32) = floor(1) = 1, mod(1,2) = 1, rwave(32) = 1|. 
\begin{myVerbatim}
(%i3) [floor(1),mod(1,2),rwave(32)];
(%o3)                              [1, 1, 1]
\end{myVerbatim} 
For \mvs|t = 64, floor(64/32) = floor(2) = 2, mod(2,2) = 0, rwave(64) = -1|. 
\begin{myVerbatim}
(%i4) [floor(2),mod(2,2),rwave(64)];
(%o4)                             [2, 0, - 1]
\end{myVerbatim} 
Hence \mvs|rwave(0 + 64) = rwave(0)| and \mvs|rwave(t)| has a period equal to \mvs|64 sec|.\\

\noindent We first look at the rectangular wave signal with a plot. 
\begin{myVerbatim}
(%i5) plot2d(rwave(t),[t,0,128],[y,-1.5,1.5],
         [ylabel," "],[style,[lines,5]],
         [gnuplot_preamble,"set grid;set zeroaxis lw 2;"])$
\end{myVerbatim}
\smallskip
\begin{figure} [h]  
   \centerline{\includegraphics[scale=.6]{ch11p7.eps} }
	\caption{ Rectangular Wave with Period 64 sec.  }
\end{figure} 

%plot2d([[discrete,[[0,0],[128,0]]], rwave(t)],
%                [t,0,128],[y,-1.5,1.5],
%         [ylabel," "],[style,[lines,2,4],[lines,8,1]],
%         [legend,false],[gnuplot_preamble,"set grid;"],
%          [psfile,"ch11p7.eps"])$

\noindent If we consider sampling this signal at intervals \mvs|dt = 1 sec|, 
   the sampling frequency will be \mvs|fs = 1/dt =|\\
   \mvs|1 hertz = 1 per sec = 64 per cycle|.
The lowest intrinsic frequency of this signal is  \mvs|f_low =|\\
 1/64 hertz|,  corresponding to the period of the rectangular wave. 
If we choose \mvs|4*df = f_low|, then \\
 \mvs|df = 1/256 = fs/ns = 1/ns|,  so \mvs|ns = 256|.
Due to the sharp corners of a rectangular wave we expect many high frequency components
  to be present in the spectrum.  
\newpage
\noindent We thus try the combination, \mvs|ns = 256|, and \mvs|fs = 1 hertz|. 
\begin{myVerbatim}
(%i6) (ns:256, fs:1)$
(%i7) (load(fft),load(qfft) )$
(%i8) dt:first(nyquist(ns,fs));
sampling interval dt =  1.0 
Nyquist integer knyq =  128 
Nyquist freq fnyq =  0.5 
freq resolution df =  0.00391 
(%o8)                                1.0
(%i9) flist : sample(rwave(t),t,ns,dt)$
(%i10) fll (flist);
(%o10)                          [- 1.0, 1.0, 256]
(%i11) makelist (flist[i],i,1,10);
(%o11) [- 1.0, - 1.0, - 1.0, - 1.0, - 1.0, - 1.0, - 1.0, - 1.0, - 1.0, - 1.0]
(%i12) tmax: ns*dt;
(%o12)                               256.0
(%i13) tflist : vf (flist,dt)$
(%i14) fll (tflist);
(%o14)                  [[0, - 1.0], [255.0, 1.0], 256]
(%i15) plot2d([rwave(t) ,[discrete,tflist]], [t,0,tmax],
              [y,-1.5,1.5],[ylabel," "],
             [style,[lines,3],[points,1,0,1]],
              [legend,false])$
\end{myVerbatim}
%\newpage
The \mv|sample_plot| invocation produces the black points of the sample on
  top and bottom of the rectangular wave shown here:
\smallskip
\begin{figure} [h]  
   \centerline{\includegraphics[scale=.8]{ch11p8.eps} }
	\caption{Black Sample Points with Rectangular Wave  }
\end{figure} 

%plot2d([rwave(t) ,[discrete,tflist]], [t,0,tmax],
%              [y,-1.5,1.5],[ylabel," "],
%             [style,[lines,8],[points,2,4,1]],
%              [legend,false],[psfile,"ch8p21.eps"])$

\newpage
\noindent Next we look at the signal frequency spectrum from \mvs|k = 0| to \mvs|k = knyq = 128|.
   Since \mvs|f = k*df|,  and \textbf{df = 1/256 \, hertz}, the low fundamental intrinsic frequency
   \mvs|f_0 = 1/64 hertz| will be located at \mvs|k = 4|, and the maximum
   possible frequency component will be \mvs|fnyq = knyq*df = 0.5 hertz = 32*f_0|. 
The spectrum plot is generated by passing the list \mv|flist| containing
  \mv|ns = 256| signal samples, along with \mv|nlw = 3|, and \mv|ymax = 0.7| to \mv|spectrum|:
\begin{myVerbatim}
(%i16) glist : fft (flist)$
(%i17) %,fll;
(%o17)                          [0.0, 0.0, 256]
(%i18) spectrum (glist,3,0.7)$
\end{myVerbatim} 
%\newpage
which produces the spectrum plot:
\smallskip
\begin{figure} [h]  
   \centerline{\includegraphics[scale=1]{ch11p9.eps} }
	\caption{Spectrum from k = 0 to knyq = 128   }
\end{figure} 

% spectrum_eps (glist,5,0.7,"ch11p9.eps")$

\noindent The plot shows lines at \mvs|k = 4, 12, 20, 28, 36, ....|.
Since \mvs|f = k*df| and \mvs|f_0 = 4*df| is the fundamental frequency, 
  the frequencies present are
  \mvs|f_0, 3*f_0, 5*f_0, 7*f_0,...| which is the ``fundamental'' plus odd harmonics of the
   fundamental.
We know there must be high frequency components to account
   for the sharp corners of the signal.
\newpage
\subsubsection{Example 4: FFT Spectrum Sidebands of a Tone Burst Before and After Filtering}
A tone burst signal consisting of a sine wave begins at \mvs|t = 0| and ends abruptly
  after forty five seconds.
The fast Fourier frequency spectrum of such a signal, when sampled over the duration
  of the sine wave plus some time following the sine wave end, will contain ``sideband''
  frequencies above and below the intrinsic frequency of the sine wave.
Many signal filters can be used to allow the easy identification of the intrinsic
  sine wave frequency by surpressing the sideband frequencies.
We use for the tone burst the sine wave: $\mathbf{\boldsymbol{\sin(2\,\pi\,t/5)}}$
 during the time interval  $\mathbf{0 \leq t \leq 45 \,sec}$,
  and then abruptly dropping to \mvs|0| for $\mathbf{t > 45 \, sec}$.
The intrinsic frequency of the sine wave is \mvs|f0 = 1/5 hertz|,
 and the corresponding period is \mvs|5 sec|, thus the
  sine wave lasts for nine periods.
We will use the following idealized model of such a signal by ignoring the actual time needed to
  end the sine wave. 
\begin{myVerbatim}
(%i1) sig(t) := if t < 45 then sin(2*%pi*t/5.0)  else 0$
\end{myVerbatim} 
\small
We will use this definition only for $\mathbf{t \geq 0}$ in our work.
With \mvs|fs| the sampling frequency, if we want about \textbf{10 \, samples/cycle} 
 (ie., \mvs|10 per 5 sec|),
  then we want \mvs|period*fs = 10|, or \mvs|fs| roughly equal to \mvs|2 hertz|.
We want the intrinsic, frequency \mvs|f0|  ( \mvs| 0.2  hertz|) to 
  \tcbr{not} be too close to the left end of the frequency spectrum plot so we can 
  see the low frequency sidebands as well as the higher frequency sidebands.\\
  
\noindent Suppose we try requiring that \mvs|f0 = 0.2 hertz = 50*df = 50*fs/ns|.
Solving for the number of signal samples \mvs|ns|, we get \mvs|ns = 50 * 2 * 5 = 500|.
To get a power of \mvs|2| we choose the closest such quantity, \mvs|ns = 512|.
We then return to our \mvs|f0| location requirement to solve for the resulting value
  of \mvs|fs| to get \textbf{fs = 512/250 = 2.048 \, hertz}. 
\begin{myVerbatim1}
(%i2) load (qfft)$
(%i3) (ns:512,fs:2.048)$
(%i4) dt : first(nyquist(ns,fs));
sampling interval dt =  0.488 
Nyquist integer knyq =  256 
Nyquist freq fnyq =  1.024 
freq resolution df =  0.004 
(%o4)                                0.488
(%i5) tmax : ns*dt;
(%o5)                                250.0
\end{myVerbatim1} 
We can plot a Maxima function defined with the \textbf{if  then  else} construct
  without problem since \textbf{plot2d} evaluates its arguments.
\begin{myVerbatim1}
(%i6) plot2d ( sig(t),[t,0,tmax],[y,-1.5,1.5],
         [style,[lines,1,0]],[ylabel," "],[nticks,100],
          [legend,false],[gnuplot_preamble,"set grid;"])$
\end{myVerbatim1}
%\newpage
which produces the plot:
\begin{figure} [h]  
   \centerline{\includegraphics[scale=.5]{ch11p10.eps} }
	\caption{Sine Wave Tone Burst, Period = 5 sec, Duration = Nine Periods  }
\end{figure} 

%plot2d ( sig(t),[t,0,tmax],[y,-1.5,1.5],
%         [style,[lines,1,4]],[ylabel," "],[nticks,100],
%          [legend,false],[gnuplot_preamble,"set grid;"],
%          [psfile,"ch8p23.eps"])$
\newpage
\normalsize
\noindent Using the \mvs|qfft| package function \mvs|sample| with a Maxima function
  defined with the \mvs|if...then| construct produces a list each element of which
  includes \mv|if  then  else|, which can be used with \textbf{plot2d}.
However, we will include an extra evaluation in defining our \textbf{flist} so we
  can see the numerical values as a list.
Note that the syntax \mv|expr, fll;| causes an extra evaluation, so to be careful
  we need to use the syntax \mv|fll (expr)|.
\begin{myVerbatim}
(%i7) flist : sample (sig(t),t,ns,dt)$
(%i8) fll (flist);
(%o8) [if 0.0 < 45.0 then 0.0 else 0.0, 
                                   if 249.51 < 45.0 then - 0.576 else 0.0, 512]
(%i9) flist : ev (flist)$
(%i10) fll (flist);
(%o10)                          [0.0, 0.0, 512]
(%i11) makelist(flist[i],i,1,10);
(%o11) [0.0, 0.576, 0.942, 0.964, 0.634, 0.0736, - 0.514, - 0.914, - 0.981, 
                                                                        - 0.69]
(%i12) tflist : vf (flist,dt)$
(%i13) fll (tflist);
(%o13)                  [[0, 0.0], [249.51, 0.0], 512]
\end{myVerbatim} 
We now use \textbf{plot2d} to show our sample points on top of our tone burst
  signal (both in black).
\begin{myVerbatim}
(%i14) plot2d([sig(t) ,[discrete,tflist]], [t,0,tmax],
              [y,-1.5,1.5],[ylabel," "],
             [style,[lines,1,0],[points,1,0,1]],
              [legend,false])$
\end{myVerbatim} 
%\newpage
which produces the plot:
\smallskip
\begin{figure} [h]  
   \centerline{\includegraphics[scale=.8]{ch11p11.eps} }
	\caption{Signal and Sample Points Which Include 0's for $\mathbf{t > 45 \,sec}$ }
\end{figure} 

% plot2d([sig(t) ,[discrete,tflist]], [t,0,tmax],
%              [y,-1.5,1.5],[ylabel," "],
%             [style,[lines,1,4],[points,2,4,1]],
%              [legend,false],[psfile,"ch11p11.eps"])$
% 
\newpage
\noindent We now plot the unfiltered tone burst spectrum after generating \textbf{glist}
  with \textbf{fft}.
\begin{myVerbatim}
(%i15) load (fft)$
(%i16) glist : fft (flist)$
(%i17) fll (glist);
(%o17)             [- 8.08763E-5, 0.00174 - 0.00286 %i, 512]
(%i18) spectrum (glist, 2, 0.1 )$
\end{myVerbatim} 
%\newpage
which produces the spectrum plot:
\begin{figure} [h]  
   \centerline{\includegraphics[scale=1]{ch11p12.eps} }
	\caption{Frequency Spectrum of Unfiltered Tone Burst}
\end{figure} 
% spectrum_eps (glist, 3, 0.1,"ch11p12.eps" )$
We see strong sideband frequencies on both the low and high side of the intrinsic
  frequency \mvs|f0 = 50*df|, which is the highest peak at \mvs|k = 50|
  (\textbf{f = \, k*df \, =  \,50*(1/250) \, = \, 1/5 \, = \, 0.02}).  
The sidebands are a mathematical artifact of the finite duration of the
  sine wave tone burst with its abrupt beginning and end.\\
  
\noindent A windowing filter which smooths out the beginning and end of the ``burst envelope''
  is used by defining a smoothed envelope signal as the product of the
  unsmoothed signal and a suitable windowing function.
The von Hann window employs a $\mathbf{\boldsymbol{\sin^{2}}}$ pinch envelope. 
\begin{myVerbatim}
(%i19) hannw(x,m) := sin(%pi*(x-1)/(m-1))^2$
(%i20) sig_w(t) := hannw(t,45)*sig(t)$
(%i21) plot2d ( sig_w(t),[t,0,tmax],[y,-1.5,1.5],
         [style,[lines,1,0]],[ylabel," "],[nticks,100],
          [legend,false],[gnuplot_preamble,"set grid;"])$
\end{myVerbatim} 
\newpage
\noindent which shows the filtered tone burst
\smallskip
\begin{figure} [h]  
   \centerline{\includegraphics[scale=.5]{ch11p13.eps} }
	\caption{Filtered Tone Burst }
\end{figure} 
%plot2d ( sig_w(t),[t,0,tmax],[y,-1.5,1.5],
%         [style,[lines,2,4]],[ylabel," "],[nticks,100],
%          [legend,false],[gnuplot_preamble,"set grid;"],
%           [psfile,"ch11p13.eps"])$

%\newpage
\small
\noindent Using now the filtered tone burst signal \mv|sig_w(t)|,
   we construct a filtered signal sample list \mv|flist_w| and
   look at the altered fast Fourier spectrum (remember \mv|dt| has been
   bound to \mv|0.488| and \mv|ns| to \mv|512|): 
\begin{myVerbatim1}
(%i22) flist_w : sample (sig_w(t),t,ns,dt)$
(%i23) fll (flist_w);
(%o23) [0.00509 (if 0.0 < 45.0 then 0.0 else 0.0), 
                           0.799 (if 249.51 < 45.0 then - 0.576 else 0.0), 512]
(%i24) flist_w : ev (flist_w)$
(%i25) fll (flist_w);
(%o25)                          [0.0, 0.0, 512]
(%i26) makelist (flist_w[i],i,1,5);
(%o26)          [0.0, 7.68319E-4, 2.63668E-6, 0.00106, 0.00293]
(%i27) glist_w : fft (flist_w)$
(%i28) makelist (glist_w[i],i,1,5);
(%o28) [1.59068E-5, 3.93191E-6 - 1.99372E-5 %i, - 1.83151E-5 %i - 1.84837E-5, 
                        2.09989E-6 %i - 2.59175E-5, 1.69797E-5 %i - 9.67992E-6]
(%i29) spectrum (glist_w, 2, 0.1)$
\end{myVerbatim1} 
%\newpage
and we see deletion of most of the sideband frequency peaks:
\smallskip
\begin{figure} [h]  
   \centerline{\includegraphics[scale=.5]{ch11p14.eps} }
	\caption{Frequency Spectrum of the Filtered Tone Burst }
\end{figure} 
% spectrum_eps (glist_w, 3, 0.1,"ch8p27.eps")$
\newpage
\normalsize
\subsubsection{Example 5: Cleaning a Noisy Signal using FFT Methods}
We use the same signal as used in Example 2, Sec. (\ref{twof}), but
  add some random noise to it.
Without the noise, the signal is 
  $\mathbf{F(t) = \boldsymbol{\cos(2\,\pi \,t)} + \boldsymbol{\sin(4\,\pi \,t)}}$.
Thus the clean signal contains the frequencies 
 $\mathbf{f_{1} = 1\, s^{-1} }$ and $\mathbf{f_{2} = 2\, s^{-1} }$,
  and two corresponding periods 
  $\mathbf{\boldsymbol{\tau}_{1} = 1/f_{1} = 1 \, sec}$ and
  $\mathbf{\boldsymbol{\tau}_{2} = 1/f_{2} = 0.5 \, sec}$.
To get a noisy looking signal we have to sample and add noise
  a lot of times within one or two periods of the
  clean signal, where we use the maximum of the intrinsic periods.
Let's try $\mathbf{N = 512}$ samples over a time \textbf{tmax = 2 \, sec}.
Then $\mathbf{N\,\boldsymbol{\delta}t = 2 \, sec}$, so \textbf{fs = 1/dt = N/2 = 256}.
The first condition on the sampling frequency is that $\mathbf{f_{s} > 2\, f_{high}}$,
  or $\mathbf{fs > 4 \, s^{-1}}$, which is certainly satisfied.
The second condition on \textbf{fs} is $\mathbf{\boldsymbol{\delta}f < f_{low}}$,
  or $\mathbf{df < 1 \, s^{-1}}$, or $\mathbf{fs/ns < 1 \, s^{-1}}$, or $\mathbf{fs < ns}$, which is also
  true.
Hence we try \textbf{ns = 512, \, fs = 256}. 
\begin{myVerbatim}
(%i1) e : cos(2*%pi*t) + sin(4*%pi*t)$
(%i2) (load(fft), load(qfft))$
(%i3) [ns:512,fs:256]$
(%i4) dt : first(nyquist(ns,fs));
sampling interval dt =  0.00391 
Nyquist integer knyq =  256 
Nyquist freq fnyq =  128.0 
freq resolution df =  0.5 
(%o4)                               0.00391
(%i5) flist : sample(e,t,ns,dt)$
(%i6) %,fll;
(%o6)                          [1.0, 0.951, 512]
(%i7) flist_noise : makelist(flist[j]+0.3*(-1.0+random(2.0)),j,1,ns)$
(%i8) %,fll;
(%o8)                        [1.2483, 1.0073, 512]
(%i9) tflist_noise : vf (flist_noise,dt)$
(%i10) %,fll;
(%o10)               [[0, 1.2483], [1.9961, 1.0073], 512]
(%i11) plot2d ([discrete,tflist_noise],[y,-2,2],
                 [style,[lines,1]],[ylabel," "])$
\end{myVerbatim} 
%\newpage
which produces the noisy signal plot
\smallskip
\begin{figure} [h]  
   \centerline{\includegraphics[scale=.7]{ch11p15.eps} }
	\caption{Noisy Signal}
\end{figure} 

% plot2d ([discrete,tflist_noise],[y,-2,2],
%                 [style,[lines,1]],[ylabel," "],
%                [psfile,"ch8p19.eps"])$
%
\newpage

\noindent One way to ``clean'' this signal is to set small numbers in the fast Fourier
  transform to zero and inverse transform back to the time domain using
  \textbf{inverse\_fft}.\\
  
\noindent In order to ``chop'' small numbers in the fast Fourier transform
  amplitude list, we need to produce that list, which we call
  \textbf{glist\_noise} from the noisy signal sample \textbf{flist\_noise} by
   using \textbf{fft}.
\begin{myVerbatim}
(%i12) glist_noise : fft (flist_noise)$
(%i13) %,fll;
(%o13)             [- 0.0022, 5.09918E-4 %i - 0.00338, 512]
\end{myVerbatim}

\noindent Before ``chopping'' small numbers in \textbf{glist\_noise}, we take a look at the
  fast Fourier frequency spectrum implied by \textbf{glist\_noise}.
\begin{myVerbatim}
(%i14) spectrum (glist_noise,2,0.6)$
\end{myVerbatim} 
%\newpage
which produces the plot
\smallskip
\begin{figure} [h]  
   \centerline{\includegraphics[scale=.7]{ch11p16.eps} }
	\caption{Noisy Signal Frequency Spectrum}
\end{figure} 
%spectrum_eps (glist_noise,3,0.6,"ch11p16.eps")$

\noindent The dominant lines are still those corresponding to the two intrinsic
  frequencies of the clean signal we started with, but there are many more frequencies
  present due to the noise.
\newpage
\noindent To see more clearly the dominant line region, we show another plot with 
  the integer \textbf{k} in the range \textbf{(0, 10)}: 
\begin{myVerbatim1}
(%i15) spectrum (glist_noise,4,0.6,0,10)$
\end{myVerbatim1} 
%\newpage
which produces the plot:
\smallskip
\begin{figure} [h]  
   \centerline{\includegraphics[scale=.55]{ch11p17.eps} }
	\caption{Noisy Signal Frequency Spectrum with k in (0,10)}
\end{figure} 
%spectrum_eps (glist_noise,6,0.6,"ch11p17.eps",0,10)$
%\newpage

\noindent We now use \textbf{fchop1} with the value \textbf{0.2} to set small floating point numbers
  less than \textbf{0.2} to \tcbr{zero} in the frequency space list \textbf{glist\_noise}, 
  and again use \textbf{spectrum} to look at the frequency spectrum associated with the
  chopped frequency space list. 
\begin{myVerbatim1}
(%i16) glist_noise_chop : fchop1(glist_noise,0.2)$
(%i17) %,fll;
(%o17)                          [0.0, 0.0, 512]
(%i18) spectrum (glist_noise_chop,2,0.6)$
\end{myVerbatim1} 
%\newpage
which produces the plot
\begin{figure} [h]  
   \centerline{\includegraphics[scale=.55]{ch11p18.eps} }
	\caption{Chopped Frequency Spectrum }
\end{figure} 
%spectrum_eps (glist_noise_chop,3,0.6,"ch11p18.eps")$

\newpage
\noindent Here is the cleaned spectrum in the range \textbf{k = (0,10)}:
\smallskip
\begin{figure} [h]  
   \centerline{\includegraphics[scale=1]{ch11p19.eps} }
	\caption{Chopped Frequency Spectrum for k in (0,10)}
\end{figure} 

%spectrum_eps (glist_noise_chop,6,0.6,"ch11p19.eps",0,10)$

\noindent We now create a \tcb{cleaned up signal list} using \textbf{inverse\_fft} on the
  chopped glist, transforming back to the time domain. 
\begin{myVerbatim}
(%i19) flist_clean : inverse_fft ( glist_noise_chop )$
(%i20) %,fll;
(%o20)      [2.22045E-16 %i + 1.0016, 0.952 - 3.16992E-15 %i, 512]
(%i21) flist_clean : realpart(flist_clean)$
(%i22) %,fll;
(%o22)                       [1.0016, 0.952, 512]
\end{myVerbatim} 
Since the \tcbr{inverse} Fourier transform will often include small imaginary parts due
  to floating point error, we took care to take the \tcb{real part} of the
  returned list before looking at the cleaned up signal.
We now construct the list of points \textbf{[t, F\_clean]}: 
\begin{myVerbatim}
(%i23) tflist_clean : vf ( flist_clean, dt )$
(%i24) %,fll;
(%o24)                [[0, 1.0016], [1.9961, 0.952], 512]
\end{myVerbatim} 
\newpage
\small
\noindent and plot first just the cleaned up signal points 
\begin{myVerbatim1}
(%i25) plot2d ([discrete,tflist_clean],[y,-2,2],
                 [style,[lines,1]],[ylabel," "])$
\end{myVerbatim1} 
%\newpage
which produces
\begin{figure} [h]  
   \centerline{\includegraphics[scale=.5]{ch11p20.eps} }
	\caption{Cleaned Signal Points}
\end{figure} 

%plot2d ([discrete,tflist_clean],[y,-2,2],
%                 [style,[lines,3]],[ylabel," "],
%                 [psfile,"ch11p20.eps"])$

\noindent We now show both the cleaned list points together with the original clean two frequency
  signal we started with, to show that the cleaned up points lie right on top of the
  original two frequency signal. 
\begin{myVerbatim1}
(%i26) plot2d([e ,[discrete,tflist_clean]], [t,0,2],
             [style,[lines,1],[points,1,0,1]],
              [legend,false])$
\end{myVerbatim1}
%\newpage
which produces the plot
\begin{figure} [h]  
   \centerline{\includegraphics[scale=.65]{ch11p21.eps} }
	\caption{Cleaned Signal Points on Top of Original Clean Signal}
\end{figure} 

% plot2d([e ,[discrete,tflist_clean]], [t,0,2],
%             [style,[lines,1,1],[points,2,4,1]],
%              [legend,false],[psfile,"ch11p21.eps"])$ 

\noindent \normalsize We see that the inverse fast Fourier transform of the chopped glist frequency spectrum
  yields a cleaned up signal, as desired.
\newpage
\subsection{Our Notation for the Discrete Fourier Transform and its Inverse} \label{notate}
Given a real valued signal $\mathbf{F(t)}$ which is to be sampled $\mathbf{N}$ times 
  at the moments (separated by equal time intervals $\mathbf{\boldsymbol{\delta}t}$)
 $\mathbf{t = 0,\, \boldsymbol{\delta}t,\, 2\,\boldsymbol{\delta}t,\,...,(N-1)\,\boldsymbol{\delta}t}$,
one needs to select the two parameters $\mathbf{N = 2^{m}}$ and $\mathbf{f_{s}}$, where the latter
  is called the sampling frequency.
The sampling time interval is then given by
\begin{equation}  \label{Eq:dt}
\mathbf{\boldsymbol{\delta}t} = \mathbf{\frac{1}{f_{s}}}
\end{equation}
(or else use the data determined value of $\mathbf{\boldsymbol{\delta}t}$ to
  calculate $\mathbf{f_{s}}$).
Note that the fast fourier transform algorithm used by Maxima assumes that
  the number of signal samples $\mathbf{N}$ is some integral power of $2$, 
  $\mathbf{N = 4, 8, 16, 32, 64, ...}$ so an experimental sample might have 
  to be padded with zeroes to achieve this condition on $\mathbf{N}$.
The sampling frequency $\mathbf{f_{s}}$  should be greater than twice the 
  highest frequency component to be identified in
  the signal and the frequency resolution $\mathbf{\boldsymbol{\delta}f}$
  should be smaller than the lowest frequency to be identified in the signal.
Given the sampling frequency $\mathbf{f_{s}}$ and the number of signal samples
  $\mathbf{N}$, the frequency resolution $\mathbf{\boldsymbol{\delta}f}$ is
  given by
\begin{equation}  \label{Eq:df}
\mathbf{\boldsymbol{\delta}f} = \mathbf{\frac{f_{s}}{N}}
\end{equation}
We will motivate this definition below.
Assuming this definition, we then require that
\begin{equation}
\mathbf{ \frac{f_{s}}{N} < f_{low} }
\end{equation}
The sampling frequency $\mathbf{f_{s}}$ thus needs to satisfy the two conditions:
\begin{equation}  \label{Eq:highlow}
\mathbf{ 2\, f_{high} < f_{s} < N\,f_{low} }
\end{equation}
A convenient choice which automatically satisfies the low frequency part of these
  conditions is to arrange that
\begin{equation}  \label{Eq:flow}
  \mathbf{f_{low} = n\,\boldsymbol{\delta}f},
\end{equation}
  where $\mathbf{n = 3 \;\text{or} \; 4}$, say.
Then that choice determines the frequency resolution to be used 
 $\mathbf{\boldsymbol{\delta}f = f_{low}/n}$, and from the definition
 of $\mathbf{\boldsymbol{\delta}f}$, Eq.(\ref{Eq:df}),  this requires that
\begin{equation} \label{Eq:fs}
\mathbf{f_{s} = N\,f_{low}/n }
\end{equation}
and then Eq.(\ref{Eq:highlow}) implies the condition on $\mathbf{N}$:
\begin{equation} \label{Eq:nmin}
\mathbf{N > \frac{2\,n\,f_{high}}{f_{low}}}
\end{equation}
In the simple case that $\mathbf{f_{low} = f_{high} = f_{0}}$
  Eq.(\ref{Eq:fs}) becomes
\begin{equation} \label{Eq:fs1}
\mathbf{f_{s} =  N \, \frac{f_{0}}{n} },
\end{equation}
Eq.(\ref{Eq:highlow}) becomes
\begin{equation}  \label{Eq:ns1}
\mathbf{N > 2\,n},
\end{equation}
and 
\begin{equation}  \label{Eq:df1}
\mathbf{ \boldsymbol{\delta}f = \frac{f_{0}}{n} }
\end{equation}
%\newpage
In Example 1, the signal frequency is $\mathbf{f_{0} = 3}$, and we chose \textbf{n = 3}.
Then, Eq.(\ref{Eq:fs1}) implies that $\mathbf{f_{s} = N}$, 
  Eq.(\ref{Eq:ns1}) implies that $\mathbf{N > 6}$, and
  Eq.(\ref{Eq:df1}) implies that $\mathbf{\boldsymbol{\delta}f = 1}$.
Since we need $\mathbf{N = 2^{m}}$ as well, we chose $\mathbf{f_{s} = N = 8}$.\\

\noindent In Example 2, $\mathbf{f_{low} = 1}$ and $\mathbf{f_{high} = 2}$, and we again
  chose \textbf{n = 3}. 
Then, Eq.(\ref{Eq:flow}) implies that $\mathbf{\boldsymbol{\delta}f = 1/3}$,
   Eq.(\ref{Eq:fs}) implies that $\mathbf{f_{s} = N/3}$,
   and Eq.(\ref{Eq:nmin}) implies that $\mathbf{N > 12}$.
Since we need $\mathbf{N = 2^{m}}$ as well, we chose $\mathbf{N = 16}$ and 
  this forces $\mathbf{f_{s} = 16/3}$.
\newpage
\noindent The $\mathbf{N}$ real numbers $\mathbf{F(0),\,F(\boldsymbol{\delta}t),\,
  F(2\,\boldsymbol{\delta}t),\,..., F(m\,\boldsymbol{\delta}t),\,..., F((N-1)\,\boldsymbol{\delta}t) }$
  can be used to define $\mathbf{N}$ complex numbers $\mathbf{G(k\,\boldsymbol{\delta}f)}$, where
  $\mathbf{k = 0,\,1,\, ... ,(N - 1)}$, according to (this incorporates Maxima's conventions):
\begin{equation}  \label{Eq:Gval}
\mathbf{G(k\,\boldsymbol{\delta}f)} = 
  \mathbf{\frac{1}{N}\,\sum_{m = 0}^{N - 1}\, F(m\,\boldsymbol{\delta}t)\,
               e^{-2\,\boldsymbol{\pi}\,i\,m\,k/N}}
\end{equation}
The Maxima conventions include where to put the factor of $\mathbf{1/N}$ and
  what sign to use in the exponent argument.
Our notation uses $\mathbf{i}$ to stand for the pure imaginary number $\mathbf{\sqrt{-1}}$.
Common engineering notation uses $\mathbf{j}$ for $\mathbf{\sqrt{-1}}$.\\

\noindent Equation (\ref{Eq:Gval}) can be exactly inverted to arrive at an
  expression for the values of the original signal at $\mathbf{N}$ discrete
  times in terms of the $\mathbf{N}$ values of the discrete Fourier transform.
\begin{equation}  \label{Eq:Fval}
\mathbf{F(m\,\boldsymbol{\delta}t)} = 
  \mathbf{\sum_{k = 0}^{N - 1}\, G(k\,\boldsymbol{\delta}f)\,
               e^{2\,\boldsymbol{\pi}\,i\,k\,m/N}}
\end{equation}
where $\mathbf{m = 0,\,1,\, ... ,(N - 1)}$.\\

\noindent Using Eqs. (\ref{Eq:df}) and (\ref{Eq:dt}), we make the replacement
\begin{equation}  \label{Eq:dfdt}
\mathbf{\frac{1}{N} =  \boldsymbol{\delta}f \, \boldsymbol{\delta}t }
\end{equation}
in Equations (\ref{Eq:Gval}) and (\ref{Eq:Fval}) to get
\begin{equation} \label{Eq:Gfk}
\mathbf{G(f_{k})} = 
  \mathbf{\frac{1}{N}\,\sum_{m = 0}^{N - 1}\, F(t_{m})\,
               e^{-2\,\boldsymbol{\pi}\,i\,f_{k}\,t_{m}}}
\end{equation}
and
\begin{equation}
\mathbf{F(t_{m})} = 
  \mathbf{\sum_{k = 0}^{N - 1}\, G(f_{k})\,
               e^{2\,\boldsymbol{\pi}\,i\,t_{m} \, f_{k}}}
\end{equation}
where $\mathbf{f_{k}} = \mathbf{k\,\boldsymbol{\delta}f}$ 
  and $\mathbf{t_{m}} = \mathbf{m\,\boldsymbol{\delta}t}$.\\
  
\noindent A simpler looking set of transform pairs can be achieved by
  letting $\mathbf{F_{m} = F(t_{m})}$ and $\mathbf{G_{k} = G(f_{k})}$, in terms
  of which Equations (\ref{Eq:Gval}) and (\ref{Eq:Fval}) become
\begin{equation}  \label{Eq:Gval2}
\mathbf{G_{k}} = 
  \mathbf{\frac{1}{N}\,\sum_{m = 0}^{N - 1}\, F_{m}\,
               e^{-2\,\boldsymbol{\pi}\,i\,m\,k/N}}
\end{equation}
and
\begin{equation}  \label{Eq:Fval2}
\mathbf{F_{m}} = 
  \mathbf{\sum_{k = 0}^{N - 1}\, G_{k}\,
               e^{2\,\boldsymbol{\pi}\,i\,k\,m/N}}
\end{equation}
We can use
\begin{equation}
\mathbf{e^{-2\,\boldsymbol{\pi}\,i\,m\,k}}=\mathbf{(e^{-2\,\boldsymbol{\pi}\,i\,k})^{m}} =
  \mathbf{(-1)^{m}} = 1
\end{equation}
to show that the fast Fourier amplitudes have the periodicity $\mathbf{N}$
\begin{equation}
\mathbf{G_{k + N}} = \mathbf{G_{k}}.
\end{equation}
We can now formally admit  negative frequencies by letting $\mathbf{k}$ take
  on negative integral values, and setting $\mathbf{k = -N/2}$ we then get
\begin{equation}
\mathbf{G_{N/2}} = \mathbf{G_{-N/2}}.
\end{equation}
\newpage
\noindent This means that the amplitude corresponding to the ``Nyquist frequency''
\begin{equation} \label{Eq:fnyq}
  \mathbf{ f_{Nyquist} = \frac{N}{2}\,\boldsymbol{\delta}f }
\end{equation}
  is the same complex  number as the amplitude corresponding to the 
  frequency $\mathbf{-f_{Nyquist}}$.\\
In the \textbf{qfft} package, the function \textbf{nyquist} calculates
  what we call the ``Nyquist integer'' $\mathbf{k_{Nyquist}}$, which is just
\begin{equation} \label{Eq:knyq}
\mathbf{k_{Nyquist} = \frac{N}{2}},
\end{equation} 
in terms of which 
\begin{equation}
\mathbf{ f_{Nyquist} = k_{Nyquist}\, \boldsymbol{\delta}f }
\end{equation}
Likewise we can show that
\begin{equation}
\mathbf{G_{N/2 + 1}} = \mathbf{G_{-N/2 + 1}}.
\end{equation}
which means that the amplitude corresponding to the frequency
  $\mathbf{(N/2 + 1)\,\boldsymbol{\delta}f = f_{Nyquist} + \boldsymbol{\delta}f}$ 
  is the same complex number as the amplitude corresponding to the 
  frequency $\mathbf{-f_{Nyquist} + \boldsymbol{\delta}f}$.
The only useful part of the spectrum in that contained in the frequency interval
  between zero and $\mathbf{f_{Nyquist} - \boldsymbol{\delta}f}$, ie., in the range \\
  $\mathbf{k = 0, 1, 2, ..., (N/2) -1}$.\\
  
\noindent In a similar manner we can show that $\mathbf{F_{m + N} = F_{m}}$,
or that $\mathbf{F(t_{m} + T) = F(t_{m})}$, where
\begin{equation}
  \mathbf{T = N\,\boldsymbol{\delta}t = \frac{N}{f_{s}} = \frac{1}{\boldsymbol{\delta}f}},
\end{equation}
  so that the fast Fourier amplitudes
  describe a signal which has the basic inevitable long period $\mathbf{T}$ no matter
  what other shorter periods (and correspondingly higher frequencies) are also
  present in the signal.
This low frequency, long period property is an artifact of the approximate nature
  of Equations (\ref{Eq:Gval}) and (\ref{Eq:Fval}).\\
  
\noindent The fast fourier transform and its inverse should be considered as a distinct
  type of transform pair rather than as an approximation to either a Fourier
  series expansion or a Fourier integral expression of a continuous spectrum.
The basic idea of the fast Fourier transform is that one has waited long enough
  for a physical system to ``settle down'' and the system is then sampled for
  a certain finite length of time $\mathbf{T_{s}}$ (we use frequency-time language 
  only for simplicity here, the same ideas apply to wavelength-spatial domain
  problems).
\newpage
\subsection{Syntax of \textbf{qfft.mac} Functions}  \label{syntax}
\begin{myVerbatim2s}
   FAST FOURIER TRANSFORM UTILITIES   
    nyquist (ns,fs) 
    sample (expr,var,ns,dvar)
    vf (flist,dvar)
    kg (glist)        
    fchop (expr)
    fchop1 (expr,small)
    current_small ()
    setsmall (val)
    spectrum ( glist, nlw, ymax, k1,k2 )    

	
1.  nyquist (ns, fs), given ns, the number of signal 
    samples, and fs, the sampling frequency,  
    returns the list [dt,knyq,fnyq,df] where dt is the time 
    interval between function samples (dt = 1/fs), knyq in the Nyquist
    integer (knyq = ns/2),  fnyq is the Nyquist frequency 
    (fnyq = fs/2 = knyq*df), and df is the frequency
    resolution (df = fs/ns) for the output of fft.
    For given ns, the values of df and dt are linked
    by the equation df*dt = 1/ns.
	Thus  nyquist(8,8); returns the list
             [0.125, 4, 4.0, 1.0], and also prints out:
       
        sampling interval dt =  0.125 
        Nyquist integer knyq =  4 
        Nyquist freq fnyq =  4.0 
        freq resolution df =  1.0
		
2. sample (expr, var, ns, dvar) constructs a list of ns
   floating point samples F(m*dvar),
   [F(0), F(dvar), F(2*dvar),...,F((ns-1)*dvar) ],
   given the expression expr depending on var.
   
      sample(cos(t),t,16,1) returns a list of 16 values 
      of cos(t) at intervals dt = 1, with list element 
      number 1 holding cos(0), element number 2 holding 
      cos(dt) = cos(1), etc.
	  
	  A signal sample list to be used with fft(flist) should
	  have a length ns which is 2 raised to some integer power,
	  2^3 = 8, 2^4 = 16,... If your experimental signal sample
	  size does not have such a power of 2 length, you should pad
	  the sample list with extra zeros.
	  
3. vf ( flist, dvar ), given a list of function samples flist,
    consisting of ns values, and the step size dvar, returns a
	list of the form (if dvar = dt, say),
   [ [0, F(0)],[dt, F(dt)],...,[(ns-1)*dt, F( (ns-1)*dt)]]
   useful for a plot.
    For example, if the length of flist is 8, vf (flist, 1) returns the list
       [ [0, F(0)], [1, F(1)], ... , [7, F(7)] ]
\end{myVerbatim2s}
\newpage
\begin{myVerbatim2s}
4.  kg ( glist ) constructs a list of [ k, abs( g(k*df) ) ] 
     for k = 0, 1, ..., knyq, which can be used with plot2d.
	 knyq is the Nyquist integer, knyq = ns/2, 
     where ns is the number of function samples and also the 
     length of flist and glist. fchop(abs(glist[j])) is 
     used to be able to plot real numbers and set tiny
	 numbers to zero.
	 
     glist is the fast Fourier transform list of complex amplitudes
	 produced by glist : fft( flist ). 
	 
	 Using inverse_fft( glist ) should produce flist again to within
	 floating point errors. Since floating point errors will also
	 introduce tiny imaginary numbers in inverse_fft ( fft (flist) )
	 (if flist is real), you can use realpart(...) to recover
	 a list of real numbers.

5.  fchop(expr) or fchop(list) sets tiny floating point
          numbers (with package default, less than 10^(-13) ) to zero.
		  
6.  fchop1(expr,small) is used to override the default value of
       the small chop value with your desired value.	  
	
      Example: fchop1(s1, 1.0e-3) to set numbers smaller than
       10^(-3) to zero in the expression or list s1.
	
7.  current_small() returns the current default small chop value.	 

8.  setsmall(val) allows you to set a new value for the
     current default small chop value; use floating point
     numbers like 2.0e-3 or 2.0E-3.
	 
9.  spectrum (glist, nlw, ymax ) creates a histogram of the frequency
    spectrum implied by glist = fft ( flist ), with line width nlw
	and vertical canvas height ymax. The range of integers
	k is 0 <= k <= knyq. The frequency associated with a given
	line is given by f = k*df, where df is the frequency 
	resolution df = fs/ns. ns is the total number of signal samples
	and fs is the sampling frequency: fs = 1/dt (dt is the time 
	interval between signal samples).
	
	spectrum (glist, nlw, ymax, k1, k2) restricts the plot to
	the range k1 <= k <= k2.
\end{myVerbatim2s}
\newpage
\subsection{The Discrete Fourier Transform Derived via a Numerical Integral Approximation}
\subsubsection*{Review of Trapezoidal Rule}
If we let $\mathbf{f_{0} = f(a),\; f_{1} = f(a + h),\; f_{2} = f(a + 2\,h),...}$
  and $\mathbf{f_{N} = f(b) = f(a + N\,h)}$, then the trapezoidal rule approximation
  is
\begin{equation}
\mathbf{\int_{a}^{b} f(x)\,dx \approx \frac{h}{2}\,(f_{0} + 2\,f_{1} + 2\,f_{2} + \cdots  + 2\,f_{N-1} + f_{N})}
\end{equation}
where $\mathbf{b = a + N\,h}$ defines $\mathbf{h = (b-a)/N}$.
If we now specialize to functions such that $\mathbf{f(a) = f(b)}$ then the trapezoidal
  rule reduces to
\begin{equation}
\mathbf{\int_{a}^{b} f(x)\,dx \approx h\,\left(f(a) + f(a+h) + f(a+2\,h) + \cdots + f(b- 2\,h) + f(b-h)\right)}
\end{equation}
with $\mathbf{h = (b-a)/N}$.\\

\noindent If we now make the replacements $\mathbf{x \rightarrow t,\;a \rightarrow 0,\;
         b \rightarrow T,\; h = (b-a)/N \rightarrow T/N = \boldsymbol{\Delta}t}$,\\
		 $\mathbf{(b-h) \rightarrow T - (T/N) = (N-1)\,\boldsymbol{\Delta}t}$, then
\begin{equation}
\mathbf{\int_{0}^{T} f(t)\,dt \approx (T/N)\,\left(f(0) + f(\boldsymbol{\Delta}t) +
         f(2\,\boldsymbol{\Delta}t) + \cdots + f(\,(N-1)\,\boldsymbol{\Delta}t) \,\right) }
\end{equation}
or
\begin{equation} \label{Eq:trap}
\mathbf{\int_{0}^{T} f(t)\,dt \approx \boldsymbol{\Delta}t \, \sum_{m=0}^{N-1}\,f(m\,\boldsymbol{\Delta}t) }
\end{equation}
where $\mathbf{\boldsymbol{\Delta}t = T/N}$ and assuming $\mathbf{f(0) = f(T)}$.
\subsubsection*{A Path to the Discrete Fourier Transform}
If we knew the value of a signal $\mathbf{F(t)}$ at all moments of the interval
  $\mathbf{0 \leq t \leq T}$ then we could evaluate the Fourier coefficients
\begin{equation}  \label{Eq:cont}
\mathbf{C_{k} = \frac{1}{T}\,\int_{0}^{T} F(t)\,e^{- 2\,\boldsymbol{\pi}\,i\,k\,t/T}\,dt }
\end{equation}
  ($\mathbf{k}$ is an integer) in terms of which the signal could be represented as the sum
\begin{equation}
\mathbf{F(t) = \sum_{k = -\infty}^{\infty}\,C_{k} \,e^{2\,\boldsymbol{\pi}\,i\,k\,t/T} }
\end{equation}
  which would be a complex form of Fourier series expansion in terms of an infinite number
  of Fourier coefficients $\mathbf{C_{k}}$.
We have adopted here sign and prefactor conventions which will lead us to Maxima's fast
  Fourier conventions.\\
  
\noindent Now assume we only know the signal at $\mathbf{N}$ discrete values
  $\mathbf{F(m\,\boldsymbol{\Delta}t)}$, where $\mathbf{m = 0,1,2,...,N-1}$,
  and that $\mathbf{F(t) = f(t+T) }$. 
We can then approximate the integral in Eq.(\ref{Eq:cont}) using the trapezoidal
  approximation expressed by Eq.(\ref{Eq:trap}).
\begin{equation}
\mathbf{C_{k} \approx \frac{1}{T}\,(T/N)\,\sum_{m=0}^{N-1}\,F(m\,\boldsymbol{\Delta}t)\,
               e^{- 2\,\boldsymbol{\pi}\,i\,k\,m\,\boldsymbol{\Delta}t/T} }
\end{equation}
Defining a ``frequency resolution'' $\mathbf{\boldsymbol{\delta}f}$ given by 
\begin{equation}   \label{Eq:dfdt2}
\mathbf{\boldsymbol{\delta}f = \frac{1}{T} = \frac{1}{\boldsymbol{\Delta}t\,N} = 
                     \frac{f_{s}}{N} }
\end{equation}
in which $\mathbf{f_{s} = 1/\boldsymbol{\Delta}t}$ is the ``sampling frequency'',
  and making the replacements $\mathbf{T \rightarrow 1/\boldsymbol{\delta}f,\;
    C_{k} \rightarrow G(k\,\boldsymbol{\delta}f)}$
\begin{equation}  \label{Eq:Gkdf}
\mathbf{G(k\,\boldsymbol{\delta}f) = \frac{1}{N}\,\sum_{m=0}^{N-1}\,F(m\,\boldsymbol{\Delta}t)\,
                   e^{- 2\,\boldsymbol{\pi}\,i\,(k\,\boldsymbol{\delta}f)\,(m\,\boldsymbol{\Delta}t)}}
\end{equation}
With $\mathbf{f_{k}} = \mathbf{k\,\boldsymbol{\delta}f}$ 
  and $\mathbf{t_{m}} = \mathbf{m\,\boldsymbol{\Delta}t}$, we can write this as
\begin{equation} 
\mathbf{G(f_{k})} = 
  \mathbf{\frac{1}{N}\,\sum_{m = 0}^{N - 1}\, F(t_{m})\,
               e^{-2\,\boldsymbol{\pi}\,i\,f_{k}\,t_{m}}}
\end{equation}
which we recognise as the same as Eq.(\ref{Eq:Gfk}).
With only $\mathbf{N}$ values of $\mathbf{F(t_{m})}$ we can only determine $\mathbf{N}$
  values of $\mathbf{G(f_{k})}$, which by convention we take to be for the values of
  the integer $\mathbf{k = 0, 1, 2,..., N-1}$.\\
  
\noindent We see from Eq.(\ref{Eq:dfdt2}) that
\begin{equation}
\mathbf{\boldsymbol{\delta}f\,\boldsymbol{\Delta}t = \frac{1}{N} }
\end{equation}
  which allows Eq.(\ref{Eq:Gkdf}) to be written as
\begin{equation}  \label{Eq:Gval3} 
\mathbf{G(k\,\boldsymbol{\delta}f) = \frac{1}{N}\,\sum_{m=0}^{N-1}\,F(m\,\boldsymbol{\Delta}t)\,
                   e^{- 2\,\boldsymbol{\pi}\,i\,k\,m/N }} 
\end{equation}
 which reproduces our starting point Eq.(\ref{Eq:Gval}) in Sec. \ref{notate}.
\subsubsection*{The Inverse Discrete Fourier Transform}
\small
Although the discrete Fourier transform, Eq.(\ref{Eq:Gval3}), is an approximation which gets
  better as $\mathbf{N}$ increases for fixed $\mathbf{T}$ (or since Eq.(\ref{Eq:dfdt2}) indicates
  that the latter condition is equivalent to the ratio $\mathbf{f_{s}/N}$ being fixed, gets
  better as $\mathbf{N}$ and $\mathbf{f_{s}}$ are each increased in the same ratio), the
  inversion formula involves no further approximations.\\
  
\noindent Multiplying both sides of Eq.(\ref{Eq:Gval3}) by 
  $\mathbf{exp(2\,\boldsymbol{\pi}\,i\,k\,n/N)}$ and then summing both sides over $\mathbf{k}$,
  and then interchanging the order of the $\mathbf{k}$ and $\mathbf{m}$ summations on the right
  hand side, results in
\begin{equation}
\mathbf{\sum_{k = 0}^{N-1} G(k\,\boldsymbol{\delta}f)\,e^{2\,\boldsymbol{\pi}\,i\,k\,n/N}} = 
          \mathbf{\frac{1}{N}\,\sum_{m = 0}^{N-1} F(m\,\boldsymbol{\Delta}t)\,
		  \sum_{k = 0}^{N-1} e^{2\,\boldsymbol{\pi}\,i\,k\,(n - m)/N} } =
		  \mathbf{\frac{1}{N}\,\sum_{m = 0}^{N-1} F(m\,\boldsymbol{\Delta}t)\,(N\,\boldsymbol{\delta}_{m,n})} =
		  \mathbf{F(n\,\boldsymbol{\Delta}t)}
\end{equation}
which reproduces our discrete inverse Fourier transform formula Eq.(\ref{Eq:Fval}), since $\mathbf{n}$
  is an arbitrary integer.\\
  
\noindent That the sum over $\mathbf{k}$ is equal to zero if $\mathbf{m \neq n}$ can be seen
  by letting $\mathbf{l = n - m}$ and using $\mathbf{e^{k\,a} = (e^{a})^{k}}$ and recognising that
  we have a simple geometric sum.
We can also let Maxima confirm this as follows: 
\begin{myVerbatim1}
(%i1) declare([k,l,N],integer)$
(%i2) sum (exp(2*%pi*%i*k*l/N),k,0,N-1),simpsum;
                                 2 %i %pi l
                               %e           - 1
(%o2)                          ----------------
                                 2 %i %pi l
                                 ----------
                                     N
                               %e           - 1
(%i3) %,demoivre;
(%o3)                                  0
\end{myVerbatim1} 
In the last step the exponentials of complex arguments are converted into their
  trig function equivalents using
\begin{equation}
\mathbf{e^{i\,\boldsymbol{\theta}} = \boldsymbol{\cos\,\boldsymbol{\theta}} + i\,\boldsymbol{\sin \,\theta}}
\end{equation}
\subsection{Fast Fourier Transform References}
We have consulted the treatment of the fast Fourier transform in the following
  books:
\begin{enumerate}
\item \textbf{Mathematica for the Sciences, Richard E. Crandall, Addison-Wesley, 1991}
\item \textbf{A First Course in Computational Physics, Paul L. Davies, John Wiley, 1994}
\item \textbf{Applied Mathematica: Getting Started, Getting it Done, William T. Shaw and
  Jason Tigg, Addison-Wesley, 1994}
\end{enumerate}

\end{document}