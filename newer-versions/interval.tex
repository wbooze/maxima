;;;;;;;;;;;;;;
newer version in papers directory  July 2010

\documentclass{article}
\usepackage{fullpage}
\title{DRAFT: Interval arithmetic, Extended numbers and Computer Algebra Systems (CAS)}
\author{Richard J. Fateman\\
Computer Science Division\\
Electrical Engineering and Computer Sciences\\
University of California at Berkeley }
\begin{document}
\maketitle
\begin{abstract}

%May, June, July, 2006

Many ambitious computer algebra systems were initially designed in a
flush of enthusiasm, with the goal of automating {\em any} symbolic
mathematical manipulation ``correctly.'' Historically, this approach
resulted in programs that implicitly used certain identities to
simplify expressions. These identities, which very likely seemed {\em
universally} true to the programmers in the heat of writing the CAS,
(and often were true in well-known abstract algebraic domains) later
needed re-examination when such systems were extended for dealing with
previously unanticipated kinds of objects by extending ``generically''
the arithmetic or other operations.  For example, approximate floats
do not have the same properties as exact integers or rationals.
Complex numbers may strain a system designed for real-valued
variables.  In the situation examined here, we consider two categories
of ``extended'' numbers: $\infty$ or {\em undefined}, and real
intervals.  We comment on issues raised by these two troublesome
notions, how their introduction into a CAS may require a (sometimes
painful) reconsideration and redesign of parts of the program, and how
they are related.
\end{abstract}
\section {Introduction}

We discuss methods for dealing with intervals and extended numbers in
a CAS. We think the interval approach presented in section 2 is novel
compared to its usual treatment, partly because our approach uses some
computations which are relatively easy to program in a symbolic
system, but would require building up much more infrastructure if
attempted in the numerical context in which interval arithmetic {\em
usually} appears. Furthermore, the CAS user may be more inclined to
allow extra computation time if it allows for better results given
that for CAS, it is sometimes the case that results can be computed to
{\em arbitrarily high precision} or when conditions permit, {\em
exactly}. Naturally one would like computations to be done efficiently
when the demands are less stringent, but if the arithmetic on the
endpoints of an interval already involves non-traditional arithmetic
and subroutine calls, it seems plausible to endorse the use of yet
additional algorithms to keep the width of the interval down.  The
same kinds of arguments hold for extended ``number-like'' objects such
as $\infty$. 
%considered in section \ref{inf}. 
Whereas the IEEE-754 floating-point standard representations of
$\infty$ and undefined are constrained to be exceptional operands
fitting within the floating-point format, a CAS can expand upon the
notation-- using extra time and space-- if the user is seeking
somewhat better results.  We will explore this later, after
considering intervals in more detail.

To return to the notion of failing ``identities'' indicated in the
abstract, here are examples that might plausibly be built in to a CAS
simplifier: for any expression $x$, $x - x = 0$ and $x/x = 1$.  

When might this not be true?  When $x$ is $\infty$ or an IEEE
floating-point {\tt NaN} or an interval, say [-1,1], other rules apply.  Yet not
applying such common simplifications, a sequence of symbolic manipulations
may result in expressions that are unnecessarily complex
and unwieldy. A path must be found that allows only legal operations on extended
expressions.  

Let us use the term {\em non-nullable} for a value or pseudo-value $x$
for which $0 \times x$ cannot be replaced by $0$, $x-x$ cannot be
replaced by $0$, and $x/x$ cannot be replaced by $1$.  A pseudo-value
might be $\pm \infty$, (unsigned) $\infty$, one of the IEEE
floating-point operands {\tt nanSingle, nanDouble, nan,} or extra
items like {\em indeterminate, undefined,} and in many contexts, {\em
real intervals}.  How these are notated in a system can vary: it might
be convenient to use the notation $\infty$ or $1/0$ for the same
concept, and {\tt NaN} or {\tt undefined} might be $\bot$.  There
might also be such concepts as bounded but not definite, or ``no
number at all'' as would be the case for the empty intersection of two
numeric intervals. In some CAS, the designers have yielded to the
temptation to use intervals as sets.  For example, they may assert
that $\lim_{x->\infty}\sin(x) = [-1,1]$. The nominal interpretation of
an interval is that it is a representation of a particular point along
a real-line segment, but we can't tell which point.  The
interpretation of [-1,1] as this limit is different: the solution is
{\em not} any such point, but ``all the points''. The usual definition
of a limit requires that the distance between a point $-1<r<1$ and an
interval $I=[-1,1]$ should be 0, when in fact interval arithmetic
tells us the distance is not zero, but $[r-1,r+1]$. Perhaps one could
assert that it is not incorrect to return $[-\infty, \infty]$ for {\em
any computation of a real limit}, but this is unlikely to receive full
credit on a calculus exam.
\medskip

\section{A closer look at intervals}

Let us revisit that assertion that [-1,1]- [-1,1] might be zero, or
not.  There is a dependency issue, well known in interval arithmetic
(or ``reliable computing'') circles, in which the inability to
determine the ``source'' of an interval, i.e. whether two intervals
are correlated, leads to wider (and therefore less useful) results.

In our example, if the expression means the set \{$x-x ~|~ x \in
[-1,1] $\} the answer is 0.  If we mean \{ $x-y ~|~ x \in [-1,1]$ and
$y \in [-1,1]$ \}, then the answer is the interval [-2,2].

Is there a neat way of distinguishing the two cases?  How
about treating expressions this way. (In Macsyma) these might be notated
\begin{verbatim}
block([x:interval(-1,1)],  x-x)
\end{verbatim}
and
\begin{verbatim}
block([x:interval(-1,1),
        y:interval(-1,1)],  x-y)
\end{verbatim}

This locution is an attempt to force a kind of interchange of
simplification and evaluation, where in the first case the expression
{\tt x-x} would be simplified to zero and in the second case the
expression x-y would not be simplified, but just evaluated using
interval arithmetic rules\footnote{Without changes in Macsyma, which
currently does not have rules for arithmetic about intervals, it
doesn't work. They both return 0.}.  Another example which illustrates
the possibility of getting tighter bounds with additional work, is the
expression $\sin(x) \times \cos(x)$ which, on the face of it, for all
real $x$, seems to be in the interval [-1,1].  In fact, the expression is in
[-1/2,1/2], a result which becomes clear by observing that $\sin(x)
\times \cos(x) = \sin(2x)/2$.

A non-constructive definition of interval arithmetic can be phrased in
terms of constraints on a function.  Elucidating the situation of
dependency requires more than one variable, so let us start with two.
Then given a real-valued function of two real-valued arguments to
evaluate: $f(x,y)$ for $x,~y$ interval-valued between $[x_0,x_1]$ and
$[y_0,y_1]$ respectively, we return an interval $[z_0,z_1]$, where the
real-valued $z_0$ is the minimum of $f(x,y)$ with the constraints $x_0
<= x <=x_1$ and $y_0 <= y <=y_1$, and the corresponding maximum is
$z_1$.  While the interval $[z_0,z_1]$ then is the ``best'' answer, it
might not be possible to find it conveniently because we have lost
information during the calculation, or the minimization computation is
computationally too costly. We tend to accept the computation of some other, looser,
version: $[w_0,w_1] \subseteq [z_0,z_1]$. That is $w_0 <= z_0$ and$ w_1
>= z_1$.  We may also need to extend the notation to allow for
situations such as division by zero.
%%\begin{verbatim}
%%\end{verbatim}

Much effort can sometimes be expended in improving ``tightness'' of the
result, and this may be useful in the long run, since any slack in the result can be magnified by subsequent
operations.

Generally there is a tradeoff between width of a result interval and
effort taken to compute it.

In our assumptions about a CAS, we mean by ``interval()'' a program
which produces a new object, not shared.\footnote{Maple has a feature
making all compound objects which simplify to the same structure
isomorphic. Implementationally they are pointers to one copy in
memory. This is helpful in some processing, but not this one.  There
is sometimes a difference between holding two chocolate-chip cookies,
one in each hand, and holding just one cookie but with both hands.}

\section{An approach}
In this section we discuss a collection of approaches for computing
with {\em real intervals}. We claim no novelty in these
matters\footnote{Indeed, any time we thought we had a new idea in
interval arithmetic, we found we were simply rediscovering something
that had been previously identified as interesting (and often analyzed
in great detail).}.  Our perspective here is to briefly introduce
ideas to a reader who may not have seen these before, or has not seen the
alternative possibilities to 'naive' intervals.  We will not
thoroughly examine them. Our objective is to provide enough background
so that we can reasonably, in the following section, point out the
relationship with pseudo-values that occur in CAS.
\subsection{Intervals are Three-tuples}

We create a numeric interval of the real line by calling a function,
{\tt Ninterval}.  For example, for the interval between -1 and 2, we
evaluate a form like this $v=${\tt Ninterval(-1,2)}.
%(setf v (ri -1 2))
What this function constructs, rather than
our usual representation of an interval as a pair, e.g. [-1,2], is
an internal data structure with {\em three} slots. The
% (describe *)
%[-1,2] is a structure of type ri.  It has these slots:
% lo                 -1
% hi                 2
% exp                #(#:g478 0 1)  ;; note, g478 is a made-up index.
first two are the visible upper and lower limits, often denoted
${\underline x}$ and ${\overline x}$, and in our example, ${\underline
x}=-1$ and ${\overline x}= 2$.  These can be exact integer or rational
numbers, machine floating-point numbers, software ``bigfloats'', or
(we do not recommend this) literal parameters or expressions (say $a$
or $\exp(b-1)$). Before constructing the third slot, we must first
manufacture a new symbolic variable name\footnote{a counter is
incremented after each newly constructed name.}.  Here we denote that
variable by \#n for some integer n. The third slot is then an
algebraic expression involving that variable, initially just \#n.  The
exact representation used is specified in the implementation, and
might be a list of polynomial coefficients, a ``tree'' or some other
form that might be convenient in a CAS.

Our example $v$ might be described this way: {\tt[-1,2,
\#5]}. However, since we (humans) ordinarily do not need to see the
extra slot, programs ordinarily will not display it in the computer
output. For purposes of exposition in this section of this paper, we
will show them.

If we compute $u= v \times v$, we create the slots in $u$ as
{\tt[0,4,\#$5^2$].}  Note that if we had two independent
intervals that happened to have the same bounds, and multiplied them,
we would compute {\tt[1,2,\#5)* [-1,2,\#6]} as {\tt [-2,4,\#7]}. 
This is the interval [-2,4], which is wider than [0,4], and
where we have generated a new variable for the use of this interval.

Let us continue our example.
If we compute $u \times v$, we can notice a coincidence in indexes in
{\tt [-1,2,\#5) * [0,4,\#$5^2$]}. By realizing that we are
computing \#$5^3$,  the resulting interval is [-1,8], instead of the result of straight
multiplication of intervals,  which would give [-4,8].

Under certain conditions we can maintain this stream of computation,
and more generally recompute the expression in the third slot. To do
this we must not only represent the polynomial, but remember the
original value.  In this example we do this by keeping the original
\#5 around as long as there is a live ``accessible'' value that uses
it. There is a convenient mechanism that is built-in to Lisp for this
purpose, a ``weak hash table''.

Another optimization we can make is for quadratic polynomials. We can
rewrite $p=3x^2+2x+1$ as $q=3(x+1/3)^2+2/9)$. This is a ``single use
expression'' and provides tighter bounds as $p([-1.1,-1.0])$ is
$[1.8,2.63]$ but $q([-1.1,1.0])$ returns the tighter (in fact optimal!) $[2.0,
2.43].$ The order of computation here is based on ``completing the square''
but unfortunately does not generalize to higher order.
\subsection{Variations}
What are the options for that final slot?  We initially considered
just expressions of the form $r x$ for a scalar number $r$ or $rx^n$
for some integer $n$ but then realized that it would be simple to use
existing subroutines for symbolic mathematics to carry around a more
elaborate form.  We initially settled on a polynomial in (any) {\em one
variable}, in some encoded canonical form. {Given such a form, we can
feed it into a program for further analysis: for evaluating a
polynomial $P$ in one variable $x$ at an interval point, the tightest
bounds can be computed by considerably more elaborate computation than
just going through the adds and multiplies: locating suitable
intervals around the zeros of $P'(x)$, then evaluating $P$ at those
places to find relative extrema. Combined with $P$ at the endpoints
(absolute min/max), we can find sharp bounds subject only to the
representation of the endpoints.  Such a computation {\em could} be
worthwhile if the advantage of finding tighter bounds is more
important than the increased computational expense.}

Any expression that turns out to be more complicated than we wish to
deal with can be given a fresh name, and from that point forward it is
assumed to be independent of all previous variables.  Thus our current
program would never notice the correlation possibility that is
inherent in the identity $\sin(x) \times \cos(x) = \sin(2 x)/2$ since
each of the expressions $\sin()$ and $\cos()$ would be given different
(presumed to be independent) names.
\medskip

Where do we draw the line, though: what is too complicated? Can we in fact
use something more elaborate than polynomials? Indeed, there are
a host of variations on interval arithmetic: our invention of a
``polynomial'' slot is subsumed under any of a number of previously documented
innovations, many of which are reviewed by Arnold Neumaier, see
\verb|http://www.mat.univie.ac.at/~neum/papers.html#taylor|.  This paper
reviews so-called Taylor forms -- an extension which revolves around carrying
a truncated Taylor series with remainder, evaluated at the midpoint of
the interval. In certain calculations this additional information can
be effective in reducing the width of the intervals being
carried.  To produce this model, we could use the extra slot
effectly as an expression of a Taylor series, with remainder, for the
number being represented.  We could start our trigonometric
calculation by considering that we are evaluating $f(x)=\sin(x)$,
where $x=[a,b]$.  We then expand sin in a Taylor series around the
midpoint, $x=x_0=(a+b)/2$, and use an expression for the remainder (error).
Note that operations on Taylor series are similar to those on polynomials. The
major difference is that the Taylor method sets a global order, and
high-order terms are truncated, with the error term calculated separately as
another interval calculation.

%[[[Just refer to Berz paper in Reliable computing...

It is important to observe that the evaluation of the remainder term,
which can itself be elaborate, may have to be done carefully too, or
some blow-up is re-inserted.  Programs for evaluating derivatives can
be generated by ``Automatic Differentiation'' programs, but these must
also be evaluated in an interval fashion. When the error term looks
sufficiently well-behaved, ``naive'' interval arithmetic may do. The
net effect of tighter bounds is that some convergent procedures
(typically Newton iteration combined with interval bisection) can
sometimes converge with {\em far fewer iterations}. This accelerated
convergence may more than compensate for the more expensive operations
at each step, as shown in the references.
%]]]

Other possibilities include allowing the extra slot to contain  a
linear function of any number of interval values (more about this later). In this case a
linear-programming method could be used to find min and max.  If the
extra slot were allowed to be a multivariate polynomial, a method
based on cylindrical algebraic decomposition might be suitable.

An idea called ``Affine Arithmetic'' for intervals has also been
pursued in the recent literature on reliable computation.  In this
model, a quantity $x$ is represented as a first-degree (``affine'')
polynomial in the values it depends upon.  In particular, $$x=x_0+
x_1e_1+x_2e_2+ \cdots + x_k e_k$$ where the $\{x_i\}$ are given ``real
numbers'' and the $\{e_i\}$ are dummy variables, assumed uncorrelated
for calculation purposes, but each of whose values is known to be in
$[-1,1]$.  When possible the representation is maintained through
ordinary arithmetic, and when it is not, the operational routine must
find an alternative acceptable approximation, by introducing another
variable $e_{n}$ and positing it as uncorrelated.  Affine arithmetic
(AA) seems especially appealing when geometric objects are defined in
which the coordinates share a set of dummy variables.  This situation
produces centrally-symmetric convex polygons or ``zonotopes''.  In two
dimensions these can be visualized as ``lozenge'' shapes that are
smaller than the enclosing rectangular intervals that would appear
from naive intervals. There are benchmarks illustrating particularly
advantageous results in approximate computations for graphics: they
can provide closer adherence to lines or planes that are being
approximated. Not all computations ultimately benefit from AA:
it may be beneficial to switch between methods to
increase overall efficiency: AA may be better for small
intervals.  An extensive comparison of interval variations is given in
the paper by Martin et al.\cite{martin}.

%(a  +b e1) +( c +d e1) --> (a+c) + (b+d)e1.
%(a  +b e1) *( c +d e1) --> (ac) + (ad+bc)e1 +bd*e1^2 
% A*e1^2 ->   A/2 +A/2*e3

See for example,\\
 \verb|http://www.icc.unicamp.br/~stolfi/EXPORT/projects/affine-arith/Welcome.html|
or\\
\verb|http://www.tecgraf.puc-rio.br/~lhf/ftp/doc/oral/aa.pdf|
%http://www.ic.unicamp.br/~stolfi/EXPORT/projects/affine-arith/Welcome.html

Implementing AA on top of our structure is straightforward, and we
have done so up to addition and (non-trivial) multiplication, for
arithmetic combining scalars and intervals.  Yet another possibility
is to try to compute an algebraic ``single use expression'' (SUE) from
some set of variables. (see whitepaper on Sun Forte Fortran by
G.W. Walster\\ {\tt
http://developers.sun.com/prodtech/cc/products/archive/whitepapers/tech-interval-final.pdf}
).  This would transform an expression like $ax+bx$ where $a$ and $b$
are scalars, but $x$ is an interval, into $(a+b)x$, which may be a
tighter expression.

Similarly, for intervals $x$ and $y$, $A= x/(x+y)$ can be rewritten as
the SUE $B=1/ (1+y/x)$, although the acceptable domains for $x$ and
$y$ have now been changed. (Look for division by interval expressions
that might include zero!)  Walster suggests obtaining the answer by
computing $C = 1- (1/(1+x/y))$ and then returning their intersection,
$B \cap C$.

One approach for SUE is for us to establish a set of heuristics for
driving an algebraic expression into a more-nearly SUE form.  A good
algorithm to find an optimal ``sharp'' solution (that is, more
efficient than exhaustive enumeration and evaluation) would be
welcome, but tradeoffs such as whether a SUE version with respect to
$x$ is better than a SUE with respect to $y$, or some other variation
is yet better, cannot be computed at compile-time, since in general
the optimimum depends not only on the expression available at
compile-time, but on the particular {\em run-time} interval values associated
with $x$ and $y$.  {Note that given any collection of ``mathematically
equivalent'' expressions transformed at compiled-time into various
interval expressions, each version of an expression could be
numerically evaluated, and the best solution emerges by computing
their intersection.}  Such evaluations must be done while aware of the
possibility of the introduction of extraneous singularities; if any of
the expressions is non-singular at a point in the interval, then it
presumably reflects a valid mathematical result. A good expression might be
a Taylor series expansion at the midpoint of the interval. This cannot be
computed without knowing the interval endpoints.

We could attempt to use some existing (essentially black-box and
possibly heuristic) programs like Mathematica's {\tt Maximize} and
{\tt Minimize}.  Yet, even when they nominally work according to their
own specifications, the answers may not be satisfactory for interval
computations. Reliable computing requires strict inclusion: these
imply stronger requirements for directed rounding of endpoints than
typical numerical min/max programs.

In (see http://citeseer.ist.psu.edu/nedialkov03interval.html),
Interval Arithmetic, Affine Arithmetic, Taylor Series Methods: Why,
What Next? (2003) by Nedialko S. Nedialkov, Vladik Kreinovich, Scott
A. Starks, various methods are considered. 
A theoretical comparison can be summarized by an
optimality result: representation by polynomials is, in their presented
context, optimal.


\subsection{Other approaches to improved interval computation}
Other techniques, such as (repeated) bisection of intervals can be
used to attempt to narrow the result width. Our suite of programs
includes this conceptually simple technique: the result for $P(x)$
would be the smallest interval containing all the ranges of $P(x_1),
..., P(x_n)$ where $x$ has been subdivided.  That is, $x$ is the union
of the $x_i$; some criteria determine which of the subdivisions to
divide again in an attempt to narrow the interval. (In particular the subdivisions
containing the current maximum and minimum should be re-divided.)

This kind of interval arithmetic is more encapsulated and therefore
``hands-off'' compared to the usual numeric version.  It takes the
form of a functional programming model: new values may be created out
of components; but we do not allow user-level operations like ``change
the lower limit of variable $x=[{\underline x},~{\overline x}]$ to be
the number 43.'' Instead, we allow the creation of a new object which
can be re-assigned to the same name: $x=[43,~{\overline x}]$.  Old
values, if they are no longer accessible, will be collected, and their
memory locations re-used.

\section {Pseudo-values: Infinities and intervals}

At this point it may not seem that there is any direct connection
between intervals and extended numbers such as infinity. In fact,
the resemblance has more to do with the techniques to resolve
difficulties than the mathematical concepts.

Note that without intervals, the production and therefore occurrence
of infinities is limited to singularities (of which division by zero
is the most obvious, but not the sole example).  With intervals, the
exceptional treatment of zero must be extended in some way to {\em
intervals that contain zero}.  Once having produced such an object,
can one make it go away?  For example, can we assure that $1/(1/x)=x$
true, even if $x$ is an interval containing zero?

\subsection{Infinity as a symbol}
One of the problems of a CAS design is that there is a tradeoff
between convenience for human users and for computation. As an
example, computationally, we might construct a system in which an
infinite value appears only as a result of certain operations, and in
each case we can find some more informative alternative. Unfortunately,
if its origin is simply ``a human typed it into the system'' we do not
have an alternative to ``just compute with this.''

What can we propose?  We can keep $\infty - \infty$ from becoming zero
if, each time we generate (a new) $\infty$, we give it a unique
subscript or index, in a manner similar to the indexed variables used
with the intervals.  That is, $\infty_1$, $\infty_2$ ,..., $\infty_n$.
In Macsyma/Maxima this is rather simple to start:
{\tt (inf\_counter:0,
infinity():=(?incf(inf\_counter),Infty[inf\_counter]), nofix(infinity))}

After this, any mention of {\tt infinity} as in {\tt infinity+3} will generate
an expression like {\tt Infty[4]+3}. In \TeX\ this could be displayed
as
$\infty_4+3$.

This is not enough: A search through the Maxima code for all producers and
consumers of infinities may reveal locations that must be changed to
conform to this model. These occur principally in the {\tt limit}, {\tt sum}, and
{\tt definite integration} facilities.  Also saving such values off-line and reading
them back much re-create the proper values, shared when appropriate.

We need a new simplifier that is able to compute a kind of pessimistic
simplified form. For example, given the expression $\infty_2-
\infty_3$, {\em assuming that there are no occurrences of} $\infty_2$
or $\infty_3$ {\em anywhere accessible in the CAS}, can be simplified
by generating an undefined: {\tt und}. Such symbols are also
indexed. If there are other occurrences of those $\infty_k$ symbols elsewhere, it
is prudent to keep the expression uncombined in the hope that at some
later point one of those symbols will interact and cancel.  We typeset
the undefined quantity with the ``bottom'' symbol, for example,
$\bot_4$.
This gives us an opportunity to consider a case where we assign {\tt x :=
limit(1/x\^2,t->0,plus)} and thus x=$\infty_5$ , a specific value. Then
{\tt x-x} is $\infty_5-\infty_5$ , and indeed this can be simplified
to $0$ because this is the difference of two of ``the same $\infty$''.

Mathematica uses a notation for infinity which provides a direction in
the complex plane; $(1+i)\infty$ is effectively
DirectedInfinity($(1+i)/\sqrt{2}$) after normalization. There is
presumably only one infinity in each direction, and so rules combining
them must either be very conservative or sometimes wrong. This design
suggests that for each directed infinity generated, there should be a
subscript as well.

This simplifier must deal with any expression with one or more
infinite or undefined objects, possibly other forms such as
``indefinite'' (but bounded) and generally should come up with the
most informative and compact result possible.  For example, {\em in
the absence of any other occurrences of the same-indexed infinities,}
$\infty_2+\infty_3$ could be replaced with $\infty_4$ as a simpler
result\footnote{ To reiterate: it would be a mistake to make this
replacement if there were another expressionalgebraically dependent on
either of the two component $\infty$ objects, since the replace would
lose dependency information.}.  In similar circumstances (no
dependencies) there is no point in retaining the details of
$(\bot_4+1)/\bot_7$.  This should be replaced by (say) $\bot_8$.  An
expression involving such objects should be simplified as much as
possible, using standard rules for combination.
This is not necessarily trivial, and for rational functions
may be solved using a resultant technique devised by Neil Soiffer in
his MS project at Berkeley (1980). That work calculated the minimal
number of ``arbitrary constants' needed to provide the same degrees of
freedom as a given expression with perhaps many such constants.

\section{insert: algebra}
It is worth observing the properties that fail when we have $\bot$ and
$\infty$ without subscripts. (or in some cases, with subscripts).

Consider that at the base of the ordinary simplifier we assume that
the otherwise uninterpreted symbols $x$, $y$, $z$ etc., come from the properties
of integral domains, which include the postulates for commutative rings.

Commutative Rings:
\begin{enumerate}
\item %1.
 Closure:
 $a,b$ in R imply $a+b$ and $a\times b$ are in $R$.
\item %2.
 Uniqueness: $a=a'$ and $b=b'$ imply $a+b=a'+b'$ and $a\times b
 =a'\times b'$
\item %3. 
commutation:  $a+b=b+a;$ $a\times b=b\times a$.
\item %4. 

Association: $a+(b+c)=(a+b)+c$; $a\times(b\times c)=(a\times b)\times c$.
\item %5.
Distribution: $a\times (b+c)=a\times b+a\times c$.
\item %6.
Additive Identity (Zero). there is an element, 0, such that $a+0=a$ for all $a$ in $R$.
\item %7.
Multiplicative Identity (One). there is an element, 1, such that $a\times 1=a$ for all $a$ in $R$.
\item %8.
Additive Inverse: $a+x=0$ has a solution $x$ in $R$ for all $a$ in $R$.

integral domain adds

%9. 
\item Cancellation: if $c$ is not zero and $c \times a=c \times b$, then $a=b$.

Other property usually assumed for the symbols include {\em ordering.}

%10. 
\item Trichotomy:  $a>0$, $a<0$ or $a=0$.

%11. 
\item Transitivity: $a<b$ and $b<c$ implies $a<c$.

Here are some consequences:

%12. 
\item $a<b$ implies $a+c < b+c$.

%13. 
\item $a<b$ and $0<c$ imples $ ac<bc$.

%14. 
\item $a<b$, $a=b$, or $a>b$.

What about fields, reals?

%15. 
\item
A field is an integral domain in which each element $e$ except 0 has an
inverse, $e^{-1}$ such that $e\times e^(-1)=1$. (This seems simple enough, but has lots of consequences.

\end{enumerate}
%.........
How do elements $\infty$ and $\bot$ fail to satisfy these postulates?

We would like to know that, if we use rules such as $0\times x = 0$, $x/x=1$ and $x-x=0$, that 
these statements are true for
any possible legal assignment of values for $x$.

Since these are not true for certain values of $x$, what should we do?

Consider clarifying =. 

The usual way of providing inference rules for equational logic, includes the following:

Substitution: If $P$ is a theorem then so is $P[x\rightarrow E]$  where 
 $P[x\rightarrow E]$ means textual substition of expression $E$ for variable $x$ in $P$.

If $P=Q$ that is, $P$ and $Q$ are of the same type and are equal, then
$E[x\rightarrow P] = E[x\rightarrow Q] $

If $P=Q$ and $Q=R$ then $P=R$.

Some CAS essentially say that two expressions are ``='' under much
looser conditions.  For example 
\begin{enumerate}
\item[a] They are equal if, for {\em nearly every} substitution of
values into the variables, the expressions evaluate to equal
expressions.  Yes, this ``nearly every'' sounds just wrong, but the
argument is made by the authors of Mathematica that a ``generic
solution'' is acceptable if all exceptions to that solution are in a
space of lower dimensionality. Example: $x/x=1$ even if it is perhaps
false if $x=0$ or if $x$ is an interval containing zero.  Another
example: For all pairs of real numbers $\langle x,y\rangle$ in ${ R}^2$,
$(x^2-y^2)/(x-y)= x+y$ except along a particular line, namely
$x-y=0$).  This is not a kind of argument that builds confidence in
the capability of the program to produce ``guaranteed'' answers or
prove other programs correct\footnote{We understand why some people
advise against learning
mathematics from a physicist.}.
\item[b] Two floating-point single- or double-precision
results are equal if they are ``close enough'' (by some relative or
absolute error).
\item[c].Two approximate or imprecise numbers are ``close enough'':
In Mathematica we can construct a very imprecise number $n$,
 {\tt n=N[1,1]}, a number 1 with one digit of accuracy. Then
this equation is ``true'': $3+n=4+n$.
\item[d] Mathematica claims that two values each {\tt Infinity} are equal, yet the
difference of them is {\tt Indeterminate}. This can't be right.
% Other examples, by mixing intervals or vectors with scalars, can result
%in questionable ``equalities''. [specify?]
\end{enumerate}

How can we reconcile the different behaviors of a CAS resulting from
such designs with the intention that a CAS can ``prove'' mathematical
theorems?  If a proof that $f(x)-g(x)$ is zero (or is non-zero) ``for
all $x$'' (where ``prove'' is implemented as simplification), is
contradicted for some $f(c)-g(c)$ at a specific value $c$? Certainly
if $c$ is not in the domain of $x$ we can make some case that this
does not matter.  What, though, if it is in the domain?

\subsubsection{What is equality?}

If we approach this question operationally from the perspective of
a CAS design, the most plausible definition is to say that
two objects are equal if a simplifier program provides the same
canonical result, that is, lexicographically identical, and programmatically
indistinguishable\footnote
{We may (and will) forbid examining the representation at some abstraction
level, otherwise we would be faced with conundrums like declaring as unequal
two integers because they are stored in different memory locations, even if
they had the same value.}, for each side of the relation. 

In Lisp this is {\tt equal}. {Lisp has {\tt eq} to
represent equality as ``same storage location'', as well as several
specialized equalities for numbers ({\tt =}), strings ({\tt string=}),
and combinations ({\tt eql}).}  It is possible to have a canonical
expression which is, for example, {\tt 3*inf[3]+14*inf[7]}.  which
would be different from {\tt 14*inf[7]+3*inf[3]}; a canonical
simplifier would rearrange one or the other so they would look the
same. Mathematics texts are generally insufficiently pedantic to make
the subtle distinctions necessary for programmers.

We can require that testing for numerical equality, as between
floating-point numbers, use a different predicate. (Advice given to
novice programmers dealing with floating-point computation is to never
test for equality; this is overkill.)

Equality for intervals has several different variations-- possibly,
certainly, impossible.


\subsubsection{An alternative approach}
Assume that manipulation in the general simplifier is bound by rules
of computation in a commutative field with identities 0,1 with algebraic
extensions for each independent indeterminate with domain Real or Complex
numbers, and that in this field, $x/x=1$ 

Any computation that we do must first make sense in this domain. After
the computation is complete, a valuation can be done in some other
domain in which an algebraic expression is associated with an ordered
list of arithmetic commands that can be performed on elements in any
domain, including elements that are infinite, undefined, intervals, or
other objects.

Except for the discomfort caused by $ x^0=1$ and   $0^x=0$, this might work.

%***

consider 8, inverse. $\infty+x= 0$ has no solution. However, if we have a
indexed set of infinities, $\infty_1,~\infty_2,$ etc.  then each $\infty_k$ has
an inverse. Real intervals spanning zero arguably do not have inverses
unless the language of intervals is expanded to exterior intervals.

consider 9. If c=inf, then ca=cb becomes inf=inf, or und=und, which does not imply a=b. 
 If c=und, then ca=cb becomes und=und.

consider 10: und has an unknown sign.

consider 12, 13. the case c=inf or und does not work.

consider 16.  und does not have an inverse. inf does not, since inf*0=und, not 1.
Arguably, indexed inf, maybe.


How do intervals fail to satisfy these postulates?
First, what intervals are allowed?
[real,real], [real,inf], [und], [-inf, real], [-inf,inf]


\subsection*{Consider clarifying =}
 Two intervals are equal if a simplifier
program provides the same canonical result for each side of the
relation. This can happen if two intervals are each degenerate (one
point), and that point is the same numbers, or if the two intervals
match in upper/lower bounds and index expression.  In particular it is
insufficient to have equal endpoints alone.


consider 8, inverse. [-1,1] +x= 0 has no solution. However, if we have
a indexed set of intervals, [-1,1,p], then [1,-1,-p] may work.
an inverse.

consider 9. If c=inf, then ca=cb becomes inf=inf, or und=und, which does not imply a=b.  If c=und, then ca=cb becomes und=und.


........end insert

A list of a few plausible rules shows that intervals and the pseudo-values
of $\infty$ and $\bot$ relate to each other. We are here concerned
only with {\em real} values and pseudo-values, and we may need to have separate
rules for {\em complex} $\bot$. This might occur if one computes  $\sqrt{[-4,-1]}$ 
which has no real value, and could arguably be $\bot$  but perhaps should be $[1,2]i$
in the context of a CAS which knows about complex numbers.

\noindent
  What is $\sin(\bot)$?  [-1,1] if $\bot$ is {\em real}, otherwise $\bot$.\\
  What is $\exp(\bot)$?   Perhaps $[0,\infty]$ if $\bot$ is {\em real}?\\
  Is $[-\infty,\infty]$ the same as real $\bot$?\\
  Is $[\bot,\bot]$ meaningful?  Is it $\bot$?\\

Et cetera.
\medskip

A CAS designer has a choice to make: are these activities to be confined to the
numeric type hierarchy (an uncomfortable choice, I think, in which
case there may be both single- and double-float infinities, and the
choice of projective or affine models of the real line are relegated
to the numeric system), or are such pseudo-values used as an overlay
of a separate algebraic system that coexists more with the language of
variables and constants (like $\pi$ and $e$).  The latter seems more
plausible in context of a CAS.

An extended discussion of what choices to make is beyond the scope of the
present paper; a doctrinaire approach is likely to annoy at least some
CAS users. We defer further discussion to a future paper.
% [[ much more could be added here]]
\subsection{Big-Oh notation}
We hesitate to step into this muck, but one might be tempted to relate
this material to Big-Oh notation in asymptotic analysis. In particular
$O(x)-O(x)$ is not zero, but $O(x) \times O(x) is changed to O(x^2)$.
The common notation is further hindered by the misuse of the symbol
``='' which is usually considered an equivalence relation.  Yet
$T(n)=O(n^2)$ does not mean $O(n^2)=T(n)$; an expression which would
be rejected as illegitimate.  Proposals to remedy this notation by
using curly inequalities or set notation have not driven out the
traditional notation.

\subsection {Notes on Mathematica}

[[ particular +/- features from WRI]].

Mathematica has Indeterminate, DirectedInfinity[], and
DirectedInfinity[z] for an infinite quantity that is a positive real
multiple of the complex number z. Among the rules, 0*Indeterminate is
Indeterminate.

In the typical CAS, including
Mathematica, 0*f[x] is simplified to zero; this is
a mistake, in some sense if we later learn that  f[x]:=Indeterminate.

Where do we go with this?  A calculus of infinities for Limit
computations perhaps. A more cautious simplifier that doesn't say $x/x
\rightarrow 1$ but $x/x \rightarrow 1 ~~(x\ne 0)$?

It would be nice to know the rules being used, instead of having to
guess them; the treatment has evolved over various versions of Mathematica.

Or this section could be left out.

{\section{Implementation notes}

For an arithmetic package in which there are typed objects
that look explicitly like tagged numeric intervals, it is
not difficult in principle to attach methods for all the component-pair
combinations that may be of interest. For example, a two-argument ``+'' in
which each of the arguments is a scalar machine double-float can be
translated fairly easily into an operation.  Running through the plethora
of additional operations, e.g. ``+'' of a interval with an exact rational number,
or a symbolic expression like ``f(x+1)'' can be tiresome, and inheritance
hierarchies cannot make all the right decision automatically. Some operations
may not make sense outside the interval context. For example, there are many more ways
of ``comparing'' intervals: which may involve inclusion, overlap, disjointedness. These
can be phrased as ``certainly'' or ``possibly''  as in ``certainly greater than''.
etc.

Using the Common Lisp Object System (CLOS), we define methods for
\begin{verbatim}
(defmethod two-arg-+ ((u Ninterval)(v Ninterval)) ....)
\end{verbatim}
etc. for all argument type pairs.

This provides a default for each of the possible comparisons used to
evaluate $x+y$.  Additionally we can define how to combine an interval
with a complex number, a matrix, or a symbolic variable (or perhaps
more likely, refuse to do so).  In an object-oriented hierarchy, it
may be possible to place intervals in a lattice with respect to some
operations and use inheritance for method resolution, but in some
sense a response like ``{\tt no method available for two-arg-+ with
types Interval and Matrix}'' will wait in the background for
unimplemented combinations.  The structure for extending the ordinary
Lisp arithmetic to intervals (namely by overloading existing n-ary
operations like ``+'' and ``*'' by using two-argument versions of
these operations) is straightforward and can be done through
macro-expansion\footnote {see generic arithmetic (in lisp/generic
directory.}.

Using CLOS for keeping track of types does not resolve how to
integrate interval ``stuff'' into a CAS, exactly.  This is because we
must patch an existing system, say Maxima, so that it will not apply
identities prematurely that are incorrect for intervals.  To
illustrate how pervasive such errors can be, note that Intervals were
introduced in Mathematica in version 3, and even after many years, in
version 5.1, {\tt Interval[{a, b}] - Interval[{a, b}]} returns 0,
which is presumably false\footnote{I reported this bug and have been
told that it is fixed in the current development version.}.

Ideally we have localized a simplification that says $x + (-1)\times x$ = 
$0\times x$  = 0.

This would operate only in the case that $x$ is nullable.  In
particular $x$ being some version of infinity, or some
as-yet-unspecified interval, makes this simplification false.  We must
assert some criterion for $x$ before establishing this exception to
the general simplification. We must also process $x$ to some extent to
see if this criterion holds.  It is rather commonplace to try to
``short circuit'' processing by asserting that if any factor in a
product simplifies to zero, the rest of the product need not be
evaluated; this leads to inconsistencies.  Perhaps we want to patch
the ``rational function package'' or alternatively argue that these
programs are doing arithmetic in a field and we are violating one of
the assumptions, evaluating at an illegitimate point.  We could try
intercepting any explicit setting of a value to Inf or Ind, but that
would not prevent manipulation of arbitrary symbols, where only later
the value is proffered.

But getting back to the Maxima simplifier in particular: we must
change it so that $0$ times $x$ is sometimes computed as indefinite,
rather than short-circuit the simplification of $x$ entirely, and just
return $0$.

%This is done in simptimes, I think.}

\section{Acknowledgments}
A version of this paper was linked to a message on the newsgroup
sci.math.symbolic in June, 2006, which provoked a number of useful
comments. Thanks especially to Christopher Creutzig.

\begin{thebibliography}{99}

\bibitem {berz-taylor}Berz-Taylor model of intervals
Reliable Computing 4: 83-97.
\bibitem {martin}
Ralph Martin, Huahao Shou, Irina Voiculescu, Adrian Bowyer and Guojin
Wang, ``Comparison of interval methods for plotting algebraic curves,''
Computer Aided Geometric Design, Volume 19, Issue 7, July 2002,
Pages 553-587.
%(http://www.sciencedirect.com/science/article/B6TYN-46C8RF8-1/2/cb147638a60fd464363ae5db7f337ede)
%Author Keywords: Subdivision; Interval analysis; Range analysis;
%Algebraic curves
\bibitem {kahan}
W. Kahan,
``How Futile are Mindless Assessments of Roundoff
in Floating-Point Computation?''
\verb|http://http.cs.berkeley.edu/~wkahan/Mindless.pdf|

\bibitem {berz}Computation and Application of Taylor Polynomials
with Interval Remainder Bounds

http://citeseer.ist.psu.edu/cache/papers/cs/8022/http:zSzzSzbt.nscl.msu.eduzSzpubzSzpaperszSzrdaiczSzrdaic.pdf/berz98computation.pdf

\bibitem{markst} http://www.nd.edu/~markst/la2000/slides272f.pdf

\bibitem {mpfi}MPFI (Revol, Rouillier) INRIA
http://www.cs.utep.edu/interval-comp/interval.02/revo.pdf

\bibitem {walster}Sun Forte Fortran / G.W. Walster
http://developers.sun.com/prodtech/cc/products/archive/whitepapers/tech-interval-final.pdf
\end{thebibliography}
\end{document}

\section{where is the payoff?}
\begin{verbatim}


{do I believe this?  consider 

$x(x+3)$    vs $x^2+3x$ 

Let x = [-2,1]
x(x+3) is 
[-2,1]*[1,4] = [-8,4]

$x^2+3x$  is 
[0,4]+[-6,3] = [-6,7],   better, tighter..}
let x= [-2,1]
consider $x*(x+2)$         vs   $x^2+2*x$.
                     
[-2,1]*[0,3] = [-6,3]       [0,4] + [-4,2] = [-4,6]


A tight bound would be [-1,3]. On [-2,1],
 $f(w):=w^2+2*w$ has a  min at w=-1, where f(-1)=-1
and a max at 1, where f(1)=3. Our simple routine for {\em NOT}
using Horner's rule 
will not work for  (x+A)*(x+B). We need to make
more use of symbolic info: the polynomial roots.

\section {How far to extend, e.g. beyond univariate polynomials?}

\begin{verbatim}
Ninterval^NInterval -> no change, or change to Sinterval.

We suspect In the following, k is the next unused interval index.

exp(Ninterval(a,b,#n,f)) -> Ninterval(exp(a){round down},exp(b){round up},#k,1)
log(Ninterval(a,b,#n,f)) -> Ninterval(log(a) etc)
sin(Ninterval(a,b)#n,f)) -> find if rel max min occur in [a,b], check ends
                            Ninterval(low (round down if not -1), high (round up if not +1) #k, 1)


cos (etc)

must fill out the matrix of programs for all binary ops and unary ops
Ninterval OP  other
other     OP  Ninterval

* - * / expt
=, not=, >, >=, <=, <, included, overlap, disjoint.


perhaps consider reordering of operations if A op B op Ninterval?

\end{verbatim}





-------------------------------------old


Here we are principally interested in integrating into a CAS in some
useful fashion, the manipulation of real number intervals, even given
that they general violate rules of the algebraic system that we are
using for ordinary numbers and symbols.  Mathematica (now) does
Intervals in a reasonable fashion, in its latest versions.  Maple 7
does not.  Maple 7 requires the user to invoke evalr (range arithmetic
on objects that look like INTERVAL(a..b), or evalrC (complex range).
[check out newer Maple -- I only have Maple 7].  Maple seems to do its
calculations in such a way as to make the assumption that ALL
intervals with the same endpoints represent the same constant, and
thus INTERVAL(-1,1)-INTERVAL(-1,1) becomes zero, even before evalr
does its job.

One consequence of dealing with intervals is that it introduces some
variety in our thinking of the types of computational objects that
might occur in mathematical thinking outside the ring/field extensions
that --while well supported in CAS -- may not be sufficiently
expressive. Resolving what to do about intervals may help in thinking
about computing with other pseudo values Infinity - Infinity as
Indeterminate. \\ 
Indeterminate - Indeterminate as Indeterminate \\


Interestingly, in MMA 5.1 Interval[{a,b}] - Interval[{a,b}] comes out
 0.  Oh well.


......................

{\section{Other extended numbers}




\section {Symbolic endpoints for intervals}

Intervals with one or two symbolic endpoints, Sinterval(low,hi)  are generally
not worth attempting to simplify. They usually get unwieldy, fast.
Perhaps sums are worthwhile, but a product of
two Sintervals is rather clumsy, involving min and max.
An explicit product with a numeric constant may be worth
simplifying:  3*Sinterval(low,hi) --> Sinterval(3*low,3*hi).

An explicit product with a symbol x probably cannot be 
simplified, at least if x may be an interval, or not.

Mathematica is rather cautious about its equivalent to
Sinterval, and we could poke around and see what rules it uses.

Intervals with two numeric endpoints (real numbers: single or double
floats, rationals, bignums, bigfloats, QD, MPFR). We exclude 
Infinities, NaNs, from the possible endpoints.  Given the
context of a symbolic system, we would like to deal with UND
(etc) in a symbolic context, not as a sub-case of an interval.
There are, however, things that can be plausibly most easily
expressed as intervals. For example, 
numbers (or alternatively, a representation for ``any real number''
could be an interval from $-\infty$ to $infty$.)  Or by an abuse that
seems common, but can be very harmful, this notation
could be ``the set of all possible real numbers.''

Among the other plausible interval concepts:  an interval containing
no numbers at all, say [+0,-0] or an exterior interval [1,-1]: the complement of 
an ordinary interval [-1,1]. This can be done in an affine or a projective
model (refs);  it is possible to represent the union or intersection
of intervals, closed or open endpoints, etc.  


We expect that both endpoints should be the same type and precision,
and that given the usual usage of intervals these will be
floating-point.  However, rational operations on rational intervals
may be useful, especially since they provide exact (free of rounding)
intervals.}

\end{document}
example from Hongkun Liang and Mark Stadtherr, Univ Notre Dame
AIChe Annual Meeting session 272, Nov. 2000,

f(x) = x ln x for X = [0.3, 0.4]
Using interval arithmetic
F(X) = [0.3, 0.4] * [-1.2040,-0.9163] = [-0.482,-0.275]
 Using Taylor model (3rd order, x_0 = 0.35)
F(X) = -0:3674 - 0:04982 *(X - x0) + 1.4286 * (X - x0)^2
-1.3605 * (X - x0)^3 + [-1.25 * 10^-4, 4.86 * 10^-5]
= [-0.370,-0.361]
The exact range is [-0.368,-0.361]


try also
f = sin(2x) + sin(3x) + cos(4x) for X = [0:2; 0:5]

............. do the log example............
3rd order

%keepfloat:true; taylor(x*ln(x),x,0.35,4);

% -0.36744-0.04982*(X-0.35)+1.42857*(X-0.35)^2-1.36055*(X-0.35)^3+1.94364*(X-0.35)^4
$$ -0.36744-0.04982\,\left(x-0.35\right)+1.42857\,\left(x-0.35\right)%
 ^{2}-1.36055\,\left(x-0.35\right)^{3}+1.94364\,\left(x-0.35\right)^{%
 4}+\cdots  $$

good answer should be about [1/e, 0.3*ln(0.3)] = appx [-.36788,-.36119185].
note that the 4th derivative of $x \ln x$ is $2/x^3$. 
% (* 0.35 (log 0.35)) = -0.36743772
Now try evaluating taylor series expanded at ([0.3,0.4]-0.35) = [-0.05,0.05]
 to 3rd order. This is 
[-0.05,0.05]
$[-0.05,0.05]^2= [0, 0.0025]$
$[-0.05,0.05]^3= [-1.25\times 10^{-4}, 1.25\times 10^{-4}]$

(+ -0.36744 (* -0.04982 0.05) (* 1.42857 0.0025) (* -1.36055 1.25e-4))

%%= -0.36652964 with error term   $1/4! \times 2/\(xi)^3 \times(\xi-0.35)^4$

%% what is the largest value of 2/(xi)^3 for xi in [0.3,0.4]? It is at 0.3, and
%% the value is 74.07407407407406d0
%% that times (1/4!*(0.05)^3

%% which looks like , bounding,..
%%  1/4! *  2/(0.3)^3*0.5^4
%% (* 1/12 (expt 0.3 3)(expt 0.5 4)) = 1.4062e-4
%% (* 1/12 (expt 0.4 3)(expt 0.5 4)) = 3.3333e-4

 $1/12\times  ([0.3,0.4] -0.35])=$ %??

  $0.08333 \times [0.027, 0.064] \times [0, 6.25\times 10^{-4}]= 3.333\times 10^{-4}$

%%  (* 0.08333  0.064 6.25e-4) = 3.333\times 10^{-6}$
%    (* 0.08333  0.027 6.25e-4) = 1.4061937e-6

[-0.36744, -0.3673066]

%%??? not right. slide gives eror as [-1.25e-4, 4.86e-5]
%%total interval is [-0.370, -0.361]


%%% hummmmmm

{--- this next example is not so convincing --

$$f(x)= \sin(x_0)+\cos(x_0)(x-x_0)-sin(x_0)(x-x_0)^2/2+ \cdots$$ or
using a remainder formula,
where $\xi$ is some value between $x$ and $x_0$, and we (somewhat arbitrarily)
cut off the series after one term:

$$f(x)= \sin(x_0)+\cos(\xi)(\xi-x_0) = \sin(x_0)+ R_s$$

And similarly for $g(x)=\cos(x)$,

$$g(x)= \cos(x_0)-\sin(\xi)(\xi-x_0)= \cos(x_0) + R_c$$

If we then $f(x) \times g(x)$ we get  $\sin(x_0)\cos(x_0) + R_s \cos(x_0)+R_c\sin(x_0)$
 + additional remainder terms from the product of the Taylor series which have been
truncated.

Now consider $\sin(x)\cos(x)$ evaluated on $x=[-1,1]$. Since $\sin(x)$
is about [-0.84, 0.84], and $\cos(x)$ is about [0.54, 1], the naive interval
product is [-0.84, 0.84].  Next consider the Taylor model, using
$x_0=0.0$ as the midpoint.  Conveniently, $x_0=0,~\sin(x_0)=0,~
\cos(x_0)=1$, so in computing $f\times g$ almost all the terms simplify nicely
leaving us only to compute $ R_s =-\xi \cos(\xi)$, knowing only that
$\xi$ is some value in [-1,1].  A naive computation of this gives us
[-1,1]*[0.54,1] or [-1,1], but if we could do something slightly cleverer, maybe
we could get
[-0.54, 0.54]. This would be much better than [-.84, 84], and not much wider
than the tightest answer, [-0.5, 0.5].  

What if we evaluated on $x=[-1/2, 1/2]$ instead?
%-1/4,1/4
then $\sin(x)= [-0.48,48]$
% [-0.248, 0.248]
$\cos(x)= [0.87,1]$.
% [0.9699,1]
The naive interval product is  $[-0.48, 48]$.
% [-0.248, 0.248]
Then $ R_s =-\xi \cos(\xi)$, knowing only that
      $[-1/2,1/2] \times [0.87,1] = [-1/2,1/2]$
%      $[-0.25,0.25] \times [0.96,1] = [-1/4,1/4]$

%DO I BELIEVE THIS??
%CHECK IT OUT again.

%%Carrying another term in the Taylor series:
f(x)=sin(x0)+cos(x0)*(x-x0)   - sin(z)*(z-x0)^2/2
g(x)=cos(x0)-sin(x0)*(x-x0)   - cos(z)*(z-x0)^2/2

Evaluate f*g on X=[-1,1]
x0=0, sin(x0)=0, cos(x0)=1
f(X) = 0 + x              -sin(z)*z^2/2   
g(X) = 1 + 0              -cos(z)*z^2/2

f*g= x +                - x*(cos(z)+sin(z)) *z^2/2 

what can we say about this?  looks worse!?   z^2/2 = [0,1/2]. cos(z)+sin(z) =? [-.3,1.84]
}

..............
note from G W Walster in Forte doc
consider  
A= x/(x+y).   This can be rewritten as the SUE

B=1/ (1+y/x)

But the domains of A and B are not the same!

A is defined except when x+y=0.  
B is defined except when x=0  or  x=-y \ne 0

For intervals,  
0 \not\in X+Y,
0 \not\in 1+y/x  and 0 \not \in x.

What about 

C = 1-  (1/(1+x/y)) ?

different domain restrictions!

unless A and B return R* (undef.) this is sharp:


B \cap  C % intersection.

Walster asks
When is it legitimate to substitute
one interval expression for another? Is substitution possible when the enclosed
expressions have different domains? The analysis of this and other questions is
facilitated by introducing the closed R* -system in which the containment set of
an interval expression is always defined.

................
some local rules. do we assume inf[a] is positive or unsigned?
This is just writing off the top of my head, and must be checked.


ADDITION

IMPORTANT: assume non-unique inf[a], und[a]

UND RULES


n*und[a]+m*und[a] = (n+m)*und[a]  for n,m numbers
n*und[a]+m*und[a] = 0 if n+m=0   
 in particular, und[a]-und[a] = 0

n*und[a]+ {anything else} = no change

INF RULES
n*inf[a]+m*inf[a]= (m+n)*inf[a]
n*inf[a]+m*inf[a]= 0 if (m+n) = 0
  in particular inf[a]-inf[a] =0

n*inf[a]+m*inf[b] = no change?

Assume Q does not involve inf[a].
n*inf[a]+Q = no change?

IMPORTANT: assume UNIQUE inf[a], und[a]

und[a]*+{any} = und[a]

 n*inf[a] = inf[a] for n nonzero ; re-use symbol
          = und[new] for n zero

inf[a]+inf[b] = und[new]
inf[a]*inf[b] = inf[new] or nochange.

MULTIPLICATION

assume non-unique inf[a]

und[a]*und[a] = no change

und[a]/und[a] = 1  (or und???)

und[a]^n*und[a]^m = und[a]^(n+m)  for n,m numbers?
inf[a]^n*inf[a]^m = inf[a]^(n+m)  for n,m numbers?
 special case und[a]^0?  Should it be 1??
 special case inf[a]^0?  Should it be 1??
und[a]^n = und[new] .  what if n=0? plausibly 1??

und[a]* {anything} = und[a]];reuse

inf[a]*inf[b] = inf[c]

inf[a]^n*inf[b]^m = no change

n*inf[a]/m*inf[a]= n/m
n*inf[a]/m*inf[a]= und if m=0 . what if n=0?
n*inf[a]+m*inf[b] = no change?

inf[a]^n*B  with n negative, B free of inf[..] or und[..]  --> 0
Assume Q does not involve inf.
n*inf[a]+Q = no change?


If inf[a] is unique (no possible collapse with other occurrences of inf[a])
If inf[b] is unique (no possible collapse with other occurrences of inf[b])

 n*inf[a] = inf[a] for n nonzero
          = und[new] for n zero

inf[a]+inf[b] = und
inf[a]*inf[b] = inf[c] ? or inf[a]? or und? 

DIVISION RULES
There are no division rules because $A/B$ is the same as $AB^{-1}$. Look at rules for powers.	
